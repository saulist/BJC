\ShortTitle{Apocalypse}\BookTitle{Apocalypse}\BFont
\begin{multicols}{2}
\Chap{1}
\VerseOne{}La Révélation de Jésus Christ, que Dieu lui a donnée pour découvrir à ses serviteurs les choses qui doivent arriver bientôt, et qui les a fait connaître en les envoyant par son Ange à Jean son serviteur ;
\VS{2}Qui a annoncé la parole de Dieu, et le témoignage de Jésus-Christ, et toutes les choses qu'il a vues.
\VS{3}Bienheureux est celui qui lit, et ceux qui écoutent les paroles de cette prophétie, et qui gardent les choses qui y sont écrites : car le temps est proche.
\VS{4}Jean aux sept Eglises qui sont en Asie, que la grâce et la paix vous soient données de la part de celui QUI EST, QUI ÉTAIT, et QUI EST A VENIR, et de la part des sept Esprits qui sont devant son trône.
\VS{5}Et de la part de Jésus-Christ, qui est le témoin fidèle, le premier-né d'entre les morts, et le Prince des Rois de la terre.
\VS{6}A lui [dis-je], qui nous a aimés, et qui nous a lavés de nos péchés dans son sang, et nous a faits Rois et Sacrificateurs à Dieu son Père, à lui [soit] la gloire et la force aux siècles des siècles, Amen !
\VS{7}Voici il vient avec les nuées, et tout œil le verra, et ceux même qui l'ont percé ; et toutes les Tribus de la terre se lamenteront devant lui ; oui, Amen !
\VS{8}Je suis l'Alpha et l'Oméga, le commencement et la fin, dit le Seigneur, QUI EST, QUI ÉTAIT, et QUI EST A VENIR, le Tout-Puissant.
\VS{9}Moi Jean, [qui suis] aussi votre frère et qui participe à l'affliction, au règne, et à la patience de Jésus-Christ, j'étais en l'île appelée Patmos pour la parole de Dieu, et pour le témoignage de Jésus-Christ.
\VS{10}Or je fus [ravi] en esprit un jour de Dimanche, et j'entendis derrière moi une grande voix, comme [est le son] d'une trompette,
\VS{11}Qui disait :Je suis l'Alpha et l'Oméga, le premier et le dernier : Ecris dans un livre ce que tu vois, et envoie-le aux sept Eglises qui sont en Asie ; [savoir] à Ephèse, à Smyrne, à Pergame, à Thyatire, à Sardes, à Philadelphie, et à Laodicée.
\VS{12}Alors je me tournai pour voir [celui dont] la voix m'avait parlé, et m'étant tourné, je vis sept chandeliers d'or ;
\VS{13}Et au milieu des sept chandeliers d'or un [personnage] semblable à un homme, vêtu d'une longue robe, et ceint d'une ceinture d'or à l'endroit des mamelles.
\VS{14}Sa tête et ses cheveux [étaient] blancs comme de la laine blanche, et comme de la neige, et ses yeux [étaient] comme une flamme de feu.
\VS{15}Ses pieds [étaient] semblables à de l'airain très-luisant, comme s'ils eussent été embrasés dans une fournaise ; et sa voix était comme le bruit des grosses eaux.
\VS{16}Et il avait en sa main droite sept étoiles, et de sa bouche sortait une épée aiguë à deux tranchants, et son visage était semblable au soleil, quand il luit en sa force.
\VS{17}Et lorsque je l'eus vu, je tombai à ses pieds comme mort, et il mit sa main droite sur moi, en me disant : ne crains point, je suis le premier, et le dernier ;
\VS{18}Et je vis, mais j'ai été mort, et voici, je suis vivant aux siècles des siècles, Amen ! Et je tiens les clefs de l'enfer et de la mort.
\VS{19}Ecris les choses que tu as vues, celles qui sont [présentement], et celles qui doivent arriver ensuite.
\VS{20}Le mystère des sept étoiles que tu as vues en ma main droite, et les sept chandeliers d'or. Les sept étoiles sont les Anges des sept Eglises ; et les sept chandeliers que tu as vus sont les sept Eglises.
\Chap{2}
\VerseOne{}Ecris à l'Ange de l'Eglise d'Ephèse : celui qui tient les sept étoiles en sa main droite, et qui marche au milieu des sept chandeliers d'or, dit ces choses :
\VS{2}Je connais tes œuvres, ton travail et ta patience, et [je sais] que tu ne peux souffrir les méchants, et que tu as éprouvé ceux qui se disent être apôtres, et ne le sont point, et que tu les as trouvés menteurs.
\VS{3}Et que tu as souffert, et que tu as eu patience, et que tu as travaillé pour mon Nom, et que tu ne t'es point lassé.
\VS{4}Mais j'ai [quelque chose] contre toi, c'est que tu as abandonné ta première charité.
\VS{5}C'est pourquoi souviens-toi d'où tu es déchu, et t'en repens, et fais les premières œuvres ; autrement je viendrai à toi bientôt, et j'ôterai ton chandelier de son lieu, si tu ne te repens.
\VS{6}Mais pourtant tu as ceci [de bon] que tu hais les actions des Nicolaïtes, lesquelles je hais moi aussi.
\VS{7}Que celui qui a des oreilles, écoute ce que l'Esprit dit aux Eglises. A celui qui vaincra je lui donnerai à manger de l'arbre de vie, qui est au milieu du paradis de Dieu.
\VS{8}Ecris aussi à l'Ange de l'Eglise de Smyrne : Le premier et le dernier, qui a été mort, et qui est retourné en vie, dit ces choses :
\VS{9}Je connais tes œuvres, ton affliction et ta pauvreté (mais tu es riche), et le blasphème de ceux qui se disent être Juifs, et qui ne le sont point, mais [qui sont] la Synagogue de satan.
\VS{10}Ne crains rien des choses que tu as à souffrir. Voici, il arrivera que le Démon mettra quelques-uns de vous en prison, afin que vous soyez éprouvés ; et vous aurez une affliction de dix jours. Sois fidèle jusques à la mort, et je te donnerai la couronne de vie.
\VS{11}Que celui qui a des oreilles, écoute ce que l'Esprit dit aux Eglises. Celui qui vaincra sera mis à couvert de la mort seconde.
\VS{12}Ecris aussi à l'Ange de l'Eglise de Pergame : Celui qui a l'épée aiguë à deux tranchants, dit ces choses.
\VS{13}Je connais tes œuvres, et où tu habites, [savoir] là où est le siège de satan, et que cependant tu retiens mon Nom, et que tu n'as point renoncé ma foi, non pas même lorsqu'Antipas, mon fidèle martyr, a été mis à mort entre vous, là où satan habite.
\VS{14}Mais j'ai quelque peu de chose contre toi : c'est que tu en as là qui retiennent la doctrine de Balaam, lequel enseignait Balac à mettre un scandale devant les enfants d'Israël, afin qu'ils mangeassent des choses sacrifiées aux idoles, et qu'ils se livrassent à la fornication.
\VS{15}Ainsi tu en as, toi aussi, qui retiennent la doctrine des Nicolaïtes ; ce que je hais.
\VS{16}Repens-toi : autrement je viendrai à toi bientôt ; et je combattrai contre eux par l'épée de ma bouche.
\VS{17}Que celui qui a des oreilles, écoute ce que l'Esprit dit aux Eglises. A celui qui vaincra je lui donnerai à manger de la manne qui est cachée, et je lui donnerai un caillou blanc, et sur [ce] caillou sera écrit un nouveau nom, que nul ne connaît, sinon celui qui le reçoit.
\VS{18}Ecris aussi à l'Ange de l'Eglise de Thyatire : Le Fils de Dieu, qui a ses yeux comme une flamme de feu, et dont les pieds sont semblables à de l'airain très luisant, dit ces choses.
\VS{19}Je connais tes œuvres, [ta] charité, ton ministère, ta foi, ta patience, et que tes dernières œuvres surpassent les premières ;
\VS{20}Mais j'ai quelque peu de chose contre toi : c'est que tu souffres que [cette] femme Jézabel, qui se dit prophétesse, enseigne, et qu'elle séduise mes serviteurs, pour les porter à la fornication, et pour [leur faire] manger des choses sacrifiées aux idoles.
\VS{21}Et je lui ai donné du temps, afin qu'elle se repentit de sa prostitution ; mais elle ne s'est point repentie.
\VS{22}Voici, je vais la réduire au lit, et [mettre] dans une grande affliction ceux qui commettent adultère avec elle, s'ils ne se repentent de leurs œuvres ;
\VS{23}Et je ferai mourir de mort ses enfants ; et toutes les Eglises connaîtront que je suis celui qui sonde les reins et les cœurs ; et je rendrai à chacun de vous selon ses œuvres.
\VS{24}Mais je vous dis à vous et aux autres qui sont à Thyatire, à tous ceux qui n'ont point cette doctrine, et qui n'ont point connu les profondeurs de satan, comme ils parlent, que je ne mettrai point sur vous d'autre charge.
\VS{25}Mais retenez ce que vous avez, jusqu'à ce que je vienne.
\VS{26}Car à celui qui aura vaincu, et qui aura gardé mes œuvres jusqu'à la fin, je lui donnerai puissance sur les nations :
\VS{27}Et il les gouvernera avec une verge de fer, et elles seront brisées comme les vaisseaux d'un potier, selon que je l'ai aussi reçu de mon Père.
\VS{28}Et je lui donnerai l'étoile du matin.
\VS{29}Que celui qui a des oreilles, écoute ce que l'Esprit dit aux Eglises.
\Chap{3}
\VerseOne{}Ecris aussi à l'Ange de l'Eglise de Sardes : Celui qui a les sept Esprits de Dieu, et les sept étoiles, dit ces choses : Je connais tes œuvres ; c'est que tu as le bruit de vivre, mais tu es mort.
\VS{2}Sois vigilant, et confirme le reste qui s'en va mourir ; car je n'ai point trouvé tes œuvres parfaites devant Dieu.
\VS{3}Souviens-toi donc des choses que tu as reçues et entendues, et garde-les, et te repens, mais si tu ne veilles pas, je viendrai contre toi comme le larron, et tu ne sauras point à quelle heure je viendrai contre toi.
\VS{4}[Toutefois] tu as quelque peu de personnes aussi à Sardes, qui n'ont point souillé leurs vêtements, et qui marcheront avec moi en vêtements blancs, car ils en sont dignes.
\VS{5}Celui qui vaincra, sera vêtu de vêtements blancs, et je n'effacerai point son nom du Livre de vie, mais je confesserai son nom devant mon Père, et devant ses Anges.
\VS{6}Que celui qui a des oreilles, écoute ce que l'Esprit dit aux Eglises.
\VS{7}Ecris aussi à l'Ange de l'Eglise de Philadelphie : le Saint et le Véritable, qui a la clef de David, qui ouvre, et nul ne ferme ; qui ferme, et nul n'ouvre, dit ces choses.
\VS{8}Je connais tes œuvres : voici, je t'ai ouvert une porte, et personne ne la peut fermer ; parce que tu as un peu de force, que tu as gardé ma parole, et que tu n'as point renoncé mon Nom.
\VS{9}Voici, je ferai venir ceux dela Synagogue de satan qui se disent Juifs, et ne le sont point, mais mentent ; voici, [dis-je], je les ferai venir et se prosterner à tes pieds, et ils connaîtront que je t'aime.
\VS{10}Parce que tu as gardé la parole de ma patience, je te garderai aussi de l'heure de la tentation qui doit arriver dans tout le monde, pour éprouver ceux qui habitent sur la terre.
\VS{11}Voici, je viens bientôt ; tiens ferme ce que tu as, afin que personne ne t'enlève ta couronne.
\VS{12}Celui qui vaincra, je le ferai être une colonne dans le Temple de mon Dieu, et il n'en sortira plus ; et j'écrirai sur lui le Nom de mon Dieu, et le nom de la cité de mon Dieu, qui est la nouvelle Jérusalem, laquelle descend du Ciel de devers mon Dieu, et mon nouveau Nom.
\VS{13}Que celui qui a des oreilles, écoute ce que l'Esprit dit aux Eglises.
\VS{14}Ecris aussi à l'Ange de l'Eglise de Laodicée : l'Amen, le témoin fidèle et véritable, le commencement de la créature de Dieu, dit ces choses.
\VS{15}Je connais tes œuvres, c'est que tu n'es ni froid, ni bouillant ; ô si tu étais ou froid, ou bouillant !
\VS{16}Parce donc que tu es tiède, et que tu n'es ni froid, ni bouillant, je te vomirai de ma bouche.
\VS{17}Car tu dis : je suis riche, et je suis dans l'abondance, et je n'ai besoin de rien ; mais tu ne connais pas que tu es malheureux, misérable, pauvre, aveugle et nu.
\VS{18}Je te conseille d'acheter de moi de l'or éprouvé par le feu, afin que tu deviennes riche ; et des vêtements blancs, afin que tu sois vêtu, et que la honte de ta nudité ne paraisse point ; et d'oindre tes yeux de collyre, afin que tu voies.
\VS{19}Je reprends et châtie tous ceux que j'aime ; aie du zèle, et te repens.
\VS{20}Voici, je me tiens à la porte, et je frappe : si quelqu'un entend ma voix, et m'ouvre la porte, j'entrerai chez lui, je souperai avec lui, et lui avec moi.
\VS{21}Celui qui vaincra, je le ferai asseoir avec moi sur mon trône, ainsi que j'ai vaincu, et je suis assis avec mon Père dans son trône.
\VS{22}Que celui qui a des oreilles, écoute ce que l'Esprit dit aux Eglises.
\Chap{4}
\VerseOne{}Après ces choses je regardai, et voici une porte fut ouverte au Ciel ; et la première voix que j'avais ouïe comme d'une trompette, et qui parlait avec moi, [me] dit : monte ici, et je te montrerai les choses qui doivent arriver à l'avenir.
\VS{2}Et sur-le-champ je fus ravi en esprit : et voici, un trône était posé au Ciel, et quelqu'un était assis sur le trône.
\VS{3}Et celui qui y était assis, paraissait semblable à une pierre de jaspe, et de sardoine ; et autour du trône paraissait un arc-en-ciel, semblable à une émeraude.
\VS{4}Et il y avait autour du trône vingt-quatre sièges ; et je vis sur les sièges vingt-quatre Anciens assis, vêtus d'habillements blancs, et ayant sur leurs têtes des couronnes d'or.
\VS{5}Et du trône sortaient des éclairs et des tonnerres, et des voix ; et il y avait devant le trône sept lampes de feu ardentes, qui sont les sept Esprits de Dieu.
\VS{6}Et au devant du trône il y avait une mer de verre, semblable à du cristal ; et au milieu du trône et autour du trône quatre animaux, pleins d'yeux devant et derrière.
\VS{7}Et le premier animal [était] semblable à un lion ; le second animal, [était] semblable à un veau ; le troisième animal avait la face comme un homme ; et le quatrième animal était semblable à un aigle qui vole.
\VS{8}Et les quatre animaux avaient chacun six ailes à l'entour ; et par dedans ils [étaient] pleins d'yeux ; et ils ne cessent point de dire jour et nuit : Saint ! Saint ! Saint ! le Seigneur Dieu Tout-puissant, QUI ÉTAIT, QUI EST, et QUI EST A VENIR.
\VS{9}Or quand les animaux rendaient gloire et honneur et des actions de grâces à celui qui était assis sur le trône, à celui qui est vivant aux siècles des siècles,
\VS{10}Les vingt-quatre Anciens se prosternaient devant celui qui était assis sur le trône, et adoraient celui qui est vivant aux siècles des siècles, et ils jetaient leurs couronnes devant le trône, en disant :
\VS{11}Seigneur, tu es digne de recevoir gloire, honneur et puissance ; car tu as créé toutes choses ; c'est par ta volonté qu'elles existent et qu'elles ont été créées.
\Chap{5}
\VerseOne{}Puis je vis dans la main droite de celui qui était assis sur le trône, un Livre écrit dedans et dehors, scellé de sept sceaux.
\VS{2}Je vis aussi un Ange [remarquable] par sa force, qui criait à haute voix : qui est-ce qui est digne d'ouvrir le Livre, et d'en délier les sceaux ?
\VS{3}Mais nul ne pouvait ni dans le Ciel, ni sur la terre, ni au-dessous de la terre ouvrir le Livre, ni le regarder.
\VS{4}Et je pleurais fort, parce que personne n'était trouvé digne d'ouvrir le Livre, ni de le lire, ni de le regarder.
\VS{5}Et un des Anciens me dit : ne pleure point ; voici, le Lion qui est de la Tribu de Juda, la racine de David, a vaincu pour ouvrir le Livre, et pour en délier les sept sceaux.
\VS{6}Et je regardai, et voici il y avait au milieu du trône et des quatre animaux, et au milieu des Anciens, un agneau qui se tenait là comme mis à mort, ayant sept cornes, et sept yeux, qui sont les sept Esprits de Dieu, envoyés par toute la terre.
\VS{7}Et il vint, et prit le Livre de la main droite de celui qui était assis sur le trône.
\VS{8}Et quand il eut pris le Livre, les quatre animaux et les vingt-quatre Anciens se prosternèrent devant l'agneau, ayant chacun des harpes et des fioles d'or, pleines de parfums, qui sont les prières des Saints.
\VS{9}Et ils chantaient un nouveau cantique, en disant : tu es digne de prendre le Livre, et d'en ouvrir les sceaux ; car tu as été mis à mort, et tu nous as rachetés à Dieu par ton sang, de toute Tribu, Langue, peuple, et nation ;
\VS{10}Et tu nous as faits Rois et Sacrificateurs à notre Dieu ; et nous régnerons sur la terre.
\VS{11}Puis je regardai, et j'entendis la voix de plusieurs Anges autour du trône et des Anciens, et leur nombre était de plusieurs millions.
\VS{12}Et ils disaient à haute voix : L'agneau qui a été mis à mort est digne de recevoir puissance, richesses, sagesse, force, honneur, gloire et louange.
\VS{13}J'entendis aussi toutes les créatures qui sont au Ciel, et en la terre, et sous la terre, et dans la mer, et toutes les choses qui y sont, disant : à celui qui est assis sur le trône, et à l'agneau, soit louange, honneur, gloire, et force, aux siècles des siècles !
\VS{14}Et les quatre animaux disaient : Amen ! Et les vingt-quatre Anciens se prosternèrent et adorèrent [celui qui est] vivant aux siècles des siècles.
\Chap{6}
\VerseOne{}Et quand l'agneau eut ouvert l'un des sceaux, je regardai, et j'entendis l'un des quatre animaux, qui disait, comme avec une voix de tonnerre : Viens, et vois.
\VS{2}Et je regardai, et je vis un cheval blanc ; et celui qui était monté dessus avait un arc, et il lui fut donné une couronne ; et il sortit victorieux, et afin de vaincre.
\VS{3}Et quand il eut ouvert le second sceau, j'entendis le second animal, qui disait : Viens, et vois.
\VS{4}Et il sortit un autre cheval, qui était roux ; et il fut donné à celui qui était monté dessus, de pouvoir ôter la paix de la terre, afin qu'on se tue l'un l'autre ; et il lui fut donné une grande épée.
\VS{5}Et quand il eut ouvert le troisième sceau, j'entendis le troisième animal qui disait : Viens, et vois ; et je regardai, et je vis un cheval noir, et celui qui était monté dessus avait une balance en sa main.
\VS{6}Et j'entendis au milieu des quatre animaux une voix qui disait : Le chenix de froment pour un denier, et les trois chenix d'orge pour un denier ; mais ne nuis point au vin, ni à l'huile.
\VS{7}Et quand il eut ouvert le quatrième sceau, j'entendis la voix du quatrième animal, qui disait : Viens, et vois.
\VS{8}Et je regardai, et je vis un cheval fauve ; et celui qui était monté dessus avait nom la Mort, et l'Enfer suivait après lui ; et il leur fut donné puissance sur la quatrième partie de la terre, pour tuer avec l'épée, par la famine, par la mortalité, et par les bêtes sauvages de la terre.
\VS{9}Et quand il eut ouvert le cinquième sceau, je vis sous l'autel les âmes de ceux qui avaient été tués pour la parole de Dieu, et pour le témoignage qu'ils avaient maintenu.
\VS{10}Et elles criaient à haute voix, disant : Jusqu'à quand, Seigneur, qui es saint et véritable, ne juges-tu point, et ne venges-tu point notre sang de ceux qui habitent sur la terre ?
\VS{11}Et il leur fut donné à chacun des robes blanches, et il leur fut dit qu'ils se reposassent encore un peu de temps, jusqu'à ce que le nombre de leurs compagnons de service, et de leurs frères qui doivent être mis à mort comme eux, soit complet.
\VS{12}Et je regardai quand il eut ouvert le sixième sceau, et voici, il se fit un grand tremblement de terre, et le soleil devint noir comme un sac fait de poil, et la lune devint toute comme du sang.
\VS{13}Et les étoiles du ciel tombèrent sur la terre, comme lorsque le figuier étant agité par un grand vent, laisse tomber ses figues [encore] vertes.
\VS{14}Et le ciel se retira comme un Livre qu'on roule ; et toutes les montagnes, et les îles furent remuées de leurs places.
\VS{15}Et les Rois de la terre, les Princes, les riches, les capitaines, les puissants, tout esclave, et tout [homme] libre se cachèrent dans les cavernes, et entre les rochers des montagnes.
\VS{16}Et ils disaient aux montagnes et aux rochers : tombez sur nous, et cachez-nous de devant la face de celui qui est assis sur le trône, et de devant la colère de l'agneau ;
\VS{17}Car la grande journée de sa colère est venue ; et qui est-ce qui pourra subsister ?
\Chap{7}
\VerseOne{}Après cela je vis quatre Anges qui se tenaient aux quatre coins de la terre, et qui retenaient les quatre vents de la terre, afin qu'aucun vent ne soufflât sur la terre, ni sur la mer, ni sur aucun arbre.
\VS{2}Puis je vis un autre Ange qui montait du côté de l'Orient, tenant le sceau du Dieu vivant, et il cria à haute voix aux quatre Anges qui avaient eu ordre de nuire à la terre, et à la mer,
\VS{3}[Et leur] dit : Ne nuisez point à la terre, ni à la mer, ni aux arbres, jusqu'à ce que nous ayons marqué les serviteurs de notre Dieu sur leurs fronts.
\VS{4}Et j'entendis que le nombre des marqués était de cent quarante-quatre mille, qui furent marqués de toutes les Tribus des enfants d'Israël.
\VS{5}[Savoir] de la Tribu de Juda, douze mille marqués ; de la Tribu de Ruben, douze mille marqués ; de la Tribu de Gad, douze mille marqués ;
\VS{6}De la Tribu d'Aser, douze mille marqués ; de la Tribu de Nephthali, douze mille marqués ; de la Tribu de Manassé, douze mille marqués ;
\VS{7}De la Tribu de Siméon, douze mille marqués ; de la Tribu de Lévi, douze mille marqués ; de la Tribu d'Issachar, douze mille marqués ;
\VS{8}De la Tribu de Zabulon, douze mille marqués ; de la Tribu de Joseph, douze mille marqués ; de la Tribu de Benjamin, douze mille marqués.
\VS{9}Après cela, je regardai, et voici une grande multitude [de gens], que personne ne pouvait compter, de toutes nations, Tribus, peuples et Langues, lesquels se tenaient devant le trône, et en la présence de l'agneau, vêtus de longues robes blanches, et ayant des palmes en leurs mains ;
\VS{10}Et ils criaient à haute voix, en disant : Le salut est de notre Dieu, qui est assis sur le trône, et de l'agneau.
\VS{11}Et tous les Anges se tenaient autour du trône, et des Anciens, et des quatre animaux, et ils se prosternèrent devant le trône sur leurs faces, et adorèrent Dieu,
\VS{12}En disant : Amen ! louange, gloire, sagesse, actions de grâces, honneur, puissance et force soient à notre Dieu, aux siècles des siècles, Amen !
\VS{13}Alors un des Anciens prit la parole, et me dit : ceux-ci, qui sont vêtus de longues robes blanches, qui sont-ils, et d'où sont-ils venus ?
\VS{14}Et je lui dis : Seigneur, tu le sais. Et il me dit : Ce sont ceux qui sont venus de la grande tribulation, et qui ont lavé et blanchi leurs longues robes dans le sang de l'agneau.
\VS{15}C'est pourquoi ils sont devant le trône de Dieu, et ils le servent jour et nuit dans son Temple ; et celui qui est assis sur le trône habitera avec eux.
\VS{16}Ils n'auront plus de faim, ni de soif, et le soleil ne frappera plus sur eux, ni aucune chaleur.
\VS{17}Car l'agneau qui est au milieu du trône les paîtra, et les conduira aux vives fontaines des eaux ; et Dieu essuiera toutes les larmes de leurs yeux.
\Chap{8}
\VerseOne{}Et quand il eut ouvert le septième sceau, il se fit un silence au ciel d'environ une demie-heure.
\VS{2}Et je vis les sept Anges qui assistent devant Dieu, auxquels furent données sept trompettes.
\VS{3}Et un autre Ange vint, et se tint devant l'autel, ayant un encensoir d'or, et plusieurs parfums lui furent donnés pour offrir avec les prières de tous les Saints, sur l'autel d'or qui est devant le trône.
\VS{4}Et la fumée des parfums avec les prières des Saints monta de la main de l'Ange devant Dieu.
\VS{5}Puis l'Ange prit l'encensoir, et l'ayant rempli du feu de l'autel, il le jeta en la terre ; et il se fit des tonnerres, des voix, des éclairs, et un tremblement de terre.
\VS{6}Alors les sept Anges qui avaient les sept trompettes, se préparèrent pour sonner des trompettes.
\VS{7}Et le premier Ange sonna de la trompette, et il se fit de la grêle et du feu, mêlés de sang, qui furent jetés en la terre ; et la troisième partie des arbres fut brûlée, et toute herbe verte aussi fut brûlée.
\VS{8}Et le second Ange sonna de la trompette ; et [je vis] comme une grande montagne ardente de feu, qui fut jetée en la mer ; et la troisième partie de la mer devint du sang.
\VS{9}Et la troisième partie des créatures vivantes qui [étaient] en la mer, mourut ; et la troisième partie des navires périt.
\VS{10}Et le troisième Ange sonna de la trompette, et il tomba du ciel une grande étoile ardente comme un flambeau, et elle tomba sur la troisième partie des fleuves, et dans les fontaines des eaux.
\VS{11}Le nom de l'étoile est Absinthe ; et la troisième partie des eaux devint absinthe, et plusieurs des hommes moururent par les eaux, à cause qu'elles étaient devenues amères.
\VS{12}Puis le quatrième Ange sonna de la trompette ; et la troisième partie du soleil fut frappée, et la troisième partie aussi de la lune, et la troisième partie des étoiles, de sorte que la troisième partie en fut obscurcie ; et la troisième partie du jour fut privée de la lumière, et [la troisième partie] de la nuit fut tout de même sans clarté.
\VS{13}Alors je regardai, et j'entendis un Ange qui volait par le milieu du ciel, et qui disait à haute voix : Malheur ! malheur ! malheur ! aux habitants de la terre à cause du son des trompettes des trois autres Anges qui doivent sonner de la trompette.
\Chap{9}
\VerseOne{}Alors le cinquième Ange sonna de la trompette, et je vis une étoile qui tomba du ciel en la terre, et la clef du puits de l'abîme lui fut donnée.
\VS{2}Et il ouvrit le puits de l'abîme ; et une fumée monta du puits comme la fumée d'une grande fournaise ; et le soleil et l'air furent obscurcis de la fumée du puits.
\VS{3}Et de la fumée du puits il sortit des sauterelles [qui se répandirent] par la terre, et il leur fut donné une puissance semblable à la puissance qu'ont les scorpions de la terre.
\VS{4}Et il leur fut dit, qu'elles ne nuisissent point à l'herbe de la terre, ni à aucune verdure, ni à aucun arbre, mais seulement aux hommes qui n'ont point la marque de Dieu sur leurs fronts.
\VS{5}Et il leur fut permis non de les tuer, mais de les tourmenter durant cinq mois ; et leurs tourments sont semblables aux tourments que donne le scorpion quand il frappe l'homme.
\VS{6}Et en ces jours-là les hommes chercheront la mort, mais ils ne la trouveront point ; et ils désireront de mourir, mais la mort s'enfuira d'eux.
\VS{7}Or la forme des sauterelles était semblable à des chevaux préparés pour la bataille, et sur leurs têtes il y avait comme des couronnes semblables à de l'or, et leurs faces étaient comme des faces d'hommes.
\VS{8}Et elles avaient les cheveux comme des cheveux de femmes ; et leurs dents étaient comme des dents de lions.
\VS{9}Et elles avaient des cuirasses comme des cuirasses de fer ; et le bruit de leurs ailes était comme le bruit des chariots, quand plusieurs chevaux courent au combat.
\VS{10}Et elles avaient des queues semblables [à des queues] de scorpions, et avaient des aiguillons en leurs queues ; et leur puissance [était] de nuire aux hommes durant cinq mois.
\VS{11}Et elles avaient pour Roi au-dessus d'elles l'Ange de l'abîme, qui a nom en Hébreu, Abaddon, et dont le nom est en grec Apollyon.
\VS{12}Un malheur est passé, et voici venir encore deux malheurs après celui-ci.
\VS{13}Alors le sixième Ange sonna de sa trompette, et j'entendis une voix [sortant] des quatre cornes de l'autel d'or qui [est] devant la face de Dieu,
\VS{14}Laquelle dit au sixième Ange qui avait la trompette : Délie les quatre Anges qui sont liés sur le grand fleuve Euphrate.
\VS{15}On délia donc les quatre Anges qui étaient prêts pour l'heure, le jour, le mois et l'année ; afin de tuer la troisième partie des hommes.
\VS{16}Et le nombre de l'armée à cheval était de deux cents millions : car j'entendis [que c'était là] leur nombre.
\VS{17}Et je vis aussi dans la vision les chevaux, et ceux qui étaient montés dessus, ayant des cuirasses de feu, d'hyacinthe et de soufre ; et les têtes des chevaux [étaient] comme des têtes de lions ; et de leur bouche sortait du feu, de la fumée et du soufre.
\VS{18}La troisième partie des hommes fut tuée par ces trois choses, [savoir] par le feu, par la fumée, et par le soufre qui sortaient de leur bouche.
\VS{19}Car leur puissance était dans leur bouche et dans leurs queues ; et leurs queues [étaient] semblables à des serpents, et elles avaient des têtes par lesquelles elles nuisaient.
\VS{20}Mais le reste des hommes qui ne furent point tués par ces plaies, ne se repentit pas des œuvres de leurs mains, pour ne point adorer les Démons, les idoles d'or, d'argent, de cuivre, de pierre, et de bois, qui ne peuvent ni voir, ni ouïr, ni marcher.
\VS{21}Ils ne se repentirent point aussi de leurs meurtres, ni de leurs empoisonnements, ni de leur impudicité, ni de leurs larcins.
\Chap{10}
\VerseOne{}Alors je vis un autre Ange puissant, qui descendait du ciel, environné d'une nuée, sur la tête duquel était l'arc-en-ciel ; et son visage était comme le soleil, et ses pieds comme des colonnes de feu.
\VS{2}Et il avait en sa main un petit Livre ouvert, et il mit son pied droit sur la mer, et le gauche sur la terre ;
\VS{3}Et il cria à haute voix, comme lorsqu'un lion rugit ; et quand il eut crié, les sept tonnerres firent entendre leurs voix.
\VS{4}Et après que les sept tonnerres eurent fait entendre leurs voix, j'allais [les] écrire ; mais j'entendis une voix du ciel qui me disait : cachette les choses que les sept tonnerres ont fait entendre, et ne les écris point.
\VS{5}Et l'Ange que j'avais vu se tenant sur la mer et sur la terre, leva sa main vers le ciel,
\VS{6}Et jura par celui qui est vivant aux siècles des siècles, lequel a créé le ciel avec les choses qui y sont, et la terre avec les choses qui y sont, et la mer avec les choses qui y sont, qu'il n'y aurait plus de temps ;
\VS{7}Mais qu'aux jours de la voix du septième Ange, quand il commencera à sonner de la trompette, le mystère de Dieu sera consommé, comme il l'a déclaré à ses serviteurs les prophètes.
\VS{8}Et la voix du ciel que j'avais ouïe, me parla encore, et me dit : va, et prends le petit Livre ouvert, qui est en la main de l'Ange qui se tient sur la mer et sur la terre.
\VS{9}Je m'en allai donc vers l'Ange, et je lui dis : donne-moi le petit Livre ; et il me dit : prends-le, et le dévore ; et il remplira tes entrailles d'amertume, mais il sera doux dans ta bouche comme du miel.
\VS{10}Je pris donc le petit Livre de la main de l'Ange, et je le dévorai : et il était doux dans ma bouche comme du miel ; mais quand je l'eus dévoré, mes entrailles furent remplies d'amertume.
\VS{11}Alors il me dit : il faut que tu prophétises encore à plusieurs peuples, et [à plusieurs] nations, Langues et Rois.
\Chap{11}
\VerseOne{}Alors il me fut donné un roseau semblable à une verge, et il se présenta un Ange, qui me dit : lève-toi et mesure le temple de Dieu, et l'autel, et ceux qui y adorent.
\VS{2}Mais laisse à l'écart le parvis qui est hors du Temple, et ne le mesure point ; car il est donné aux Gentils ; et ils fouleront aux pieds la sainte Cité durant quarante-deux mois.
\VS{3}Mais je [la] donnerai à mes deux Témoins qui prophétiseront durant mille deux cent soixante jours, et ils seront vêtus de sacs.
\VS{4}Ceux-ci sont les deux oliviers, et les deux chandeliers, qui se tiennent en la présence du Seigneur de la terre.
\VS{5}Et si quelqu'un leur veut nuire, le feu sort de leur bouche, et dévore leurs ennemis ; car si quelqu'un leur veut nuire, il faut qu'il soit ainsi tué.
\VS{6}Ceux-ci ont le pouvoir de fermer le ciel, afin qu'il ne pleuve point durant les jours de leur prophétie ; ils ont aussi le pouvoir de changer les eaux en sang, et de frapper la terre de toutes sortes de plaies, toutes les fois qu'ils voudront.
\VS{7}Et quand ils auront achevé [de rendre] leur témoignage, la bête qui monte de l'abîme leur fera la guerre, les vaincra, et les tuera ;
\VS{8}Et leurs corps morts [seront étendus] dans les places de la grande Cité, qui est appelée spirituellement Sodome, et Egypte ; où aussi notre Seigneur a été crucifié.
\VS{9}Et ceux des Tribus, des peuples, des Langues, et des nations verront leurs corps morts durant trois jours et demi, et ils ne permettront point que leurs corps morts soient mis dans des sépulcres.
\VS{10}Et les habitants de la terre en seront tout joyeux, ils en feront des réjouissances, ils s'enverront des présents les uns aux autres ; parce que ces deux Prophètes auront tourmenté ceux qui habitent sur la terre.
\VS{11}Mais après ces trois jours et demi, l'Esprit de vie [venant] de Dieu entra en eux, et ils se tinrent sur leurs pieds, et une grande crainte saisit ceux qui les virent.
\VS{12}Après cela ils ouïrent une forte voix du ciel, leur disant : montez ici ; et ils montèrent au ciel sur une nuée, et leurs ennemis les virent.
\VS{13}Et à cette même heure-là il se fit un grand tremblement de terre ; et la dixième partie de la Cité tomba, et sept mille hommes furent tués par ce tremblement de terre ; et les autres furent épouvantés, et donnèrent gloire au Dieu du ciel.
\VS{14}Le second malheur est passé ; et voici, le troisième malheur viendra bientôt.
\VS{15}Le septième Ange donc sonna de la trompette, et il se fit entendre au ciel de grandes voix, qui disaient : Les Royaumes du monde sont soumis à notre Seigneur, et à son Christ, et il régnera aux siècles des siècles.
\VS{16}Alors les vingt-quatre Anciens qui sont assis devant Dieu dans leurs sièges, se prosternèrent sur leurs faces, et adorèrent Dieu,
\VS{17}En disant : Nous te rendons grâces, Seigneur Dieu tout-puissant, QUI ES, QUI ÉTAIS, et QUI ES A VENIR, de ce que tu as fait éclater ta grande puissance, et de ce que tu as agi en Roi.
\VS{18}Les nations se sont irritées, mais ta colère est venue, et le temps des morts est venu pour être jugés, et pour donner la récompense à tes serviteurs les Prophètes, et aux Saints, et à ceux qui craignent ton Nom, petits et grands, et pour détruire ceux qui corrompent la terre.
\VS{19}Alors le Temple de Dieu fut ouvert au ciel, et l'Arche de son alliance fut vue dans son Temple ; et il y eut des éclairs, des voix, des tonnerres, un tremblement de terre, et une grosse grêle.
\Chap{12}
\VerseOne{}Et un grand signe parut au Ciel, [savoir], une femme revêtue du soleil, sous les pieds de laquelle était la lune, et sur sa tête une couronne de douze étoiles.
\VS{2}Elle était enceinte, et elle criait étant en travail d'enfant, souffrant les grandes douleurs de l'enfantement.
\VS{3}Il parut aussi un autre signe au ciel, et voici un grand dragon roux ayant sept têtes et dix cornes, et sur ses têtes sept diadèmes ;
\VS{4}et sa queue traînait la troisième partie des étoiles du ciel, lesquelles il jeta en la terre ; puis le dragon s'arrêta devant la femme qui devait accoucher, afin de dévorer son enfant, dès qu'elle l'aurait mis au monde.
\VS{5}Et elle accoucha d'un fils, qui doit gouverner toutes les nations avec une verge de fer ; et son enfant fut enlevé vers Dieu, et vers son trône.
\VS{6}Et la femme s'enfuit dans un désert, où elle a un lieu préparé de Dieu, afin qu'on la nourrisse là mille deux cent soixante jours.
\VS{7}Et il y eut une bataille au ciel : Michel et ses Anges combattaient contre le dragon ; et le dragon et ses Anges combattaient [contre Michel].
\VS{8}Mais ils ne furent pas les plus forts, et ils ne purent plus se maintenir dans le ciel.
\VS{9}Et le grand dragon, le serpent ancien, appelé le Diable et Satan, qui séduit le monde, fut précipité en la terre, et ses Anges furent précipités avec lui.
\VS{10}Alors j'ouïs une grande voix dans le ciel, qui disait : Maintenant est le salut, la force, le règne de notre Dieu, et la puissance de son Christ ; car l'accusateur de nos frères, qui les accusait devant notre Dieu jour et nuit, a été précipité.
\VS{11}Et ils l'ont vaincu à cause du sang de l'Agneau, et à cause de la parole de leur témoignage, et ils n'ont point aimé leurs vies, [mais les ont exposées] à la mort.
\VS{12}C'est pourquoi réjouissez-vous, cieux, et vous qui y habitez. Mais malheur [à vous] habitants de la terre et de la mer ; car le Diable est descendu vers vous en grande fureur, sachant qu'il a peu de temps.
\VS{13}Or, quand le dragon eut vu qu'il avait été jeté en la terre, il persécuta la femme qui avait accouché d'un fils.
\VS{14}Mais deux ailes d'une grande aigle furent données à la femme, afin qu'elle s'envolât de devant le serpent en son lieu, où elle est nourrie par un temps, par des temps, et par la moitié d'un temps.
\VS{15}Et le serpent jeta de sa gueule de l'eau comme un fleuve après la femme, afin de la faire emporter par le fleuve.
\VS{16}Mais la terre aida à la femme ; car la terre ouvrit son sein, et elle engloutit le fleuve que le dragon avait jeté de sa gueule.
\VS{17}Alors le dragon fut irrité contre la femme, et s'en alla faire la guerre contre les autres qui sont de la semence de la femme, qui gardent les commandements de Dieu, et qui ont le témoignage de Jésus-Christ.
\VS{18}Et je me tins sur le sable [qui borde] la mer.
\Chap{13}
\VerseOne{}Et je vis monter de la mer une bête qui avait sept têtes et dix cornes, et sur ses cornes dix diadèmes, et sur ses têtes un nom de blasphème.
\VS{2}Et la bête que je vis était semblable à un léopard, ses pieds étaient comme les pieds d'un ours ; sa gueule était comme la gueule d'un lion ; et le dragon lui donna sa puissance, son trône, et une grande autorité.
\VS{3}Et je vis l'une de ses têtes comme blessée à mort, mais sa plaie mortelle fut guérie ; et toute la terre en étant dans l'admiration alla après la bête.
\VS{4}Et ils adorèrent le dragon qui avait donné le pouvoir à la bête, et ils adorèrent aussi la bête, en disant : qui est semblable à la bête, et qui pourra combattre contre elle ?
\VS{5}Et il lui fut donné une bouche qui proférait de grandes choses, et des blasphèmes ; et il lui fut aussi donné le pouvoir d'accomplir quarante-deux mois.
\VS{6}Et elle ouvrit sa bouche en blasphèmes contre Dieu, blasphémant son Nom, et son tabernacle, et ceux qui habitent au ciel.
\VS{7}Et il lui fut donné de faire la guerre aux Saints, et de les vaincre. Il lui fut aussi donné puissance sur toute Tribu, Langue et nation.
\VS{8}De sorte qu'elle sera adorée par tous ceux qui habitent sur la terre, desquels les noms ne sont point écrits au Livre de vie de l'Agneau, immolé dès la fondation du monde.
\VS{9}Si quelqu'un a des oreilles, qu'il écoute.
\VS{10}Si quelqu'un mène en captivité, il sera mené en captivité ; si quelqu'un tue avec l'épée, il faut qu'il soit lui-même tué avec l'épée. Ici est la patience et la foi des Saints.
\VS{11}Puis je vis une autre bête qui montait de la terre, et qui avait deux cornes semblables à celles de l'Agneau ; mais elle parlait comme le dragon.
\VS{12}Et elle exerçait toute la puissance de la première bête, en sa présence, et faisait que la terre et ses habitants adorassent la première bête, dont la plaie mortelle avait été guérie.
\VS{13}Et elle faisait de grands prodiges, même jusqu'à faire descendre le feu du ciel sur la terre devant les hommes.
\VS{14}Et elle séduisait les habitants de la terre, à cause des prodiges qu'il lui était donné de faire devant la bête, commandant aux habitants de la terre de faire une image à la bête qui avait reçu le coup [mortel] de l'épée, et qui néanmoins était vivante.
\VS{15}Et il lui fut permis de donner une âme à l'image de la bête, afin que même l'image de la bête parlât, et qu'elle fît que tous ceux qui n'auraient point adoré l'image de la bête, fussent mis à mort.
\VS{16}Et elle faisait que tous, petits et grands, riches et pauvres, libres et esclaves, prenaient une marque en leur main droite, ou en leurs fronts ;
\VS{17}Et qu'aucun ne pouvait acheter, ni vendre, s'il n'avait la marque ou le nom de la bête, ou le nombre de son nom.
\VS{18}Ici est la sagesse : que celui qui a de l'intelligence, compte le nombre de la bête ; car c'est un nombre d'homme, et son nombre [est] six cent soixante-six.
\Chap{14}
\VerseOne{}Puis je regardai, et voici, l'Agneau se tenait sur la montagne de Sion, et il y avait avec lui cent quarante-quatre mille [personnes], qui avaient le Nom de son Père écrit sur leurs fronts.
\VS{2}Et j'entendis une voix du ciel comme le bruit des grandes eaux, et comme le bruit d'un grand tonnerre ; et j'entendis une voix de joueurs de harpe, qui jouaient de leurs harpes,
\VS{3}Et qui chantaient comme un cantique nouveau devant le trône, et devant les quatre animaux, et devant les Anciens ; et personne ne pouvait apprendre le cantique, que les cent quarante-quatre mille qui ont été achetés d'entre ceux de la terre.
\VS{4}Ce sont ceux qui ne se sont point souillés avec les femmes, car ils sont vierges ; ce sont ceux qui suivent l'Agneau quelque part qu'il aille ; et ce sont ceux qui ont été achetés d'entre les hommes pour être des prémices à Dieu, et à l'Agneau.
\VS{5}Et il n'a été trouvé aucune fraude en leur bouche ; car ils sont sans tache devant le trône de Dieu.
\VS{6}Puis je vis un autre Ange qui volait par le milieu du ciel, ayant l'Evangile éternel, afin d'évangéliser à ceux qui habitent sur la terre, et à toute nation, Tribu, Langue et peuple ;
\VS{7}Disant à haute voix : Craignez Dieu, et lui donnez gloire ; car l'heure de son jugement est venue ; et adorez celui qui a fait le ciel et la terre, la mer et les fontaines des eaux.
\VS{8}Et un autre Ange le suivit, disant : Elle est tombée, elle est tombée Babylone, cette grande Cité, parce qu'elle a abreuvé toutes les nations du vin de la fureur de son impudicité.
\VS{9}Et un troisième Ange suivit ceux-là, disant à haute voix : Si quelqu'un adore la bête et son image, et qu'il en prenne la marque sur son front, ou en sa main,
\VS{10}Celui-là aussi boira du vin de la colère de Dieu, du vin pur versé dans la coupe de sa colère, et il sera tourmenté de feu et de soufre devant les saints Anges, et devant l'Agneau.
\VS{11}Et la fumée de leur tourment montera aux siècles des siècles, et ceux-là n'auront nul repos ni jour ni nuit qui adorent la bête et son image, et quiconque prend la marque de son nom.
\VS{12}Ici est la patience des Saints ; ici [sont] ceux qui gardent les commandements de Dieu, et la foi de Jésus.
\VS{13}Alors j'entendis une voix du ciel me disant : écris : Bienheureux sont les morts qui dorénavant meurent au Seigneur ; oui pour certain, dit l'Esprit ; car ils se reposent de leurs travaux, et leurs œuvres les suivent.
\VS{14}Et je regardai, et voici une nuée blanche, et sur la nuée quelqu'un assis, semblable à un homme, ayant sur sa tête une couronne d'or, et en sa main une faucille tranchante.
\VS{15}Et un autre Ange sortit du Temple, criant à haute voix à celui qui était assis sur la nuée : jette ta faucille, et moissonne ; car c'est ton heure de moissonner, parce que la moisson de la terre est mûre.
\VS{16}Alors celui qui était assis sur la nuée, jeta sa faucille sur la terre, et la terre fut moissonnée.
\VS{17}Et un autre Ange sortit du Temple qui est au Ciel, ayant aussi une faucille tranchante.
\VS{18}Et un autre Ange sortit de l'autel, ayant puissance sur le feu ; et il cria, jetant un grand cri à celui qui avait la faucille tranchante, disant : jette ta faucille tranchante, et vendange les grappes de la vigne de la terre, car ses raisins sont mûrs.
\VS{19}Et l'Ange jeta en la terre sa faucille tranchante : et vendangea la vigne de la terre, et il jeta [la vendange] en la grande cuve de la colère de Dieu.
\VS{20}Et la cuve fut foulée hors de la Cité ; et de la cuve il sortit du sang [qui allait] jusqu'aux freins des chevaux dans [l'étendue] de mille six cents stades.
\Chap{15}
\VerseOne{}Puis je vis au ciel un autre signe, grand et admirable, [savoir] sept Anges qui avaient les sept dernières plaies ; car c'est par elles que la colère de Dieu est consommée.
\VS{2}Je vis aussi comme une mer de verre mêlée de feu, et ceux qui avaient obtenu la victoire sur la bête, sur son image, sur sa marque, et sur le nombre de son nom, se tenant sur la mer qui était comme de verre, et ayant les harpes de Dieu,
\VS{3}Qui chantaient le Cantique de Moïse serviteur de Dieu, et le Cantique de l'Agneau, en disant : que tes œuvres sont grandes et merveilleuses, ô Seigneur Dieu tout-puissant ! tes voies [sont] justes et véritables, ô Roi des Saints !
\VS{4}Seigneur, qui ne te craindra, et qui ne glorifiera ton Nom ? car tu es Saint toi seul, c'est pourquoi toutes les nations viendront et se prosterneront devant toi ; car tes jugements sont pleinement manifestés.
\VS{5}Et après ces choses je regardai, et voici le Temple du Tabernacle du témoignage fut ouvert au ciel.
\VS{6}Et les sept Anges qui avaient les sept plaies sortirent du Temple, vêtus d'un lin pur et blanc, et ceints sur leurs poitrines avec des ceintures d'or.
\VS{7}Et l'un des quatre animaux donna aux sept Anges sept fioles d'or, pleines de la colère du Dieu vivant aux siècles des siècles.
\VS{8}Et le Temple fut rempli de la fumée qui [procédait] de la majesté de Dieu et de sa puissance ; et personne ne pouvait entrer dans le Temple jusqu'à ce que les sept plaies des sept Anges fussent accomplies.
\Chap{16}
\VerseOne{}Alors j'ouïs du Temple une voix éclatante, qui disait aux sept Anges : allez, et versez sur la terre les fioles de la colère de Dieu.
\VS{2}Ainsi le premier [Ange] s'en alla, et versa sa fiole sur la terre ; et un ulcère malin et dangereux attaqua les hommes qui avaient la marque de la bête, et ceux qui adoraient son image.
\VS{3}Et le second Ange versa sa fiole sur la mer, et elle devint comme le sang d'un corps mort, et toute âme qui vivait dans la mer, mourut.
\VS{4}Et le troisième Ange versa sa fiole sur les fleuves, et sur les fontaines des eaux, et elles devinrent du sang.
\VS{5}Et j'entendis l'Ange des eaux, qui disait : Seigneur, QUI ES, QUI ÉTAIS, et QUI SERAS, tu es juste, parce que tu as fait un tel jugement.
\VS{6}A cause qu'ils ont répandu le sang des Saints et des Prophètes, tu leur as aussi donné du sang à boire ; car ils [en] sont dignes.
\VS{7}Et j'en ouïs un autre du Sanctuaire, disant : certainement Seigneur Dieu tout-puissant, tes jugements [sont] véritables, et justes.
\VS{8}Puis le quatrième Ange versa sa fiole sur le soleil, et [le pouvoir] lui fut donné de brûler les hommes par le feu.
\VS{9}De sorte que les hommes furent brûlés par de grandes chaleurs, et ils blasphémèrent le Nom de Dieu qui a puissance sur ces plaies ; mais ils ne se repentirent point pour lui donner gloire.
\VS{10}Après cela le cinquième Ange versa sa fiole sur le siège de la bête, et le règne de la bête devint ténébreux, et [les hommes] se mordaient la langue à cause de la douleur qu'ils ressentaient.
\VS{11}Et à cause de leurs peines et de leurs plaies ils blasphémèrent le Dieu du Ciel ; et ne se repentirent point de leurs œuvres.
\VS{12}Puis le sixième Ange versa sa fiole sur le grand fleuve d'Euphrate, et l'eau de ce [fleuve] tarit, afin que la voie des Rois de devers le soleil levant fût ouverte.
\VS{13}Et je vis sortir de la gueule du dragon, et de la gueule de la bête, et de la bouche du faux prophète, trois esprits immondes, semblables à des grenouilles ;
\VS{14}Car ce sont des esprits diaboliques, faisant des prodiges, et qui s'en vont vers les Rois de la terre et du monde universel, pour les assembler pour le combat de ce grand jour du Dieu tout-puissant.
\VS{15}Voici, je viens comme le larron ; bienheureux est celui qui veille, et qui garde ses vêtements, afin de ne marcher point nu, et qu'on ne voie point sa honte.
\VS{16}Et il les assembla au lieu qui est appelé en Hébreu Armageddon.
\VS{17}Puis le septième Ange versa sa fiole dans l'air ; et il sortit du Temple du Ciel une voix tonnante qui procédait du trône, disant : c'est fait.
\VS{18}Alors il se fit des éclairs, et des voix, et des tonnerres, et il se fit un grand tremblement de terre, un tel tremblement, [dis-je], et si grand, qu'il n'y en eut jamais de semblable depuis que les hommes ont été sur la terre.
\VS{19}Et la grande Cité fut divisée en trois parties, et les villes des nations tombèrent ; et la grande Babylone vint en mémoire devant Dieu, pour lui donner la coupe du vin de l'indignation de sa colère.
\VS{20}Et toute île s'enfuit, et les montagnes ne furent plus trouvées.
\VS{21}Et il descendit du ciel sur les hommes une grêle prodigieuse du poids d'un talent ; et les hommes blasphémèrent Dieu à cause de la plaie de la grêle ; car la plaie qu'elle fit fut fort grande.
\Chap{17}
\VerseOne{}Alors l'un des sept Anges qui avaient les sept fioles, vint, et il me parla, et me dit : Viens, je te montrerai la condamnation de la grande prostituée, qui est assise sur plusieurs eaux ;
\VS{2}Avec laquelle les Rois de la terre ont commis fornication, et qui a enivré du vin de sa prostitution les habitants de la terre.
\VS{3}Ainsi il me transporta en esprit dans un désert ; et je vis une femme montée sur une bête de couleur d'écarlate, pleine de noms de blasphème, et qui avait sept têtes et dix cornes.
\VS{4}Et la femme était vêtue de pourpre et d'écarlate, et parée d'or, de pierres précieuses, et de perles ; et elle tenait à la main une coupe d'or, pleine des abominations de l'impureté de sa prostitution.
\VS{5}Et il y avait sur son front un nom écrit, mystère, la grande Babylone, la mère des impudicités et des abominations de la terre.
\VS{6}Et je vis la femme enivrée du sang des Saints, et du sang des martyrs de Jésus ; et quand je la vis je fus saisi d'un grand étonnement.
\VS{7}Et l'Ange me dit : pourquoi t'étonnes-tu ? je te dirai le mystère de la femme et de la bête qui la porte, laquelle a sept têtes et dix cornes.
\VS{8}La bête que tu as vue, a été, et n'est plus, mais elle doit monter de l'abîme, et puis être détruite ; et les habitants de la terre, dont les noms ne sont point écrits au Livre de vie dès la fondation du monde, s'étonneront voyant la bête qui était, qui n'est plus, et qui toutefois est.
\VS{9}C'est ici qu'est l'intelligence pour quiconque a de la sagesse. Les sept têtes sont sept montagnes sur lesquelles la femme est assise.
\VS{10}Ce sont aussi sept Rois, les cinq sont tombés ; l'un est, et l'autre n'est pas encore venu ; et quand il sera venu, il faut qu'il demeure pour un peu de temps.
\VS{11}Et la bête qui était, et qui n'est plus, c'est aussi un huitième [Roi], elle vient des sept, mais elle tend à sa ruine.
\VS{12}Et les dix cornes que tu as vues, sont dix Rois, qui n'ont pas encore commencé à régner, mais ils prendront puissance comme Rois, en même temps avec la bête.
\VS{13}Ceux-ci ont un même dessein, et ils donneront leur puissance et leur autorité à la bête.
\VS{14}Ceux-ci combattront contre l'Agneau ; mais l'Agneau les vaincra ; parce qu'il est le Seigneur des Seigneurs, et le Roi des Rois ; et ceux qui sont avec lui, [sont du nombre] des appelés, des élus et des fidèles.
\VS{15}Puis il me dit : Les eaux que tu as vues, et sur lesquelles la prostituée est assise, sont des peuples, des nations et des Langues.
\VS{16}Mais les dix cornes que tu as vues à la bête, sont ceux qui haïront la prostituée, qui la désoleront, la dépouilleront, et mangeront sa chair, et la brûleront au feu.
\VS{17}Car Dieu a mis dans leurs cœurs de faire ce qu'il lui plaît, et de former un même dessein, et de donner leur Royaume à la bête, jusqu'à ce que les paroles de Dieu soient accomplies.
\VS{18}Et la femme que tu as vue, c'est la grande Cité, qui a son règne sur les Rois de la terre.
\Chap{18}
\VerseOne{}Après ces choses je vis descendre du ciel un autre Ange, qui avait une grande puissance, et la terre fut illuminée de sa gloire.
\VS{2}Il cria avec force à haute voix, et il dit : Elle est tombée, elle est tombée la grande Babylone, et elle est devenue la demeure des Démons, et la retraite de tout esprit immonde, et le repaire de tout oiseau immonde et exécrable.
\VS{3}Car toutes les nations ont bu du vin de sa prostitution effrénée ; et les Rois de la terre ont commis fornication avec elle ; et les marchands de la terre sont devenus riches de l'excès de son luxe.
\VS{4}Puis j'entendis une autre voix du ciel, qui disait : Sortez de Babylone mon peuple, afin que vous ne participiez point à ses péchés, et que vous ne receviez point de ses plaies.
\VS{5}Car ses péchés sont montés jusqu'au ciel, et Dieu s'est souvenu de ses iniquités.
\VS{6}Rendez-lui ainsi qu'elle vous a fait, et payez-lui au double selon ses œuvres ; et dans la même coupe où elle vous a versé [à boire], versez-lui-en au double.
\VS{7}Autant qu'elle s'est glorifiée, et qu'elle a été dans les délices, donnez-lui autant de tourment et d'affliction ; car elle dit en son cœur : je siège en Reine, je ne suis point veuve, et je ne verrai point de deuil.
\VS{8}C'est pourquoi ses plaies, qui sont la mort, le deuil, et la famine, viendront en un même jour, et elle sera entièrement brûlée au feu ; car le Seigneur Dieu qui la jugera, est puissant.
\VS{9}Et les Rois de la terre, qui ont commis fornication avec elle, et qui ont vécu dans les délices, la pleureront, et mèneront deuil sur elle en se battant la poitrine, quand ils verront la fumée de son embrasement ;
\VS{10}Et ils se tiendront loin pour la crainte de son tourment, et diront : hélas ! hélas ! Babylone, la grande Cité, cette Cité si puissante, comment ta condamnation est-elle venue en un moment ?
\VS{11}Les marchands de la terre aussi pleureront, et mèneront deuil à cause d'elle, parce que personne n'achète plus de leur marchandise ;
\VS{12}Qui sont des marchandises d'or, d'argent, de pierres précieuses, de perles, de fin lin, de pourpre, de soie, d'écarlate, de toute sorte de bois odoriférant, de toute espèce de meubles d'ivoire, et de toute espèce de vaisseaux de bois très précieux, d'airain, de fer, et de marbre ;
\VS{13}Du cinnamome, des parfums, des essences, de l'encens, du vin, de l'huile, de la fine fleur de farine, du blé, des bêtes de charge, des brebis, des chevaux, des chariots, des esclaves, et des âmes d'hommes.
\VS{14}Car les fruits du désir de ton âme se sont éloignés de toi ; et toutes les choses délicates et excellentes sont péries pour toi ; et dorénavant tu ne trouveras plus ces choses.
\VS{15}Les marchands, [dis-je], de ces choses, qui en sont devenus riches, se tiendront loin d'elle, pour la crainte de son tourment, pleurant et menant deuil ;
\VS{16}Et disant : hélas ! hélas ! la grande Cité, qui était vêtue de fin lin, de pourpre, d'écarlate, qui était parée d'or, ornée de pierres précieuses, et de perles, comment en un instant ont été dissipées tant de richesses ?
\VS{17}Tout pilote aussi, toute la troupe de ceux qui montent sur les navires, tous les matelots, et tous ceux qui trafiquent sur la mer, se tiendront loin ;
\VS{18}Et voyant la fumée de son embrasement, ils s'écrieront en disant : quelle [cité était] semblable à cette grande Cité !
\VS{19}Ils jetteront de la poussière sur leurs têtes, pleurant, et menant deuil, ils crieront en disant : hélas ! hélas ! la grande Cité, dans laquelle tous ceux qui avaient des navires sur la mer, étaient devenus riches par son opulence ; comment a-t-elle été désolée en un moment ?
\VS{20}Ô ciel ! réjouis-toi à cause d'elle ; et vous aussi, saints Apôtres et Prophètes [réjouissez-vous] : car Dieu l'a punie à cause de vous.
\VS{21}Puis un Ange d'une grande force prit une pierre, [qui était] comme une grande meule, et la jeta dans la mer, en disant : Ainsi sera jetée avec impétuosité Babylone, cette grande Cité ; et elle ne sera plus trouvée.
\VS{22}Et la voix des joueurs de harpe, des musiciens, des joueurs de hautbois, et de ceux qui sonnent de la trompette, ne sera plus ouïe en toi ; et tout ouvrier de quelque métier que ce soit, ne sera plus trouvé en toi ; et le bruit de la meule ne sera plus ouï en toi.
\VS{23}Et la lumière de la chandelle ne luira plus en toi ; et la voix de l'époux et de l'épouse ne sera plus ouïe en toi ; parce que tes marchands étaient des Princes en la terre ; et parce que par tes empoisonnements toutes les nations ont été séduites.
\VS{24}Et en elle a été trouvé le sang des Prophètes, et des Saints, et de tous ceux qui ont été mis à mort sur la terre.
\Chap{19}
\VerseOne{}Or après ces choses, j'entendis une voix d'une grande multitude au Ciel, disant : Alleluia ! le salut, la gloire, l'honneur et la puissance [appartiennent] au Seigneur notre Dieu.
\VS{2}Car ses jugements sont véritables et justes, parce qu'il a fait justice de la grande prostituée, qui a corrompu la terre par son impudicité ; et qu'il a vengé le sang de ses serviteurs [versé] de la main de la prostituée.
\VS{3}Et ils dirent encore : Alleluia ! et sa fumée monte aux siècles des siècles.
\VS{4}Et les vingt-quatre Anciens et les quatre animaux se jetèrent sur leurs faces, et adorèrent Dieu, qui était assis sur le trône, en disant : Amen ! Alleluia !
\VS{5}Et il sortit du trône une voix qui disait : louez notre Dieu, vous tous ses serviteurs, et vous qui le craignez, tant les petits que les grands.
\VS{6}J'entendis ensuite comme la voix d'une grande assemblée, et comme le bruit de grandes eaux, et comme l'éclat de grands tonnerres, disant : Alleluia ! car le Seigneur notre Dieu tout-puissant a pris possession de son Royaume.
\VS{7}Réjouissons-nous, tressaillons de joie, et donnons-lui gloire ; car les noces de l'Agneau sont venues, et son épouse s'est parée.
\VS{8}Et il lui a été donné d'être vêtue de fin lin pur et éclatant. Or ce fin lin désigne la justice des Saints.
\VS{9}Alors il me dit : Ecris : Bienheureux [sont] ceux qui sont appelés au banquet des noces de l'Agneau. Il me dit aussi : ces paroles de Dieu sont véritables.
\VS{10}Alors je me jetai à ses pieds pour l'adorer ; mais il me dit : garde-toi de le faire ; je suis ton compagnon de service, et [le compagnon] de tes frères qui ont le témoignage de Jésus, adore Dieu ; car le témoignage de Jésus est l'Esprit de prophétie.
\VS{11}Puis je vis le Ciel ouvert, et voici un cheval blanc ; et celui qui était monté dessus était appelé FIDÈLE et VÉRITABLE, qui juge et combat justement.
\VS{12}Et ses yeux étaient comme une flamme de feu ; il y avait sur sa tête plusieurs diadèmes, et il portait un nom écrit que nul n'a connu, que lui seul.
\VS{13}Il était vêtu d'une robe teinte dans le sang, et son nom s'appelle LA PAROLE DE DIEU.
\VS{14}Et les armées qui sont au Ciel le suivaient sur des chevaux blancs, vêtues de fin lin blanc, et pur.
\VS{15}Et il sortait de sa bouche une épée tranchante, pour en frapper les nations ; car il les gouvernera avec une verge de fer, et il foulera la cuve du vin de l'indignation et de la colère du Dieu tout-puissant.
\VS{16}Et sur son vêtement et sur sa cuisse étaient écrits ces mots : LE ROI DES ROIS, ET LE SEIGNEUR DES SEIGNEURS.
\VS{17}Puis je vis un Ange se tenant dans le soleil, qui cria à haute voix, et dit à tous les oiseaux qui volaient par le milieu du ciel : venez, et assemblez vous au banquet du grand Dieu ;
\VS{18}Afin que vous mangiez la chair des Rois, la chair des capitaines, la chair des puissants, la chair des chevaux, et de ceux qui sont montés dessus, et la chair de toute sorte de personnes libres, esclaves, petits et grands.
\VS{19}Alors je vis la bête, et les Rois de la terre, et leurs armées assemblées pour faire la guerre contre celui qui était monté sur le cheval, et contre son armée.
\VS{20}Mais la bête fut prise, et avec elle le faux-prophète qui avait fait devant elle les prodiges par lesquels il avait séduit ceux qui avaient la marque de la bête, et qui avaient adoré son image ; et ils furent tous deux jetés tout vifs dans l'étang ardent de feu et de soufre ;
\VS{21}Et le reste fut tué par l'épée qui sortait de la bouche de celui qui était monté sur le cheval, et tous les oiseaux furent rassasiés de leur chair.
\Chap{20}
\VerseOne{}Après cela je vis descendre du Ciel un Ange, qui avait la clef de l'abîme, et une grande chaîne en sa main ;
\VS{2}Lequel saisit le dragon, [c'est-à-dire], le serpent ancien, qui est le Diable et satan, et le lia pour mille ans ;
\VS{3}Et il le jeta dans l'abîme, et l'enferma, et mit le sceau sur lui, afin qu'il ne séduise plus les nations, jusqu'à ce que les mille ans soient accomplis ; après quoi il faut qu'il soit délié pour un peu de temps.
\VS{4}Et je vis des trônes, sur lesquels [des gens] s'assirent, et [l'autorité] de juger leur fut donnée, [et je vis] les âmes de [ceux qui avaient été] décapités pour le témoignage de Jésus, et pour la Parole de Dieu, qui n'avaient point adoré la bête ni son image, et qui n'avaient point pris sa marque en leurs fronts, ou en leurs mains, lesquels devaient vivre et régner avec Christ mille ans.
\VS{5}Mais le reste des morts ne doit point ressusciter jusqu'à ce que les mille ans soient accomplis ; c'est la première résurrection.
\VS{6}Bienheureux et saint [est] celui qui a part à la première résurrection ; la mort seconde n'a point de puissance sur eux, mais ils seront Sacrificateurs de Dieu, et de Christ, et ils régneront avec lui mille ans.
\VS{7}Et quand les mille ans seront accomplis, satan sera délié de sa prison ;
\VS{8}Et il sortira pour séduire les nations qui [sont] aux quatre coins de la terre, Gog et Magog ; pour les assembler en bataille, et leur nombre est comme le sable de la mer.
\VS{9}Et ils montèrent [et se répandirent] sur la largeur de la terre, et ils environnèrent le camp des Saints, et la Cité bien-aimée, mais Dieu fit descendre du feu du Ciel, qui les dévora.
\VS{10}Et le Diable qui les séduisait, fut jeté dans l'étang de feu et de soufre, où est la bête et le faux-prophète ; et ils seront tourmentés jour et nuit, aux siècles des siècles.
\VS{11}Puis je vis un grand trône blanc, et quelqu'un assis dessus, de devant lequel s’enfuit la terre et le ciel ; et il ne se trouva point de lieu pour eux.
\VS{12}Je vis aussi les morts grands et petits se tenant devant Dieu, et les Livres furent ouverts ; et un autre Livre fut ouvert, [qui était le Livre] de vie ; et les morts furent jugés sur les choses qui étaient écrites dans les Livres, [c'est-à-dire], selon leurs œuvres.
\VS{13}Et la mer rendit les morts qui étaient en elle, et la mort et l'enfer rendirent les morts qui étaient en eux ; et ils furent jugés chacun selon ses œuvres.
\VS{14}Et la mort et l'enfer furent jetés dans l'étang de feu : c'est la mort seconde.
\VS{15}Et quiconque ne fut pas trouvé écrit au Livre de vie, fut jeté dans l'étang de feu.
\Chap{21}
\VerseOne{}Puis je vis un nouveau Ciel et une nouvelle terre ; car le premier ciel et la première terre avaient disparu, et la mer n'était plus.
\VS{2}Et moi, Jean, je vis la sainte Cité, la nouvelle Jérusalem, qui descendait du Ciel, de devers Dieu, parée comme une épouse qui s'est ornée pour son mari.
\VS{3}Et j'entendis une grande voix du ciel, disant : voici le Tabernacle de Dieu avec les hommes, et il habitera avec eux ; et ils seront son peuple, et Dieu lui-même sera leur Dieu, [et il sera] avec eux.
\VS{4}Et Dieu essuiera toutes larmes de leurs yeux, et la mort ne sera plus ; et il n'y aura plus ni deuil, ni cri, ni travail ; car les premières choses sont passées.
\VS{5}Et celui qui était assis sur le trône, dit : voici, je fais toutes choses nouvelles. Puis il me dit : Ecris, car ces paroles sont véritables et certaines.
\VS{6}Il me dit aussi : tout est accompli ; je suis l'Alpha et l'Oméga, le commencement, et la fin. A celui qui aura soif je lui donnerai de la fontaine d'eau vive, sans qu'elle lui coûte rien.
\VS{7}Celui qui vaincra, héritera toutes choses ; et je lui serai Dieu, et il me sera fils.
\VS{8}Mais quant aux timides, aux incrédules, aux exécrables, aux meurtriers, aux fornicateurs, aux empoisonneurs, aux idolâtres et à tous menteurs, leur part sera dans l'étang ardent de feu et de soufre, qui est la mort seconde.
\VS{9}Alors un des sept Anges qui avaient eu les sept fioles pleines des sept dernières plaies, s'approcha de moi et me parla en disant : Viens et je te montrerai l'Epouse, qui est la femme de l'Agneau.
\VS{10}Et il me transporta en esprit sur une grande et haute montagne, et il me montra la grande Cité, la sainte Jérusalem, qui descendait du Ciel de devers Dieu,
\VS{11}Ayant la gloire de Dieu ; et sa lumière était semblable à une pierre très précieuse, comme à une pierre de jaspe tirant sur le cristal.
\VS{12}Et elle avait une grande et haute muraille, avec douze portes, et aux portes douze Anges ; et des noms écrits sur elles, qui sont les noms des douze Tribus des enfants d'Israël.
\VS{13}Du côté de l'Orient, trois portes ; du côté de l'Aquilon, trois portes ; du côté du Midi, trois portes ; et du côté de l'Occident, trois portes.
\VS{14}Et la muraille de la Cité avait douze fondements, et les noms des douze Apôtres de l'Agneau étaient écrits dessus.
\VS{15}Et celui qui parlait avec moi avait un roseau d'or pour mesurer la Cité, ses portes et sa muraille.
\VS{16}Et la cité était bâtie en carré, et sa longueur était aussi grande que sa largeur. Il mesura donc la Cité avec le roseau [d'or], jusqu'à douze mille stades ; la longueur, la largeur et la hauteur étaient égales.
\VS{17}Puis il mesura la muraille, [qui fut] de cent quarante-quatre coudées, de la mesure du personnage, c'est-à-dire, de l'Ange.
\VS{18}Et le bâtiment de la muraille était de jaspe, mais la Cité était d'or pur, semblable à du verre fort transparent.
\VS{19}Et les fondements de la muraille de la Cité étaient ornés de toute pierre précieuse : le premier fondement était de jaspe ; le second, de saphir ; le troisième, de calcédoine ; le quatrième, d'émeraude ;
\VS{20}Le cinquième, de sardonyx ; le sixième, de sardoine ; le septième, de chrysolithe ; le huitième, de béryl ; le neuvième, de topaze ; le dixième, de chrysoprase ; le onzième, d'hyacinthe ; le douzième, d'améthyste.
\VS{21}Et les douze portes [étaient] douze perles ; chacune des portes était d'une perle ; et la rue de la cité était d'or pur, comme du verre le plus transparent.
\VS{22}Et je ne vis point de Temple en elle ; parce que le Seigneur Dieu Tout-puissant et l'Agneau en sont le Temple.
\VS{23}Et la Cité n'a pas besoin du soleil ni de la lune, pour luire en elle ; car la clarté de Dieu l'a éclairée, et l'Agneau est son flambeau.
\VS{24}Et les nations qui auront été sauvées, marcheront à la faveur de sa lumière ; et les Rois de la terre y apporteront ce qu'ils ont de plus magnifique et de plus précieux.
\VS{25}Et ses portes ne seront point fermées de jour ; or il n'y aura point là de nuit.
\VS{26}Et on y apportera ce que les Gentils ont de plus magnifique et de plus précieux.
\VS{27}Il n'y entrera aucune chose souillée, ni personne qui s'abandonne à l'abomination et au mensonge ; mais seulement ceux qui sont écrits au Livre de vie de l'Agneau.
\Chap{22}
\VerseOne{}Puis il me montra un fleuve pur d'eau vive, transparent comme du cristal, qui sortait du trône de Dieu et de l'Agneau.
\VS{2}Et au milieu de la place de la Cité, et des deux côtés du fleuve était l'arbre de vie, portant douze fruits, et rendant son fruit chaque mois ; et les feuilles de l'arbre [sont] pour la santé des Gentils.
\VS{3}Et toute chose maudite ne sera plus, mais le trône de Dieu et de l'Agneau sera en elle, et ses serviteurs le serviront ;
\VS{4}Et ils verront sa face, et son Nom sera sur leurs fronts.
\VS{5}Et il n'y aura plus là de nuit ; et il ne sera plus besoin de la lumière de la lampe ni du soleil : car le Seigneur Dieu les éclaire, et ils régneront aux siècles des siècles.
\VS{6}Puis il me dit : ces paroles sont certaines et véritables, et le Seigneur le Dieu des saints Prophètes a envoyé son Ange, pour manifester à ses serviteurs les choses qui doivent arriver bientôt.
\VS{7}Voici, je viens bientôt ; bienheureux est celui qui garde les paroles de la prophétie de ce Livre.
\VS{8}Et moi Jean, je suis celui qui ai ouï et vu ces choses ; et après les avoir ouïes et vues, je me jetai à terre pour me prosterner aux pieds de l'Ange qui me montrait ces choses.
\VS{9}Mais il me dit : garde-toi de le faire ; car je suis ton Compagnon de service, et [le Compagnon] de tes frères les Prophètes, et de ceux qui gardent les paroles de ce Livre ; adore Dieu.
\VS{10}Il me dit aussi : ne cachette point les paroles de la prophétie de ce Livre, parce que le temps est proche.
\VS{11}Que celui qui est injuste, soit injuste encore ; et que celui qui est souillé se souille encore ; et que celui qui est juste, soit plus juste encore ; et que celui qui est saint, soit sanctifié encore.
\VS{12}Or voici, je viens bientôt ; et ma récompense est avec moi, pour rendre à chacun selon son œuvre.
\VS{13}Je suis l'Alpha et l'Oméga, le premier et le dernier, le commencement et la fin.
\VS{14}Bienheureux sont ceux qui font ses commandements, afin qu'ils aient droit à l'Arbre de vie, et qu'ils entrent par les portes dans la Cité.
\VS{15}Mais les Chiens, les empoisonneurs, les fornicateurs, les meurtriers, les idolâtres, et quiconque aime et commet fausseté seront [laissés] dehors.
\VS{16}Moi Jésus, j'ai envoyé mon Ange pour vous confirmer ces choses dans les Eglises. Je suis la racine et la postérité de David ; l'étoile brillante du matin.
\VS{17}Et l'Esprit et l'Epouse disent : viens. ; que celui aussi qui l’entend, dise : viens ; et que celui qui a soif, vienne ; et quiconque veut de l'eau vive, en prenne, sans qu'elle lui coûte rien.
\VS{18}Or je proteste à quiconque entend les paroles de la prophétie de ce Livre, que si quelqu'un ajoute à ces choses, Dieu fera tomber sur lui les plaies écrites dans ce Livre.
\VS{19}Et si quelqu'un retranche quelque chose des paroles du Livre de cette prophétie, Dieu lui enlèvera la part qu'il a dans le Livre de vie, dans la sainte Cité, et dans les choses qui sont écrites dans ce Livre.
\VS{20}Celui qui rend témoignage de ces choses, dit : Certainement je viens bientôt, Amen ! Oui, Seigneur Jésus, viens !
\VS{21}Que la grâce de notre Seigneur Jésus-Christ soit avec vous tous, Amen !
\PPE{}
\end{multicols}
