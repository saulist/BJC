\ShortTitle{Esther}\BookTitle{Esther}\BFont
\begin{multicols}{2}
\TextTitle{[Un festin de sept jour au palais de Suse]}
\Chap{1}
\VerseOne{}Or il arriva qu’au temps d’Assuérus (1), de cet Assuérus qui régnait depuis les Indes jusqu'en Ethiopie, sur cent vingt-sept provinces ;
\VS{2}[Il arriva, dis-je], en ce temps-là, que le roi Assuérus était assis sur le trône royal à Suse, dans la capitale.
\VS{3}La troisième année de son règne, il fit un festin à tous les principaux princes de ses pays ; à ses serviteurs, à l’armée des Perses et de Mèdes, aux nobles et aux chefs des provinces qui furent réunis devant lui,
\VS{4}pour leur montrer la gloire de la richesse de son royaume et la splendeur de sa grandeur, durant plusieurs jours, pendant cent quatre-vingts jours.
\VS{5}Lorsque ces jours furent achevés, le roi fit pour tout le peuple qui se trouvait à Suse, la capitale, depuis le plus grand jusqu'au plus petit, un festin pendant sept jours, dans la cour du jardin du palais royal.
\VS{6}Des étoffes blanches, vertes et violettes, étaient attachées par des cordons de byssus et de pourpre à des anneaux d'argent et à des colonnes de marbre. Les lits étaient d'or et d'argent sur un pavé de porphyre, de marbre, de nacre, et de pierres noires.
\VS{7}On servait à boire dans des vases d'or, de différentes espèces, et il y avait du vin royal en abondance, selon la libéralité du roi.
\VS{8}On ne forçait personne à boire, car le roi avait ordonné à tous les chefs de sa maison de se conformer à la volonté de chacun.
\VS{9}La reine Vasthi fit aussi un festin aux femmes dans la maison royale du roi Assuérus.
\TextTitle{[Destitution de la reine Vasthi]}
\VS{10}Or le septième jour, le cœur du roi était réjoui par le vin, il ordonna à Mehuman, Biztha, Harbona, Bigtha, Abagtha, Zéthar, et Carcas, les sept eunuques qui servaient devant le roi Assuérus,
\VS{11}d’amener en sa présence la reine Vasthi, portant la couronne royale, afin de montrer sa beauté aux peuples et aux princes, car elle était belle de figure.
\VS{12}Les eunuques transmirent l’ordre du roi à la reine Vasthi, mais elle refusa de venir. Et le roi fut très irrité, et il s’enflamma de colère.
\VS{13}Alors le roi dit aux sages qui avaient la connaissance des temps. Car le roi traitait ainsi les affaires en présence de tous ceux qui connaissaient les lois et le droit.
\VS{14}Il avait auprès de lui, Carschena, Schéthar, Admatha, Tarsis, Mérès, Marsena, Memucan, sept princes de Perse et de Médie, qui voyaient la face du roi et qui occupaient le premier rang dans le royaume.
\VS{15}Que faut-il faire dit-il, selon les lois, à la reine Vasthi, pour n'avoir pas observé l’ordre que le roi Assuérus lui a ordonné par les eunuques ?
\VS{16}Alors Memucan répondit en présence du roi et des princes : La reine Vasthi n'a pas seulement mal agi contre le roi, mais aussi contre tous les princes et tous les peuples qui sont dans toutes les provinces du roi Assuérus.
\VS{17}Car l'action de la reine parviendra à la connaissance de toutes les femmes, et les portera à mépriser leurs maris ; elles diront : Le roi Assuérus avait ordonné qu'on fasse venir en sa présence la reine, et elle n'y est pas allée.
\VS{18}Dès ce jour, les princesses de Perse et de Médie qui auront appris l’action de la reine répondront de même à tous les princes du roi ; ce sera une marque de mépris et un sujet de colère.
\VS{19}Si le roi le trouve bon, qu'un édit royal soit publié de sa part, et qu'il soit écrit parmi les lois de Perse et de Médie, avec défense de la transgresser, que Vasthi ne vienne plus devant le roi Assuérus et le roi donnera sa royauté à une compagne, qui sera meilleure qu'elle.
\VS{20}L’édit du roi sera présenté et connu dans tout son royaume, quelque grand qu'il soit, toutes les femmes honoreront leurs maris (1), depuis le plus grand jusqu'au plus petit.
\VS{21}Cette parole plut au roi et aux princes, et le roi fit selon la parole de Memucan.
\VS{22}Il envoya des lettres à toutes les provinces du royaume, à chaque province selon son écriture et à chaque peuple selon sa langue ; elles portaient que tout homme devait être le maître de sa maison (2), et qu’il parlerait la langue de son peuple.
\TextTitle{[Sélection d'une autre reine]}
\Chap{2}
\VerseOne{}Après ces choses, quand la colère du roi Assuérus fut calmée, il se souvint de Vasthi, de ce qu'elle avait fait, et de ce qui avait été décrété à son sujet.
\VS{2}Les serviteurs qui servaient le roi dirent : Qu'on cherche pour le roi des jeunes filles, vierges, et belles de figure.
\VS{3}Que le roi désigne des commissaires dans toutes les provinces de son royaume chargés de rassembler toutes les jeunes filles, vierges et belles de figure, dans Suse, la capitale, dans la maison des femmes sous la charge d'Hégué, eunuque du roi et gardien des femmes, qu'on leur donne les parfums nécessaires pour leur toilette ;
\VS{4}et la jeune fille qui plaira au roi régnera à la place de Vasthi. Ce discours plût au roi, et il fit ainsi.
\VS{5}Or, il y avait à Suse, la capitale, un juif nommé Mardochée, fils de Jaïr, fils de Schimeï, fils de Kis, Benjamite,
\VS{6}qui avait été emmené de Jérusalem (1), parmi les captifs déportés avec Jeconia, roi de Juda, par Nebucadnetsar, roi de Babylone.
\VS{7}Il élevait Hadassa, qui est Esther, fille de son oncle ; car elle n'avait ni père ni mère. La jeune fille était belle de taille et très belle de figure. Après la mort de son père et de sa mère, Mardochée l'avait prise pour fille.
\VS{8}Lorsqu’on eut publié l’ordre du roi et son édit, un grand nombre de jeunes filles furent rassemblées à Suse, la capitale, sous la charge d'Hégaï. Esther fut aussi amenée dans la maison du roi, sous la charge d'Hégaï, gardien des femmes.
\VS{9}La jeune fille lui plut, et trouva grâce à ses yeux, il s’empressa de lui fournir les parfums nécessaires pour sa toilette, et pour sa subsistance, lui donna sept jeunes filles choisies, et établies dans la maison du roi, il lui fit changer d'appartement, et la logea, elle et ses servantes, dans le meilleur des appartements de la maison des femmes.
\VS{10}Esther ne fit connaître ni son peuple ni sa parenté, car Mardochée lui avait ordonné de ne rien raconter.
\VS{11}Tous les jours Mardochée allait et venait devant la cour de la maison des femmes, pour savoir comment se portait Esther, et comment on s’occupait d'elle.
\VS{12}Chaque jeune fille allait à son tour vers le roi Assuérus, après s’être conformée au décret concernant les femmes pendant douze mois (2). C'est ainsi que s'accomplissaient les jours de leurs préparatifs, six mois avec de l'huile de myrrhe, et six autres mois avec des aromates et des parfums en usage parmi les femmes.
\VS{13}C'est ainsi que la jeune fille entrait vers le roi ; et, quand elle passait de la maison des femmes à la maison du roi, on lui laissait prendre ce qu’elle voulait.
\VS{14}Elle y entrait le soir, et le matin elle retournait dans la seconde maison des femmes sous la charge de Schaaschgaz, eunuque du roi et gardien des concubines. Elle ne retournait plus vers le roi, à moins que le roi n’en ait le désir et qu'elle soit appelée par son nom.
\TextTitle{[Esther, reine de Suse]}
\VS{15}Quand son tour d’aller vers le roi fut arrivé, Esther, fille d'Abichaïl, oncle de Mardochée qui l’avait prise pour sa fille, ne demanda rien sinon ce qui fut ordonné par Hégaï, eunuque du roi et gardien des femmes. Esther trouva grâce aux yeux de tous ceux qui la voyaient.
\VS{16}Ainsi Esther fut amenée auprès du roi Assuérus, dans sa maison royale, le dixième mois, qui est le mois de Tébeth, la septième année de son règne.
\VS{17}Le roi aima Esther plus que toutes les autres femmes, elle obtint sa grâce et sa bienveillance plus que toutes les vierges. Il mit la couronne royale sur sa tête, et l'établit reine à la place de Vasthi.
\VS{18}Le roi fit alors un grand festin à tous les princes de ses pays, et à ses serviteurs, un festin en l’honneur d'Esther ; il donna du repos aux provinces, et fit des présents selon la puissance du roi.
\VS{19}Or pendant qu'on assemblait les vierges pour la seconde fois, Mardochée s’assit à la porte du roi.
\VS{20}Esther n’avait fait connaître ni sa parenté ni son peuple, car Mardochée le lui avait défendu. Elle faisait tout ce que lui disait Mardochée, comme à l’époque où elle était élevée par lui.
\TextTitle{[Mardochée sauve la vie du roi]}
\VS{21}En ces jours-là, Mardochée s’assit à la porte du roi, Bigthan et Théresch, deux eunuques du roi, gardes du seuil, s’irritèrent et cherchèrent à mettre la main sur le roi Assuérus.
\VS{22}Mardochée ayant eu connaissance de l’affaire, informa la reine Esther, qui le redit au roi de la part de Mardochée.
\VS{23}On vérifia l’affaire et on trouva que cela était exact, les deux eunuques furent pendus à un bois, et cela fut écrit dans le livre des chroniques en présence du roi.
\TextTitle{[Conspiration de Haman contre les Juifs]}
\Chap{3}
\VerseOne{}Après ces choses, le roi Assuérus fit de grands honneurs à Haman, fils d'Hammedatha, l’Agaguite ; il l'éleva en dignité et plaça son siège au-dessus de tous les princes qui étaient auprès de lui.
\VS{2}Tous les serviteurs du roi qui étaient à la porte du roi s'inclinaient et se prosternaient devant Haman, car le roi l’avait ainsi ordonné. Mais Mardochée ne s'inclinait pas et ne se prosternait pas devant lui.
\VS{3}Les serviteurs du roi, qui étaient à la porte du roi, disaient à Mardochée : Pourquoi transgresses-tu l’ordre du roi ?
\VS{4}Comme ils le lui répétaient chaque jour et qu'il ne les écoutait pas, ils le rapportèrent à Haman, pour voir si Mardochée tiendrait ferme dans sa résolution ; car il leur avait déclaré qu'il était juif.
\VS{5}Haman vit que Mardochée ne s'inclinait pas et ne se prosternait pas devant lui et il fut rempli de colère.
\VS{6}Mais il dédaigna de porter la main sur Mardochée seul, car on lui avait rapporté de quel peuple était Mardochée. Haman chercha à exterminer tous les juifs, le peuple de Mardochée qui se trouvait dans tout le royaume d'Assuérus.
\VS{7}Au premier mois, qui est le mois de Nissan, la douzième année du roi Assuérus, on jeta le pur, c'est-à-dire le sort, devant Haman, pour chaque jour et pour chaque mois, jusqu’au douzième mois, qui est le mois d'Adar.
\VS{8}Haman dit au roi Assuérus : Il y a un peuple dispersé dans toutes les provinces de ton royaume, qui se tient à part parmi les peuples. Leurs lois sont différentes de celles de tous les autres peuples, ils n’observent pas les lois du roi. Il n'est pas dans l’intérêt du roi de le laisser en repos.
\VS{9}S'il plaît au roi, qu'on écrive l’ordre de les faire périr, et je pèserai dix mille talents d'argent entre les mains de ceux qui s’occupent des affaires, pour les porter dans le trésor du roi.
\VS{10}Le roi ôta son anneau de sa main, et le donna à Haman fils de Hammedatha, l’Agaguite, l’adversaire des Juifs.
\VS{11}Outre cela, le roi dit à Haman : Cet argent t'est donné avec ce peuple ; fais-en ce que tu voudras.
\VS{12}Le treizième jour du premier mois, les secrétaires du roi furent appelés, et on écrivit selon l’ordre d'Haman, aux satrapes du roi, aux gouverneurs de chaque province et aux princes de chaque peuple, à chaque province selon son écriture et à chaque peuple selon sa langue. Ce fut au nom du roi Assuérus que l’on écrivit, et on scella avec l'anneau du roi.
\VS{13}Les lettres furent envoyées par des coureurs dans toutes les provinces du roi, afin qu'on extermine, qu’on tue et qu’on fasse périr tous les juifs, jeunes et vieux, petits enfants et femmes, en un seul jour, le treizième du douzième mois, qui est le mois d'Adar, et pour que leurs biens soient livrés au pillage.
\VS{14}Ces lettres qui furent écrites portaient une copie de l’édit, qui devait être publié dans chaque province, et invitaient publiquement tous les peuples, à se tenir prêts pour ce jour-là.
\VS{15}Ainsi les coureurs partirent en toute hâte d’après l’ordre du roi. L'édit fut aussi publié dans Suse, la capitale. Or le roi et Haman étaient assis pour boire, pendant que la ville de Suse était dans la confusion.
\TextTitle{[Esther avertie du complot d'Haman]}
\Chap{4}
\VerseOne{}Mardochée, ayant appris ce qui se passait, déchira ses vêtements et se couvrit d'un sac et de la cendre. Puis il alla au milieu de la ville en poussant avec force des cris amers,
\VS{2}et se rendit jusqu'à la porte du roi, or il était interdit d'entrer dans le palais du roi revêtu d'un sac.
\VS{3}Dans chaque province, partout où arrivait l’ordre du roi et son édit, il y eut une grande désolation parmi les juifs ; ils jeûnaient, pleuraient, gémissaient, et beaucoup se couchaient sur le sac et la cendre.
\VS{4}Les servantes d'Esther et ses eunuques vinrent lui raconter ces choses, et la reine fut très effrayée. Elle envoya des vêtements à Mardochée pour le couvrir et lui faire ôter son sac, mais il ne les prit pas.
\VS{5}Alors Esther appela Hathac, l'un des eunuques que le roi avait établi pour la servir, et elle le chargea de demander à Mardochée ce qui s’était passé et pourquoi il agissait ainsi.
\VS{6}Hathac sortit donc vers Mardochée sur la place de la ville, devant la porte du roi.
\VS{7}Mardochée lui raconta tout ce qui lui était arrivé, et la somme d'argent qu'Haman avait promis de payer comptant au trésor du roi, pour la destruction des juifs.
\VS{8}Il lui donna aussi une copie de l'édit publié dans Suse en vue de leur extermination, afin qu’il le montre à Esther et lui fasse tout connaître ; et il ordonna qu’Esther se rende chez le roi pour implorer sa miséricorde, et faire une requête en faveur de son peuple.
\VS{9}Hathac vint rapporter à Esther les paroles de Mardochée.
\TextTitle{[Mardochée incite Esther à risquer sa vie pour ses frères]}
\VS{10}Esther chargea Hathac de dire à Mardochée :
\VS{11}Tous les serviteurs du roi et le peuple des provinces du roi savent qu'il existe une loi prescrivant la peine de mort contre quiconque, homme ou femme, entre chez le roi, dans la cour intérieure sans avoir été appelé ; à moins que le roi ne lui tende le sceptre d'or, celui-là a la vie sauve. Or il y a déjà trente jours que je n'ai pas été appelée pour entrer chez le roi.
\VS{12}On rapporta les paroles d'Esther à Mardochée.
\VS{13}Mardochée fit cette réponse à Esther : Ne t’imagine pas que tu échapperas seule d'entre tous les juifs parce que tu es dans la maison du roi.
\VS{14}Mais si tu te tais et gardes le silence en ce temps-ci, les juifs seront secourus et délivrés par un autre moyen, mais toi et la maison de ton père vous périrez. Et qui sait si tu n'es pas arrivée à la royauté pour un temps comme celui-ci ?
\TextTitle{[Esther demande un jeûne]}
\VS{15}Esther fit cette réponse à Mardochée :
\VS{16}Va, rassemble tous les juifs qui se trouvent à Suse, et jeûnez pour moi, sans manger ni boire pendant trois jours, ni la nuit ni le jour. Moi aussi et mes servantes nous jeûnerons de même, puis j'entrerai chez le roi, malgré la loi ; et si je dois périr, je périrai.
\VS{17}Mardochée s'en alla, et fit comme Esther lui avait ordonné.
\TextTitle{[Esther se présente devant le roi]}
\Chap{5}
\VerseOne{}Le troisième jour, Esther mit des vêtements royaux et se présenta dans la cour intérieure de la maison du roi, devant la maison du roi. Le roi était assis sur le trône dans la maison royale, en face de l’entrée de la maison.
\VS{2}Dès que le roi vit la reine Esther debout dans la cour, elle trouva grâce à ses yeux ; le roi tendit à Esther le sceptre d'or qui était dans sa main. Esther s'approcha, et toucha le bout du sceptre.
\VS{3}Le roi lui dit : Qu'as-tu, reine Esther, et que demandes-tu ? Quand ce serait la moitié du royaume, elle te serait donnée.
\VS{4}Esther répondit : Si le roi le trouve bon, que le roi vienne aujourd'hui avec Haman au festin que je lui ai préparé.
\VS{5}Alors le roi dit : Qu'on fasse venir en toute hâte, Haman, pour accomplir la parole d'Esther. Le roi vint donc avec Haman au festin qu'Esther avait préparé.
\VS{6}Le roi dit à Esther, pendant qu’on buvait le vin : Quelle est ta demande ? Elle te sera accordée. Quelle est ta requête ? Quand ce serait la moitié du royaume, tu l’obtiendras.
\VS{7}Esther répondit et dit : Voici ce que je demande et ce que je désire.
\VS{8}Si j'ai trouvé grâce aux yeux du roi, et si le roi trouve bon d'accorder ma requête, que le roi et Haman viennent au festin que je leur préparerai, et je donnerai demain une réponse au roi selon sa parole.
\VS{9}Haman sortit ce jour-là, joyeux et le cœur content. Mais aussitôt qu'il vit, à la porte du roi, Mardochée, qui ne se levait ni ne tremblait devant lui, il fut rempli de colère contre Mardochée.
\VS{10}Il sut toutefois se contenir, et il alla dans sa maison. Puis il envoya chercher ses amis et Zéresch, sa femme.
\VS{11}Haman leur parla de la magnificence de ses richesses, du nombre de ses fils, et tout ce qu’avait fait le roi pour le rendre puissant, et comment il l'avait élevé au-dessus des princes et des serviteurs du roi.
\VS{12}Puis Haman ajouta : Même la reine Esther n'a fait venir que moi et le roi au festin qu'elle a fait, et je suis encore invité demain chez elle avec le roi.
\VS{13}Mais tout cela n’est d’aucun intérêt, aussi longtemps que je verrai Mardochée, le juif, assis à la porte du roi.
\VS{14}Zéresch sa femme, et tous ses amis lui répondirent : Qu'on prépare un bois haut de cinquante coudées, et demain matin dis au roi qu'on y pende Mardochée ; et tu iras joyeux au festin avec le roi. Cette parole plut à Haman, et il fit préparer le bois.
\TextTitle{[Le roi Assuérus se souvient de Mardochée]}
\Chap{6}
\VerseOne{}Cette nuit-là, le roi ne put dormir, il fit apporter le livre des annales, les chroniques. On les lut devant le roi,
\VS{2}et l’on trouva écrit ce que Mardochée avait rapporté au sujet de la conspiration de Bigthan et de Théresch, les deux eunuques du roi, gardes du seuil, qui avaient cherché à mettre la main sur le roi Assuérus.
\VS{3}Le roi dit : Quel honneur et quelle distinction a-t-on accordé à Mardochée pour cela ? Il n’a rien reçu répondirent les serviteurs du roi.
\VS{4}Le roi dit : Qui est dans la cour ? Haman était venu dans la cour extérieure de la maison du roi, pour demander au roi de pendre Mardochée au bois qu'il avait préparé.
\VS{5}Les serviteurs du roi répondirent : C’est Haman qui se tient dans la cour. Et le roi dit : Qu'il entre.
\VS{6}Haman entra, et le roi lui dit : Que faudrait-il faire à un homme que le roi désire honorer ? Haman se dit en lui-même : A qui le roi voudrait-il faire plus honneur qu'à moi ?
\VS{7}Haman répondit au roi : Pour un homme que le roi désire honorer,
\VS{8}qu’on lui apporte le vêtement royal, dont le roi se revêt, et qu'on lui amène le cheval que le roi monte, et qu'on lui mette la couronne royale sur la tête.
\VS{9}Et qu'ensuite on donne ce vêtement et ce cheval à quelqu'un des principaux et des plus grands chefs qui sont auprès du roi, et qu'on revête l'homme que le roi prend plaisir d'honorer, et qu'on le fasse aller à cheval par les rues de la ville ; et qu'on crie devant lui : C'est ainsi qu'on doit faire à l'homme que le roi prend plaisir d'honorer.
\VS{10}Alors le roi dit à Haman : Prends tout de suite le vêtement, et le cheval, comme tu l'as dit, et fais ainsi à Mardochée, le juif qui est assis à la porte du roi ; ne néglige rien de tout ce que tu as déclaré.
\VS{11}Et Haman prit le vêtement et le cheval, il revêtit Mardochée, il le promena à cheval à travers les rues de la ville, et il criait devant lui : C'est ainsi que l’on fait à l'homme que le roi désire honorer.
\VS{12}Mardochée retourna à la porte du roi, et Haman se retira en hâte dans sa maison, pleurant et ayant la tête voilée.
\VS{13}Haman raconta à Zéresch, sa femme, et à tous ses amis, tout ce qui lui était arrivé. Ses sages, et Zéresch, sa femme, lui répondirent : Si Mardochée devant lequel tu as commencé à tomber, est de la race des juifs, tu n'auras pas le dessus sur lui, mais tu tomberas certainement devant lui.
\VS{14}Comme ils parlaient encore avec lui, les eunuques du roi vinrent, et se hâtèrent d'amener Haman au festin qu'Esther avait préparé.
\TextTitle{[Esther plaide sa cause et celle de son peuple]}
\Chap{7}
\VerseOne{}Le roi et Haman allèrent au festin chez la reine Esther.
\VS{2}Le roi dit encore à Esther, ce second jour, pendant qu’on buvait le vin : Quelle est ta demande, reine Esther ? Elle te sera donnée. Que désires-tu ? Quand ce serait la moitié du royaume, cela te sera accordé.
\VS{3}Alors la reine Esther répondit, et dit : Si j'ai trouvé grâce à tes yeux, ô roi ! et si le roi le trouve bon, que ma vie me soit donnée à ma demande, et que mon peuple me soit donné à ma prière.
\VS{4}Car nous avons été vendus, mon peuple et moi, pour être détruits, tués, exterminés. Si nous avions été vendus pour être esclaves et serviteurs, j’aurais gardé le silence, bien que l'oppresseur ne saurait compenser le dommage fait au roi.
\VS{5}Le roi Assuérus parla et dit à la reine Esther : Qui est-il et où est l’homme dont le cœur est consacré à faire cela ?
\VS{6}Esther répondit : L'oppresseur, l'ennemi, c’est Haman, ce méchant ! Alors Haman fut terrifié en présence du roi et de la reine.
\TextTitle{[Haman pendu au gibet qu'il avait dressé]}
\VS{7}Le roi, dans sa colère, se leva et quitta le festin, il entra dans le jardin du palais. Haman resta pour demander grâce pour sa vie à la reine Esther, car il voyait bien que sa perte était résolue par le roi.
\VS{8}Puis le roi revint du jardin du palais dans la salle du festin, il vit Haman qui s’était précipité sur le lit où était Esther, et il dit : Serait-ce encore pour faire violence sous mes yeux à la reine dans cette maison ? Dès que la parole fut sortie de la bouche du roi, on voila le visage d'Haman.
\VS{9}Et Harbona, l'un des eunuques, dit en présence du roi : Voici, le bois préparé par Haman pour Mardochée, qui a parlé pour le bien du roi, est dressé dans la maison d'Haman, à une hauteur de cinquante coudées. Le roi dit : Qu’on y pende Haman !
\VS{10}On pendit Haman au bois qu'il avait préparé pour Mardochée. Et la colère du roi fut apaisée.
\TextTitle{[Un décret royal fait échouer le complot d'Haman]}
\Chap{8}
\VerseOne{}Ce même jour, le roi Assuérus donna à la reine Esther la maison d'Haman, l'oppresseur des juifs ; et Mardochée fut introduit devant le roi, car Esther avait déclaré quel était son lien de parenté avec elle.
\VS{2}Le roi ôta son anneau, qu'il avait repris à Haman, et le donna à Mardochée ; Esther établit Mardochée sur la maison d'Haman.
\VS{3}Esther parla encore en présence du roi. Elle se jeta à ses pieds, elle pleura, elle l’implora d’empêcher les effets de la méchanceté d'Haman, l’Agaguite, et la réussite de ses projets contre les juifs.
\VS{4}Le roi tendit le sceptre d'or à Esther qui se releva et resta debout devant le roi.
\VS{5}Elle dit : Si le roi le trouve bon, et si j'ai trouvé grâce devant lui, si mes paroles semblent convenables au roi et si je suis agréable à ses yeux, qu'on écrive pour révoquer les lettres conçues par Haman, fils d'Hammedatha, l’Agaguite, qu'il écrivit afin de détruire les juifs qui sont dans toutes les provinces du roi.
\VS{6}Car comment pourrais-je voir le mal qui atteindrait mon peuple, et comment pourrais-je voir la destruction de ma race ?
\VS{7}Le roi Assuérus dit à la reine Esther et au juif Mardochée : Voici, j'ai donné la maison d'Haman à Esther, et il a été pendu au bois pour avoir étendu sa main contre les juifs.
\VS{8}Ecrivez donc, au nom du roi, en faveur des juifs comme il vous plaira, et scellez l'écrit de l'anneau du roi ; car un édit écrit au nom du roi et scellé de l'anneau du roi ne peut être révoqué.
\VS{9}En ce temps, le vingt-troisième jour du troisième mois, qui est le mois de Sivan, les secrétaires du roi furent appelés, et on écrivit, comme Mardochée l’ordonna, aux juifs, aux satrapes, aux gouverneurs, et aux princes des cent vingt-sept provinces, de l’Inde jusqu'en Ethiopie, à chaque province selon son écriture, à chaque peuple selon sa langue, et aux juifs selon leur écriture et selon leur langue.
\VS{10}On écrivit les lettres au nom du roi Assuérus, et on les scella de l'anneau du roi. On les envoya par des coureurs, ayant pour montures des chevaux et des mulets nés de juments.
\VS{11}Par ces lettres, le roi accordait aux juifs, qui étaient dans chaque ville la permission de se rassembler et de défendre leur vie, de détruire, de tuer, et d’exterminer toute force armée du peuple et de quelque province que ce soit, qui prendraient les armes pour les attaquer, ainsi que leurs petits enfants et leurs femmes, et de piller leurs biens ;
\VS{12}et cela en un seul jour, dans toutes les provinces du roi Assuérus, le treizième jour du douzième mois, qui est le mois d'Adar.
\VS{13}Ces lettres écrites portaient une copie de l’édit qui devait être publié dans chaque province, et informaient tous les peuples que les juifs seraient prêts en ce jour à se venger de leurs ennemis.
\VS{14}Les coureurs, montés sur des chevaux et des mulets, partirent aussitôt et en toute hâte, d’après l’ordre du roi. L'édit fut aussi publié dans Suse, la capitale.
\TextTitle{[Mardochée honoré]}
\VS{15}Mardochée sortit de chez le roi, en vêtement royal violet et blanc, avec une grande couronne d'or, et une robe de byssus et de pourpre. La ville de Suse poussait des cris, et elle fut dans la joie.
\VS{16}Il eut pour les juifs du bonheur et de la joie, des réjouissances et des honneurs.
\VS{17}Dans chaque province et dans chaque ville, partout où arrivaient l’ordre du roi et son décret, il y eut pour les juifs de la joie, des réjouissances, des festins, et des fêtes. Et beaucoup de gens d'entre les peuples du pays se faisaient juifs, parce que la crainte des juifs les avait saisis.
\TextTitle{[Les juifs triomphent de leurs ennemis]}
\Chap{9}
\VerseOne{}Le douzième mois, qui est le mois d'Adar, le treizième jour du mois, où l’ordre du roi et son décret devaient être exécutés, au jour où les ennemis des juifs espéraient dominer, ce fut le contraire qui arriva, les juifs dominèrent sur leurs ennemis.
\VS{2}Les juifs se rassemblèrent dans leurs villes, dans toutes les provinces du roi Assuérus, pour mettre la main sur ceux qui cherchaient leur perte ; et personne ne put leur résister, car la crainte qu'on avait d'eux avait saisi tous les peuples.
\VS{3}Et tous les princes des provinces, les satrapes, les gouverneurs, et ceux qui s’occupaient des affaires du roi, soutenaient les juifs, à cause de la terreur que leur inspirait Mardochée.
\VS{4}Car Mardochée était puissant dans la maison du roi, et sa renommée se répandait dans toutes les provinces, parce qu’il devenait de plus en plus puissant.
\VS{5}Les juifs frappèrent tous leurs ennemis à coups d'épée, ils les tuèrent et les détruisirent ; ils traitèrent selon leurs désirs ceux qui les haïssaient.
\VS{6}Dans Suse, la capitale, les juifs tuèrent et firent périr cinq cents hommes.
\VS{7}Ils tuèrent aussi Parschandatha, Dalphon, Aspatha,
\VS{8}Poratha, Adalia, Aridatha,
\VS{9}Parmaschtha, Arizaï, Aridaï, et Vajezatha,
\VS{10}les dix fils d'Haman, fils d'Hammedatha, l'oppresseur des juifs. Mais ils ne mirent pas leurs mains au pillage.
\VS{11}Ce jour-là, on rapporta au roi le nombre de ceux qui avaient été tués dans Suse, la capitale.
\VS{12}Le roi dit à la reine Esther : Dans Suse, la capitale, les juifs ont tué et détruit cinq cents hommes, et les dix fils d'Haman, qu'auront-ils fait dans le reste des provinces du roi ? Quelle est ta demande ? Et elle te sera accordée. Que désires-tu encore ? Tu l’obtiendras.
\VS{13}Esther répondit : Si le roi le trouve bon qu'il soit permis aux juifs, qui sont à Suse, d’agir encore demain selon le décret d’aujourd'hui, et que l'on pende au bois les dix fils d'Haman.
\VS{14}Et le roi ordonna de faire ainsi. L'édit fut publié dans Suse. On pendit les dix fils d'Haman ;
\VS{15}et les juifs qui étaient dans Suse se rassemblèrent encore le quatorzième jour du mois d'Adar et tuèrent dans Suse trois cents hommes. Mais ils ne mirent pas la main au pillage.
\VS{16}Les autres juifs qui étaient dans les provinces du roi se rassemblèrent, et défendirent leur vie ; ils eurent du repos et furent délivrés de leurs ennemis, et ils tuèrent soixante-quinze mille hommes de ceux qui les haïssaient. Mais ils ne mirent pas la main au pillage.
\VS{17}Ces choses arrivèrent le treizième jour du mois d'Adar, et le quatorzième du même mois ils se reposèrent, et ils en firent un jour de festin et de joie.
\VS{18}Les juifs qui étaient dans Suse, s'assemblèrent le treizième et le quatorzième jour du même mois, et ils se reposèrent le quinzième jour, et ils en firent un jour de festin et de joie.
\VS{19}C'est pourquoi les juifs des campagnes qui habitent dans des villes sans murailles, font le quatorzième jour du mois d'Adar, un jour de réjouissance, de festin et de fête, où l’on s’envoie des portions les uns aux autres.
\TextTitle{[Esther confirme l'instauration la fête des Purim]}
\VS{20}Mardochée écrivit ces choses, et il envoya les lettres à tous les juifs qui étaient dans toutes les provinces du roi Assuérus, auprès et au loin.
\VS{21}Il leur prescrivait de célébrer chaque année le quatorzième jour et le quinzième jour du mois d'Adar.
\VS{22}Comme les jours où les juifs avaient obtenu du repos en se délivrant de leurs ennemis, de célébrer le mois où leur angoisse fut changée en joie, et leur deuil en jour heureux, et de faire de ces jours des jours de festin et de joie, où l’on s’envoie des portions les uns aux autres, et des dons aux pauvres.
\VS{23}Les juifs s’engagèrent à faire ce qu’ils avaient déjà commencé et ce que Mardochée leur prescrivit.
\VS{24}Car Haman, fils d'Hammedatha, l’Agaguite, l'oppresseur de tous les juifs, avait projeté de détruire les juifs, et il avait jeté le pur, c'est-à-dire le sort, afin de les détruire et de les tuer ;
\VS{25}mais Esther s’étant présentée devant le roi, le roi ordonna par écrit que le méchant projet qu'Haman avait imaginé contre les juifs, retombe sur sa tête, et qu'on le pende au bois, lui et ses fils.
\VS{26}C'est pourquoi on appelle ces jours-là purim, du nom de pur (1). D’après tout le contenu de cette lettre, et selon ce qu’ils avaient eux-mêmes vu et ce qui leur était arrivé,
\VS{27}Les juifs établirent et adoptèrent pour eux, pour leur postérité, et pour tous ceux qui s’attacheraient à eux, l’engagement de ne pas manquer de célébrer chaque année ces deux jours, selon le mode prescrit et au temps fixé.
\VS{28}Ces jours devaient être rappelés et observés de génération en génération, dans chaque famille, dans chaque province et dans chaque ville ; et ces jours de Purim ne devaient jamais être abolis au milieu des juifs, ni le souvenir s’en effacer parmi leurs descendants.
\VS{29}La reine Esther, fille d'Abichaïl, écrivit aussi avec le juif Mardochée, de manière pressante pour la seconde fois, pour confirmer la lettre sur les Purim.
\VS{30}On envoya des lettres à tous les juifs, dans les cent vingt-sept provinces du royaume d'Assuérus. Elles contenaient des paroles de paix et de vérité,
\VS{31}pour établir ces jours de Purim au temps fixé, comme Mardochée le juif et la reine Esther les avaient établis pour eux, et comme ils les avaient établis pour eux-mêmes et pour leur postérité, à l’occasion de leur jeûne et de leurs cris.
\VS{32}Ainsi l'édit d'Esther confirma l’institution des Purim, et cela fut écrit dans le livre.
\TextTitle{[Mardochée établi dans la cours du roi]}
\Chap{10}
\VerseOne{}Le roi Assuérus imposa un tribut au pays, et aux îles de la mer.
\VS{2}Tous les faits concernant ses exploits, et les détails sur la grandeur à laquelle le roi éleva Mardochée, ne sont-ils pas écrits dans le livre des chroniques des rois de Médie et de Perse ?
\VS{3}Car Mardochée le juif était le premier après le roi Assuérus ; grand parmi les juifs et agréable à la multitude de ses frères, il chercha le bien-être de son peuple, et parla pour la paix de toute sa race.
\PPE{}
\end{multicols}
