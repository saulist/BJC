\ShortTitle{2 Thessaloniciens}\BookTitle{2 Thessaloniciens}\BFont
\noindent\hrulefill
{\footnotesize
\textit{
\bigskip
{\centering{}
\\Thème : Le jour de Christ
\\Auteur : Paul
\\Date de rédaction : Env. 51\\}
}
%\bigskip
\textit{
\\Autrefois appelée Therme ou Therma qui signifie « source chaude », Thessalonique reçut son nouveau nom par  Cassandre l’un des successeurs  d’Alexandre le Grand (356 av. J.-C. - 323 av. J.-C.) en l’honneur de sa femme Thessalonike, sœur  de ce dernier. Cette ville est située au Nord de la Grèce actuelle, sur la côte de la mer Égée. Ce pays était divisé en deux parties. Dans la région du Nord, la Macédoine, se trouvaient les villes de Philippes, Thessalonique et Bérée. Quant à la région du Sud, l’Achaïe, comportait les villes d’Athènes et de Corinthe. Aujourd’hui, la ville s’appelle Salonique.
%\bigskip
\\La seconde lettre de Paul aux Thessaloniciens fut rédigée peu de temps après la première. Elle fut motivée par des troubles survenus dans la communauté à la suite d'une annonce basée sur une lettre faussement attribuée à Paul prétendant que le «~jour du Seigneur~» était arrivé. Paul, dans cette seconde épître, exhortait les chrétiens de Thessalonique à tenir ferme dans leur foi malgré la persécution et leur expliquait que le « jour de Christ » devait être précédé par l’apostasie et la venue de l’homme impie. Il conclut sa lettre en demandant aux chrétiens de s’éloigner de ceux qui vivaient dans le désordre.\bigskip
}
}
\par\nobreak\noindent\hrulefill
\begin{multicols}{2}
\TextTitle{[Introduction]}
\Chap{1}
\VerseOne{}Paul, et Silvain, et Timothée : A l'église des Thessaloniciens\FTNT{Thessalonique. Voir Ac. 17:1-9.} qui est en Dieu notre Père, et en notre Seigneur Jésus-Christ.
\VS{2}Que la grâce et la paix vous soient données de la part de Dieu notre Père, et de la part du Seigneur Jésus-Christ.
\VS{3}Mes frères, nous devons continuellement rendre grâces à Dieu à votre sujet, comme il est bien raisonnable, parce que votre foi fait de grands progrès et que votre charité à l’égard des autres augmente de plus en plus.
\VS{4}Ainsi nous nous glorifions de vous dans les églises de Dieu, à cause de votre persévérance et de votre foi au milieu de toutes vos persécutions, et des afflictions que vous avez à supporter,
\TextTitle{[Consolation dans l'affliction]}
\VS{5}qui sont une manifeste démonstration du juste jugement de Dieu, afin que vous soyez jugés dignes du Royaume de Dieu, pour lequel aussi vous souffrez.
\VS{6}Car il est juste devant Dieu qu'il rende l'affliction à ceux qui vous affligent ;
\VS{7}et qu'il vous donne du repos à vous qui êtes affligés, de même qu'à nous, lorsque le Seigneur Jésus se révélera\FTNT{Révélation, du grec «~apokalupsis~», signifie «~mettre à nu, révélation d’une vérité~». C’est l’usage d'événements par lequel les choses ou la nature de certains, jusqu'ici cachées deviennent visibles à tous.} du ciel avec les anges de sa puissance ;
\VS{8}avec des flammes de feu, pour exercer la vengeance contre ceux qui ne connaissent point Dieu, contre ceux qui n'obéissent point à l'Evangile de notre Seigneur Jésus-Christ.
\VS{9}Ils auront pour châtiment une ruine éternelle, loin de la face du Seigneur, et de la gloire de sa force,
\VS{10}quand il viendra pour être glorifié en ce jour-là par ses saints, et admiré par tous ceux qui auront cru ; parce que notre témoignage auprès de vous a été cru.
\VS{11}C'est pourquoi nous prions continuellement pour vous, afin que notre Dieu vous juge dignes de la vocation, et qu'il accomplisse puissamment en vous tout le bon plaisir de sa bonté, et l’œuvre de la foi,
\VS{12}afin que le Nom de notre Seigneur Jésus-Christ soit glorifié en vous, et que vous soyez glorifiés en lui, selon la grâce de notre Dieu et Seigneur Jésus-Christ.
\TextTitle{[Le jour du Seigneur et l'avènement de l'homme impie]}
\Chap{2}
\VerseOne{}Pour ce qui concerne l'avènement\FTNT{L’avènement du Seigneur Jésus-Christ. Voir Mt. 24:1-3.} de notre Seigneur Jésus-Christ et notre réunion en lui, mes frères, nous vous prions
\VS{2}de ne pas vous laisser subitement ébranler dans votre entendement, ni troubler par une inspiration, ni par une parole, ou par quelque lettre qu’on dirait venir de nous, comme si le jour de Christ était déjà là.
\VS{3}Que personne ne vous séduise d’aucune manière, car il faut que l’apostasie soit arrivée auparavant et qu’on ait vu paraître l’homme du péché, le fils de la perdition\FTNT{Il est question ici de l’homme impie, de l’antéchrist, qui est la bête qui monte de la mer décrite par l’apôtre Jean (Ap. 13:11-18). Voir aussi Dn. 11:36-38.},
\VS{4}lequel s'oppose et s'élève contre tout ce qui est nommé Dieu, ou qu'on adore, jusqu'à être assis comme Dieu dans le temple de Dieu\FTNT{Selon les chapitres 40 à 42 d’Ezéchiel, le culte lévitique sera restauré à la fin des temps, ce qui suppose nécessairement la reconstruction du temple de Jérusalem. Cette prophétie est actuellement (2014-2015) en train de s’accomplir puisque des juifs religieux militent activement pour la réalisation de ce projet. L’organisation la plus connue œuvrant en ce sens est l’Institut du temple (fondé en 1987) qui a déjà restauré un grand nombre d’objets servant au culte. Toutefois, il ne faut pas sous-estimer la ruse de Satan, car au-delà du temple physique, il cherche prioritairement à s’assoir dans les temples spirituels que sont les chrétiens (1 Co. 6:19). Pour parvenir à ses fins, Satan a envoyé plusieurs de ses émissaires pour prêcher un autre évangile et un autre christ. C’est ainsi que de nombreuses assemblées, séduites et captivées par de faux docteurs, n’ont plus Jésus-Christ comme Seigneur, mais Satan en personne. L’apostasie étant installée premièrement dans les cœurs, l’antéchrist n’aura donc aucun mal à se faire passer pour le Christ et à s’asseoir dans le temple physique où il usurpera l’adoration qui revient au Dieu véritable.} voulant se faire passer pour un Dieu.
\VS{5}Ne vous souvenez-vous pas que je vous disais ces choses, lorsque j’étais encore chez vous ?
\VS{6}Et maintenant vous savez ce qui le retient, afin qu'il soit révélé en son temps.
\VS{7}Car le mystère de l'iniquité\FTNT{Le mystère de l’iniquité. Paul nous enseigne que ce mystère était déjà à l’œuvre au sein des églises primitives. Le prophète Zacharie, au chapitre 5 de son livre, l’avait personnifié en relatant une vision dans laquelle il vit «~deux femmes avec des ailes de cigogne~» emportant l’épha de l’iniquité des enfants d’Israël. Sur cet épha était assise une femme personnifiant l’iniquité, c’est-à-dire la femme de l’homme impie, la Babylone religieuse. Ces deux femmes aux ailes de cigogne allaient lui bâtir une maison au pays de Schinéar (Babylone selon Genèse 10:6-14).} agit déjà ; il faut seulement que celui qui le retient encore ait disparu.
\VS{8}Et alors sera révélé le méchant\FTNT{Es. 11:4}, que le Seigneur détruira par le souffle de sa bouche, et qu’il anéantira par l’éclat de son avènement :
\VS{9}L'avènement\FTNT{Il y aura un autre avènement, celui de l’homme impie.} de cet impie se fera par la puissance de Satan, avec toutes sortes de miracles, de signes, et de prodiges mensongers,
\VS{10}Et avec toutes les séductions de l’iniquité, pour ceux qui périssent parce qu'ils n'ont pas reçu l'amour de la vérité, pour être sauvés.
\VS{11}C'est pourquoi Dieu leur envoie-t-il une puissance d’égarement\FTNT{L’esprit d’égarement. Voir Ro. 1:26,28 ; 1 R. 22.}, pour qu’ils croient au mensonge ;
\VS{12}afin que tous ceux qui n’ont pas cru à la vérité, mais qui ont pris plaisir à l’iniquité soient condamnés.
\TextTitle{[Diverses exhortations]}
\VS{13}Pour nous, mes frères, bien-aimés du Seigneur, nous devons toujours rendre grâces à Dieu pour vous, de ce que Dieu vous a élus dès le commencement pour le salut par la sanctification de l'Esprit, et par la foi de la vérité.
\VS{14}C’est à quoi il vous a appelés par notre Evangile, afin que vous possédiez la gloire qui nous a été acquise par notre Seigneur Jésus-Christ.
\VS{15}C'est pourquoi, mes frères, demeurez fermes, et retenez les enseignements que vous avez appris, soit par notre parole, soit par notre lettre.
\VS{16}Or lui-même Jésus-Christ notre Seigneur et notre Dieu et Père, qui nous a aimés et qui nous a donné une consolation éternelle et une bonne espérance par sa grâce,
\VS{17}console vos cœurs, et vous affermisse en toute bonne parole, et en toute bonne œuvre.
\TextTitle{[Paul demande la prière]}
\Chap{3}
\VerseOne{}Au reste, mes frères, priez pour nous, afin que la parole du Seigneur se répande et soit glorifiée comme elle l'est chez vous,
\VS{2}et que nous soyons délivrés des hommes méchants et pervers, car tous n’ont pas la foi.
\VS{3}Le Seigneur est fidèle, il vous affermira et vous gardera du mal.
\VS{4}Nous avons à votre égard cette confiance dans le Seigneur que vous faites et que vous ferez les choses que nous recommandons.
\VS{5}Que le Seigneur veuille diriger vos cœurs vers la charité de Dieu et vers la persévérance de Christ.
\TextTitle{[Travailler jusqu'au retour du Seigneur]}
\VS{6}Nous vous recommandons aussi, mes frères, au Nom de notre Seigneur Jésus-Christ, de vous éloigner\FTNT{La séparation d’avec la mauvaise compagnie. Voir Ro. 16:17-18 ; 1 Co. 5:9-13 , 15:33 ; 2 Co. 6:14-18 ; Tit. 3:10-11 ; 2 Jn. 9-11.} de tout homme qui se dit frère, et qui vit d’une manière déréglée, et non selon les enseignements qu'il a reçus de nous.
\VS{7}Car vous savez vous-mêmes comment il faut nous imiter, car nous ne nous sommes pas conduits d’une manière déréglée parmi vous,
\VS{8}et nous n'avons mangé gratuitement le pain de personne. Mais dans le labeur et dans la peine, nous avons travaillé nuit et jour, pour n’être à la charge\FTNT{Les véritables ouvriers de Dieu ne s’attendent pas aux hommes pour avoir leur salaire. Ils mettent leur confiance en Dieu qui est leur rémunérateur. Voir Ac. 20:33-35.} d’aucun de vous.
\VS{9}Ce n’est pas que nous n'en ayons pas le droit, mais nous avons voulu vous donner en nous-mêmes un modèle à imiter.
\VS{10}Car lorsque nous étions chez vous, nous vous déclarions expressément que si quelqu'un ne veut pas travailler, qu'il ne mange pas non plus.
\VS{11}Car nous apprenons qu'il y en a quelques-uns parmi vous qui se conduisent d'une manière déréglée, qui ne travaillent pas, mais qui s’occupent de futilités.
\VS{12}C’est pourquoi nous recommandons donc à ces gens-là et nous les exhortons par notre Seigneur Jésus-Christ, à manger leur propre pain en travaillant paisiblement.
\VS{13}Mais pour vous, mes frères, ne vous lassez pas de faire le bien.
\VS{14}Et si quelqu'un n'obéit point à ce que nous vous disons par cette lettre, faites-le connaître ; et n’ayez point de relation avec lui, afin qu’il éprouve de la honte.
\VS{15}Toutefois, ne le regardez pas comme un ennemi, mais avertissez-le comme un frère.
\TextTitle{[Conclusion]}
\VS{16}Que le Seigneur de paix vous donne toujours la paix en tout temps ! Que le Seigneur soit avec vous tous.
\VS{17}Je vous salue, moi Paul, de ma propre main. C’est là ma signature dans toutes mes lettres, c'est ainsi que j'écris.
\VS{18}Que la grâce de notre Seigneur Jésus-Christ soit avec vous tous. Amen !
\PPE{}
\end{multicols}
