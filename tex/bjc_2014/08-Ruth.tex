\ShortTitle{Ruth}\BookTitle{Ruth}\BFont
\noindent\hrulefill
{\footnotesize
\textit{
\bigskip
{\centering{}
\\(Routh)
\\Signifie : Amitié, une amie
\\Thème : Les origines de la famille messianique
\\Auteur : Inconnu
\\Date de rédaction : A l’époque des juges 1220-1050 av. J.-C.\\}
}
%\bigskip
\textit{
\\Au temps des juges, tout le pays fut frappé par une famine qui poussa Elimélec, sa femme Naomi, et ses deux fils à s’installer dans le pays de Moab. Ce pays tire son nom de son fondateur Moab, né de l’inceste entre Lot et sa fille ainée.
\bigskip
\\Ils y rencontrèrent Ruth qui devint ensuite la belle fille d’Elimélec. Après la mort de son époux, cette moabite démontra  son attachement non seulement à cette famille mais également au Dieu de cette famille qui devint aussi le sien.
\bigskip
\\Au prix de sa détermination, son obéissance et son humilité, la destinée de cette femme fut complètement bouleversée. Image du rachat des nations, elle entra dans la lignée de Jésus-Christ homme.\bigskip
}
}
\par\nobreak\noindent\hrulefill
\begin{multicols}{2}
\TextTitle{[Famine en Juda]}
\Chap{1}
\VerseOne{}Au temps où les juges gouvernaient, il y eut une famine dans le pays. Un homme de Bethléem de Juda s'en alla, avec sa femme et ses deux fils, pour séjourner sur la terre de Moab.
\TextTitle{[Séjour en Moab]}
\VS{2}Le nom de cet homme était Elimélec, celui de sa femme Naomi, et les noms de ses deux fils Machlon et Kiljon ; ils étaient Ephratiens, de Bethléem de Juda. Entrés sur la terre de Moab, ils s’y établirent.
\VS{3}Elimélec, mari de Naomi, mourut, et elle resta avec ses deux fils.
\VS{4}Ils prirent pour eux des femmes Moabites, dont l'une se nommait Orpa, et la seconde Ruth\FTNT{Ruth, la Moabite, dont l’ancêtre était issu d’une relation incestueuse (Ge. 19:36-37), est devenue l’ancêtre du Messie (Mt. 1:5-6).}, et ils demeurèrent là environ dix ans.
\VS{5}Machlon et Kiljon moururent aussi tous les deux, et cette femme resta privée de ses deux fils et de son mari.
\TextTitle{[Retour en Juda]}
\VS{6}Puis elle se leva avec ses belles-filles afin de retourner de la terre de Moab, car elle entendit, au pays de Moab, que Yahweh avait visité son peuple en lui donnant du pain.
\VS{7}Elle sortit du lieu où elle habitait, avec ses deux belles-filles, et elle marcha pour revenir sur la terre de Juda.
\VS{8}Naomi dit à ses deux belles-filles : Allez, retournez chacune dans la maison de sa mère ! Que Yahweh use de bonté envers vous, comme vous l'avez fait envers ceux qui sont morts et envers moi !
\VS{9}Que Yahweh vous donne de trouver chacune du repos dans la maison d'un mari ! Et elle les embrassa. Elles levèrent leur voix, et pleurèrent ;
\VS{10}et elles lui dirent : Non, nous retournerons avec toi vers ton peuple.
\TextTitle{[Décision loyale de Ruth]}
\VS{11}Naomi dit : Retournez, mes filles ! Pourquoi viendriez-vous avec moi ? Ai-je encore dans mon sein des fils qui puissent devenir vos maris ?
\VS{12}Retournez, mes filles, allez ! Je suis trop vieille pour me remarier. Et quand je dirais : J'ai de l'espérance, quand cette nuit même je serais avec un mari, et que j'enfanterais des fils,
\VS{13}Attendriez-vous donc qu'ils aient grandi, refuseriez-vous donc des maris ? Non, mes filles ! Je suis dans une plus grande amertume que vous, car la main de Yahweh s'est éloignée de moi.
\VS{14}Et elles levèrent leur voix, et pleurèrent encore. Orpa embrassa sa belle-mère, mais Ruth s’attacha à elle.
\VS{15}Naomi dit à Ruth : Voici, ta belle-sœur est retournée vers son peuple et vers ses dieux ; retourne, après ta belle-sœur.
\VS{16}Ruth répondit : Ne me prie pas de te laisser, de me retourner et de ne pas te suivre ! Où tu iras, j'irai, où tu demeureras je demeurerai ; ton peuple sera mon peuple, et ton Dieu sera mon Dieu ;
\VS{17}Où tu mourras je mourrai, et j'y serai enterrée. Que Yahweh me traite dans toute sa rigueur, si autre chose que la mort vient à me séparer de toi !
\VS{18}Naomi, la voyant déterminée à aller avec elle, arrêta de lui parler.
\TextTitle{[Arrivée à Bethléhem]}
\VS{19}Elles marchèrent toutes deux jusqu'à ce qu'elles entrent à Bethléem. Lorsqu'elles entrèrent dans Bethléem, toute la ville fut agitée à cause d'elles, et les femmes disaient : Est-ce là Naomi ?
\VS{20}Elle leur dit : Ne m'appelez pas Naomi ; appelez-moi Mara, car le Tout-Puissant m'a remplie de beaucoup d'amertume.
\VS{21}J’étais dans l'abondance à mon départ, et Yahweh me ramène à vide. Pourquoi m'appelleriez-vous Naomi, après que Yahweh s'est prononcé contre moi, et que le Tout-Puissant m'a affligée ?
\VS{22}Ainsi revinrent de la terre de Moab Naomi et sa belle-fille, Ruth, la Moabite. Elles entrèrent dans Bethléem au commencement de la moisson des orges.
\TextTitle{[Boaz félicite Ruth des soins désintéressés dont elle entoure Naomi]}
\Chap{2}
\VerseOne{}Naomi avait un parent de son mari. C'était un homme puissant et riche, de la famille d'Elimélec, et qui s’appelait Boaz.
\VS{2}Ruth la Moabite dit à Naomi : Je te prie laisse-moi aller glaner des épis dans le champ de celui aux yeux duquel je trouverai grâce. Elle lui dit : Va, ma fille.
\VS{3}Elle s'en alla et entra dans un champ, pour glaner après les moissonneurs. Et elle arriva par hasard sur une parcelle de champ qui appartenait à Boaz, qui était de la famille d'Elimélec.
\VS{4}Or voici, Boaz vint de Bethléem, et il dit aux moissonneurs : Que Yahweh soit avec vous ! Ils lui dirent : Que Yahweh te bénisse !
\VS{5}Et Boaz dit à son serviteur qui était établi sur les moissonneurs : A qui est cette jeune fille ?
\VS{6}Le serviteur qui était établi sur les moissonneurs répondit et dit : C'est une jeune femme Moabite, qui est revenue avec Naomi de la terre de Moab.
\VS{7}Elle nous a dit : Permettez-moi de glaner et de recueillir des épis entre les gerbes, après les moissonneurs. Depuis ce matin qu'elle est venue, elle est restée jusqu'à présent, et s'est à peine assise dans la maison.
\VS{8}Boaz dit à Ruth : Ecoute, ma fille, ne va pas glaner dans un autre champ ; ne pars pas au loin, et reste avec mes servantes.
\VS{9}Regarde où l'on moissonne dans le champ, et va après elles. J'ai défendu à mes serviteurs de te toucher. Et si tu as soif, va prendre des vases, et bois de ce que les serviteurs auront puisé.
\VS{10}Alors elle tomba sur sa face, et se prosterna contre terre, et elle lui dit : Comment ai-je trouvé grâce à tes yeux, pour que tu prêtes attention à moi, moi qui suis une étrangère ?
\VS{11}Boaz lui répondit et dit : On m'a raconté tout ce que tu as fait pour ta belle-mère depuis que ton mari est mort, comment tu as laissé ton père, ta mère, et le pays de ta naissance, pour aller vers un peuple que tu ne connaissais pas auparavant.
\VS{12}Que Yahweh te récompense pour ton œuvre, et que ton salaire soit entier de la part de Yahweh le Dieu d'Israël, sous les ailes duquel tu es venue te réfugier !
\VS{13}Et elle dit : Mon seigneur, que je trouve grâce à tes yeux ! Car tu m'as consolée, et tu as parlé au cœur de ta servante. Et pourtant je ne suis pas, moi, comme l'une de tes servantes.
\VS{14}Au moment du repas, Boaz dit à Ruth : Approche-toi ici, mange du pain, et trempe ton morceau dans le vinaigre. Elle s'assit à côté des moissonneurs. On lui donna du grain rôti ; elle mangea et se rassasia, et elle garda le reste.
\VS{15}Puis elle se leva pour glaner. Boaz ordonna à ses serviteurs : Qu'elle glane même entre les gerbes, et ne lui faites pas honte.
\VS{16}Et vous retirerez même pour elle quelques poignées de gerbes, que vous lui laisserez glaner, sans la réprimander.
\VS{17}Elle glana donc dans le champ jusqu'au soir, et elle battit ce qu'elle avait glané. Il y eut environ un épha d'orge.
\VS{18}Elle l'emporta, entra dans la ville, et sa belle-mère vit ce qu'elle avait glané. Elle sortit aussi les restes de son repas, et les lui donna.
\VS{19}Sa belle-mère lui dit : Où as-tu glané aujourd'hui, et où as-tu travaillé ? Béni soit celui qui t'a reconnue ! Et Ruth raconta à sa belle-mère chez qui elle avait travaillé : L'homme chez qui j'ai travaillé aujourd'hui s’appelle Boaz.
\VS{20}Naomi dit à sa belle-fille : Qu'il soit béni de Yahweh, puisqu'il a la même bonté pour les vivants, comme il en eut pour ceux qui sont morts ! Cet homme est un proche parent, lui dit encore Naomi, il est un de ceux qui ont sur nous le droit de rachat\FTNT{Le droit de rachat : Le rédempteur est celui qui rachète une personne moyennant le paiement d'une rançon. Sous la première alliance, le rachat se faisait soit par un frère, soit par un proche parent, pour la libération de celui qui s’était fait esclave ou qui avait aliéné sa propriété ou son bien (Lé. 25:25 et 48). Sous la nouvelle alliance, Jésus-Christ est désormais notre rédempteur. Romains 3:23-24 nous dit : ~ Car tous ont péché, et sont entièrement privés de la gloire de Dieu. Et ils sont gratuitement justifiés par sa grâce, par la rédemption qui est en Jésus-Christ ~. Christ nous a rachetés de la malédiction de la loi en se donnant lui-même pour nous afin de nous délivrer de toute iniquité (Ga. 3:13 ; Ti. 2:14). Dieu s’est fait homme (Hé. 2:14-17) afin de mieux nous libérer de l'esclavage du diable par sa mort à la croix de Golgotha (Es. 60:16).}.
\VS{21}Ruth la Moabite dit : Il m'a même dit : Reste avec mes serviteurs jusqu'à ce qu'ils aient achevé toute ma moisson.
\VS{22}Et Naomi dit à Ruth, sa belle-fille : Ma fille, il est bon que tu sortes avec ses servantes, et qu'on ne te rencontre pas dans un autre champ.
\VS{23}Elle resta donc avec les servantes de Boaz, pour glaner, jusqu'à la fin de la moisson des orges et la moisson des froments. Et elle demeurait avec sa belle-mère.
\TextTitle{[Ruth dans l'obéissance de la foi]}
\Chap{3}
\VerseOne{}Naomi, sa belle-mère, lui dit : Ma fille, je voudrais chercher ton repos, afin que tu sois heureuse.
\VS{2}Maintenant Boaz, avec les servantes duquel tu as été, n'est-il pas de notre parenté ? Voici, il doit vanner cette nuit les orges qui ont été foulées dans l'aire.
\VS{3}Lave-toi et oins-toi, puis mets tes habits, et descends dans l'aire. Ne te fais pas connaître à lui, jusqu'à ce qu'il ait achevé de manger et de boire.
\VS{4}Quand il se couchera, découvre le lieu où il se couche. Ensuite, entre, découvre ses pieds, et couche-toi. Il te dira ce que tu as à faire.
\VS{5}Elle lui répondit : Je ferai tout ce que tu as dit.
\VS{6}Elle descendit à l'aire, et fit tout ce que sa belle-mère lui avait ordonné.
\VS{7}Boaz mangea et but, et son cœur était joyeux. Il vint se coucher à l'extrémité d'un tas de gerbes. Ruth vint secrètement, découvrit ses pieds, et se coucha.
\VS{8}Au milieu de la nuit, cet homme eut peur ; il se retourna et retira ses pieds, car voici, une femme était couchée à ses pieds.
\VS{9}Il dit : Qui es-tu ? Elle répondit : Je suis Ruth, ta servante ; étends le pan de ta robe sur ta servante, car tu as droit de rachat.
\VS{10}Et il dit : Ma fille, que Yahweh te bénisse ! Ce dernier trait de bonté me réjouit plus que le premier, car tu n'es pas allée après des jeunes gens, pauvres ou riches.
\VS{11}Maintenant, ma fille, ne crains pas ; je te ferai tout ce que tu me diras ; car toute la porte de mon peuple sait que tu es une femme vertueuse.
\VS{12}Il est bien vrai que j'ai droit de rachat, mais il existe un autre plus proche que moi, qui a le droit de rachat.
\VS{13}Passe ici la nuit, et demain, si cet homme veut user envers toi du droit de rachat, à la bonne heure, qu'il te rachète ; mais s'il ne lui plaît pas de te racheter, moi je te rachèterai, Yahweh est vivant ! Couche-toi jusqu'au matin.
\VS{14}Elle se coucha à ses pieds jusqu'au matin, et elle se leva avant qu'on puisse se reconnaître l'un l'autre. Boaz dit : Qu'on ne sache pas qu'une femme est entrée dans l'aire.
\VS{15}Et il dit : Donne-moi le manteau qui est sur toi, et tiens-le. Elle le tint, et il mesura six mesures d'orge, qu'il posa sur elle. Puis il entra dans la ville.
\VS{16}Ruth revint auprès de sa belle-mère, et Naomi dit : Est-ce toi ma fille ? Ruth lui raconta tout ce que cet homme avait fait pour elle.
\VS{17}Elle dit : Il m'a donné ces six mesures d'orge, en disant : Tu n'iras pas à vide vers ta belle-mère.
\VS{18}Et Naomi dit : Ma fille, assieds-toi ici jusqu'à ce que tu saches ce que l'affaire deviendra, car cet homme ne se donnera pas de repos, qu'il n'ait achevé cette affaire aujourd'hui.
\TextTitle{[Ruth comblée par le mariage]}
\Chap{4}
\VerseOne{}Boaz monta à la porte, et s'y assit. Or voici, celui qui avait le droit de rachat, et dont Boaz avait parlé, passa. Boaz lui dit : Ah ! Détourne-toi, reste ici, toi un tel. Et il se détourna, et s'assit.
\VS{2}Boaz prit dix hommes d'entre les anciens de la ville, et leur dit : Asseyez-vous ici. Et ils s'assirent.
\VS{3}Puis il dit à celui qui avait le droit de rachat : Naomi qui est revenue de la terre de Moab, a vendu la parcelle du champ qui appartenait à notre frère Elimélec.
\VS{4}J'ai parlé à tes oreilles afin de te le faire savoir et te le dire : Acquiers-la en la présence de ceux qui sont assis ici et en présence des anciens de mon peuple. Si tu veux racheter par droit de rachat, rachète-la ; mais si tu ne veux pas la racheter, déclare-le-moi, afin que je le sache. Car il n'y a pas d'autre que toi qui ait le droit de rachat, et je l'ai après toi. Et il dit : je rachèterai.
\VS{5}Boaz dit : Le jour où tu acquerras le champ de la main de Naomi, tu l'acquerras aussi de Ruth la Moabite, femme du défunt, pour maintenir le nom du défunt dans son héritage.
\VS{6}Et celui qui avait le droit de rachat dit : Je ne puis pas racheter pour mon compte, de peur de détruire mon héritage ; prends pour toi le droit de rachat, car je ne puis pas le racheter.
\VS{7}Autrefois en Israël, pour confirmer une affaire quelconque relative à un rachat ou à un échange, l'homme ôtait son soulier et le donnait à son parent : C'était là, en Israël, un témoignage qu'on cédait son droit.
\VS{8}Celui qui avait le droit de rachat dit à Boaz : Acquiers-le pour toi ! Et il ôta son soulier.
\VS{9}Alors Boaz dit aux anciens et à tout le peuple : Vous êtes aujourd'hui témoins que j'ai acquis de la main de Naomi tout ce qui appartenait à Elimélec, à Kiljon et à Machlon.
\VS{10}Et que je me suis également acquis pour femme Ruth la Moabite, femme de Machlon, pour maintenir le nom du défunt dans son héritage, et afin que le nom du défunt ne soit pas retranché d'entre ses frères et de la porte de sa ville. Vous en êtes témoins aujourd'hui !
\VS{11}Tout le peuple qui était à la porte et les anciens dirent : Nous en sommes témoins ! Que Yahweh rende la femme qui entre dans ta maison semblable à Rachel et à Léa, qui ont bâti toutes deux, la maison d'Israël ! Montre ta puissance dans Ephrata et proclame ton nom dans Bethléem !
\VS{12}Puisse la postérité que Yahweh te donnera de cette jeune femme, rendre ta maison semblable à la maison de Pérets, que Tamar enfanta à Juda !
\VS{13}Boaz prit Ruth, qui devint sa femme, et il alla vers elle. Yahweh lui fit la grâce de concevoir, et elle enfanta un fils.
\VS{14}Les femmes dirent à Naomi : Béni soit Yahweh qui ne t'a pas laissé manquer aujourd'hui d'un homme, ayant droit de rachat, et dont le nom sera proclamé en Israël !
\VS{15}Cet enfant restaurera ton âme, et sera le soutien de ta vieillesse ; car ta belle-fille, qui t'aime, l'a enfanté, et elle vaut mieux que sept fils.
\VS{16}Naomi prit l'enfant et le posa sur son sein, et elle fut sa nourrice.
\TextTitle{[Le fils de Ruth sera le grand-père de David]}
\VS{17}Les voisines lui donnèrent un nom, en disant : Un fils est né à Naomi ! Et elles l'appelèrent du nom de Obed. Ce fut le père d'Isaï, père de David.
\VS{18}Voici la généalogie de Pérets. Pérets engendra Hetsron ;
\VS{19}Hetsron engendra Ram ; Ram engendra Amminadab ;
\VS{20}Amminadab engendra Nachschon ; Nachschon engendra Salmon ;
\VS{21}Salmon engendra Boaz ; Boaz engendra Obed ;
\VS{22}Obed engendra Isaï, et Isaï engendra David.
\PPE{}
\end{multicols}
