\begin{multicols}{2}

\DicoEntry{AARON}\textit{, de l'hébreu « aharown » : « haut placé, éclairé »}\newline
Issu de la tribu de Lévi et frère aîné de Moïse, il fut le premier souverain sacrificateur en Israël. Voir Ex. 6 : 16-20 ; Ex. 4 : 14 et Ex. 28.

\DicoEntry{ABDIAS}\textit{, de l'hébreu « obadyah » : « adorateur, ou serviteur de Yahweh »}\newline
Prophète dont le livre éponyme figure dans le Tanakh, il annonça le jugement de Yahweh sur Edom à cause de son inimitié avec Israël.

\DicoEntry{ABEL}\textit{, de l'hébreu « hebel » : « souffle, vapeur »}\newline
Deuxième fils d'Adam et Eve et première victime d’homicide de l’histoire, il fut assassiné par son frère Caïn et déclaré juste par Dieu. Voir Ge. 4 : 2 + 8 et Mt. 23 : 35.

\DicoEntry{ABIRAM}\textit{, de l’hébreu « Abiram » : « le père est exalté »}\newline
Issu de la tribu de Ruben, fils d'Eliab et frère de Dathan, il conspira avec Koré contre Moïse et Aaron (No. 16 : 1-35).

\DicoEntry{ABLUTION}\textit{}\newline
Lavage de purification prescrit par la loi mosaïque et effectué avec de l'eau. Voir Ex. 29 : 4 et Hé. 9 : 10.

\DicoEntry{ABOMINATION}\textit{, du grec « bdelugma » : « chose abominable, détestable ; horreur »}\newline
Pratique violant la loi de Yahweh et manifestant l’infidélité à Dieu comme l’idolâtrie sous toutes ses formes, la magie ou l’homosexualité. Voir Lé. 18 :6-29 ; De. 29 :17-18 et Ap. 21 : 27

\DicoEntry{ABRAM}\textit{, de l’hébreu « Abriram » : «le père est exalté »}\newline
(voir Abraham)

\DicoEntry{ABRAHAM}\textit{, de l'hébreu « Abraham » : « père d'une multitude »}\newline
Hébreu, fils de Térach, et originaire d'Ur en Chaldée (Mésopotamie), Dieu lui demanda de quitter sa terre et sa famille pour Canaan lui promettant que sa postérité hériterait de cette terre. Premier patriarche, Yahweh changea son nom d’Abram à Abraham et marqua leur alliance au travers de la circoncision. De sa servante Agar, lui naquit un premier fils, Ismaël, ancêtre du peuple arabe. Cependant, celui qui hérita des promesses fut Isaac qu’Abraham eut à 99 ans de sa femme Sara. Voir Ge. 12 : 1-7 ; Ge. 17 :4-13 ; Ge. 16 et Ge. 21 : 1-8.

\DicoEntry{ACACIA}\textit{}\newline
Arbre épineux poussant en abondance dans la péninsule du Sinaï et dans la vallée du Jourdain, il est aussi appelé bois de Sittim. Son bois léger et solide est imputrescible et peut donc se conserver très longtemps. Il fut l’un des matériaux utilisés pour la fabrication des objets du culte lévitique dont l’arche. Voir Ex. 25-27.

\DicoEntry{ACHAB}\textit{, de l'hébreu « 'Ach' ab » : « un frère du père »}\newline
Fils d'Omri, il fut le septième roi d'Israël. Marié à Jézabel, fille du roi des Sidoniens, Achab et sa femme commirent de grandes abominations et s’opposèrent au prophète Elie. Il mourut lors d'une bataille contre les syriens après avoir régné 22 ans en Israël. Voir 1 R. 16 : 29-31 ; 1 R. 18 : 1-40 ; 1 R. 22 : 29-40.

\DicoEntry{ADAM}\textit{, de l'hébreu « 'adam » : « être humain »}\newline
Créé à l’image de Dieu et formé à partir de la terre, il fut le premier homme et vécut la première partie de sa vie dans le jardin d’Eden avec sa femme Eve qui fut tiré de lui. Après avoir désobéi à Dieu en goûtant le fruit de l'arbre de la connaissance du bien et du mal, ils firent entrer le péché dans l’homme et furent chassé du jardin. Adam fut le père de Caïn, Abel et Seth et mourut à 930 ans. Voir Ge. 2 : 7-8 ; Ge. 3 ; Ro. 5 : 12 ; Ge. 4 : 1-2 + 25-26 et Ge. 5 : 5.

\DicoEntry{AGAR}\textit{, de l'hébreu « hagar » : « fuite »}\newline
Servante égyptienne de Sara que cette dernière donna à Abraham comme concubine. Elle enfanta Ismaël, fils premier-né d’Abraham. Après la naissance d’Isaac, Abraham la chassa avec son fils. Voir Ge. 16 et Ge. 21 : 1-18.

\DicoEntry{AGGEE}\textit{, de l'hébreu « chaggay » : « en fête », « né un jour de fête »}\newline
Prophète dont le livre éponyme figure dans le Tanak, il exerça son ministère au retour de l'exil babylonien et encouragea le peuple et Zorobabel à reconstruire le temple.

\DicoEntry{AGNEAU}\textit{}\newline
Animal sacrifié et consommé lors de la Paque des juifs. Il préfigurait Christ, l'Agneau de Dieu qui ôte le péché du monde. Voir Ex. 12 : 1-28 et Jn. 12 : 9.

\DicoEntry{AI}\textit{}\newline
Ville probablement située au sud-est de Béthel à proximité de laquelle Abraham dressa sa tente à deux reprises. Il s’agit également de la deuxième ville que Dieu livra entre les mains de Josué après la prise de Jéricho. Voir Ge. 12 : 8 ; Ge. 13 : 3 et Jos. 8.

\DicoEntry{ALLELUIA}\textit{, de l’hébreu « haleluyah » : « Louez Yahweh »}\newline
Retrouvé à maintes reprises dans les psaumes sous la forme « Louez Yahweh », cette exclamation encourage à célébrer Dieu et à se réjouir en lui. Voir Ap. 19 : 1-6.

\DicoEntry{ALLIANCE}\textit{}\newline
Accord unissant deux ou plusieurs personnes entre elles, elle est valable uniquement si les parties concernées en respectent les termes. Dieu a conclu plusieurs alliances avec les hommes. Parmi elles, on retrouve l’alliance avec Noé (Ge. 9 : 8-17) ; Abraham (Ge. 17) ou encore David (2 S. 7 : 12-16). On distingue communément deux alliances majeures : l’ancienne alliance - conclue avec Israël au travers de Moïse (Ex. 19-34) - et la nouvelle alliance inaugurée par Jésus-Christ (Hé. 9-13).

\DicoEntry{OMEGA}\textit{}\newline
Première et dernière lettre de l'alphabet grec, la combinaison de ces deux lettres mentionnées ensemble se rapportent à l’idée que Dieu est « le premier et le dernier ». Jésus fut présenté plusieurs fois comme étant « l’alpha et l’oméga » soulevant ainsi son caractère éternel. Voir Ap. 1 : 8 ; Ap. 21 : 6 et Ap. 22 : 13.

\DicoEntry{ALYA}\textit{, de l’hébreu « alyoth » : « ascension ou élévation spirituelle »}\newline
Retour massif des juifs dispersés en terre d'Israël. Même si ce terme n’est pas employé dans la bible, ce phénomène - ayant connu plusieurs vagues depuis le XIXème siècle et se poursuivant aujourd’hui - a été prophétisé à plusieurs reprises. Voir Jé.

\DicoEntry{AME}\textit{}\newline
Partie immatérielle et immortelle de l'homme localisée dans le sang, elle est le siège des émotions et renferme la vie. Avec l'esprit et le corps, l'âme constitue l'être humain. Voir Lé. 17 : 11 ; Ac. 20 : 10 et 1 Th. 5 : 23.

\DicoEntry{AMEN}\textit{, de l’hébreu « amen » : « assuré, établi » ou « ainsi soit-il ! »}\newline
Se rapportant exclusivement à ce qui est sûr, avéré et certain, ce terme est souvent utilisé comme injection. Christ est aussi appelé l'Amen, faisant référence à la vérité qu’Il est. Voir 1 Ch. 16 : 36 ; Jé. 28 : 6 ; 2 Co. 1 : 20 et Ap. 3 : 14.

\DicoEntry{AMOUR}\textit{}\newline
Il existe plusieurs traductions et définitions du mot « amour » en grec variant selon le contexte.

« Agape » : « amour, charité, affection »
Désigne aussi bien l’amour de Dieu que l’amour fraternel que les disciples sont appelés à manifester les uns envers les autres. Cet amour, qui suggère la bonne volonté et la bienveillance, a pour source Dieu lui-même sans qui il est impossible de le produire. Voir Jn. 15 :13 ; Jn. 17 : 26 ; Ro. 5 : 5 ; 1 Co. 8 : 1 et 1 Co. 13 : 3 ; 1 Jn. 4 : 8.

« Phileo » : « aimer, montrer des signes d’amour »
Voir Jn. 21 : 17 ; 1 Co. 16 : 22.

« Philadelphia » : « amour fraternel »
Se rapporte exclusivement à l’amour que les disciples de Christ sont appelés à se manifester en tant que frères et sœurs. Voir Ro. 12 : 10 ; 1 Th. 4 : 9 et Hé. 13 : 1.

\DicoEntry{AMOS}\textit{, de l'hébreu « 'Amowc » : « qui porte le fardeau »}\newline
Prophète dont le livre éponyme figure dans le Tanakh, il annonça le jugement du Seigneur sur plusieurs nations païennes, le royaume de Juda et le royaume d'Israël.

\DicoEntry{AMMONITES}\textit{}\newline
Peuple issu de Ben-Ammi - né de l’inceste entre Lot et sa fille cadette - ils figurèrent parmi les ennemis d'Israël. Voir Ge. 19 : 30-38 et Ez. 25 : 1-7.

\DicoEntry{ANAKIM}\textit{, de l'hébreu « 'Anaqiy » : « long cou »}\newline
Descendants d'Anak, race de géants habitant Canaan avant sa conquête par le peuple d’Israël. Ils furent vaincus par Josué et Caleb qui hérita d’un de leur territoire. Voir No. 13 : 28-33 ; De. 9 : 1-3 ; Jos. 11 : 21-22 et Jos. 14 : 6-15.

\DicoEntry{ANATHEME}\textit{, du grec « anathema » : « tout ce qui est livré au malheur »}\newline
Terme désignant une personne ou une chose maudite, vouée à la destruction. Voir 1 Co. 12 : 3 et Ga. 1 : 8.

\DicoEntry{ANCIENS}\textit{, du grec « presbuteros » : « homme plus âgé »}\newline
Chez les juifs, il s’agissait des chefs de famille ou de clan qui représentaient le peuple dans les affaires religieuses et civiles. Voir Ex. 3 : 16 ; Lé. 4 : 15 et De. 31 : 28.
Dans la nouvelle alliance, les anciens (ou évêques) sont des personnes au témoignage irréprochable dont la mission est de veiller avec dévotion et humilité pour les âmes du Seigneur au sein de l’assemblée locale. Voir Ac. 14 : 23 ; 1 Ti. 5 : 17 ; 1 Pi. 5 : 1-5 et Tite 1 : 5-9.

\DicoEntry{ANGE}\textit{, de l'hébreu « mal'ak » : « messager, envoyé »}\newline
Etre spirituel au service de Dieu pouvant prendre une forme humaine. Les anges sont au service de Yawheh pour des missions spécifiques au ciel (service dans le temple de Dieu) ou sur la terre (service auprès des enfants de Dieu). Ils peuvent avoir une fonction de messager, protecteur ou combattant. Voir Lu. 1 : 26-38 ; Da. 10 : 10-13 et Ap. 12 : 7.

Ils sont organisés en catégories :, et les séraphins (Es. 6 : 1-3).

\DicoEntry{ANTECHRIST}\textit{}\newline
Aussi appelé « homme impie » et « fils de la perdition », personnage dont l'apparition se fera avant le retour glorieux du Seigneur. Il dominera le monde avant d'être vaincu par Christ. Voir 2 Th. 2 : 1-4 et Ap. 19 : 19-21.

\DicoEntry{APIS}\textit{}\newline
Divinité égyptienne symbolisant la force et la fertilité. Il est représenté sous la forme d'un veau d'or ou d'un homme à tête de taureau avec des cornes entourant un disque solaire. Les hébreux se prostituèrent plusieurs fois à son culte. Voir Ex. 32 : 1-6 et 1 R. 12 : 28-30.

\DicoEntry{APOCALYPSE}\textit{, du grec « Apokalupsis » : « mettre à nu, révélation d'une vérité, action de révéler »}\newline
Dernier livre de la nouvelle alliance écrit par l’apôtre Jean, ce récit comporte une révélation de la gloire de Jésus-Christ et raconte les derniers évènements de l’histoire de l’humanité jusqu'à l'avènement de la nouvelle Jérusalem.

\DicoEntry{APOSTASIE}\textit{, du grec « apostasia » : « action de s'éloigner de, désertion, défection »}\newline
Abandon de la foi en Jésus-Christ et de la saine doctrine se manifestant sous deux formes principales. Certaines personnes abandonnent ouvertement la foi, la communion avec Dieu et l’assemblée des saints. D’autres continuent de fréquenter les assemblées chrétiennes mais ont laissé la saine doctrine pour s’attacher à des doctrines séductrices. Voir Mt. 24 : 11-12 ; 2 Th. 2 : 3 ; 1 Ti. 4 : 1-3 ; 2 Ti. 3 : 1-8 ; 4 : 3-4 ; 2 Pi. 2 : 1-3 ; 1 Jn. 4 : 1 ; Jud. 17-19.

\DicoEntry{APOTRE}\textit{, du grec « apostolos » : « envoyé, messager, ambassadeur »}\newline
Ministre de la Parole appelé et qualifié par Dieu, il est chargé d’implanter des assemblées, d’en poser les fondements doctrinaux et de veiller à leur bon fonctionnement. Jésus choisit 12 apôtres qui le suivirent et qu’il forma durant son ministère terrestre. Paul reçut également le mandat du Seigneur pour exercer ce service au temps de l’église primitive. Voir Mc. 3 : 14 ; Ro. 1 : 1 et Ep. 4 : 11.

\DicoEntry{ARC-EN-CIEL}\textit{}\newline
Signe de l’alliance que Dieu conclut avec Noé et les générations qui le suivraient suite au déluge. Cette alliance stipulait que Yahweh ne détruirait plus les hommes par les eaux. Voir Ge. 9 : 12-17.

\DicoEntry{ARCHANGE}\textit{, du grec « archaggelos » : « chef des anges »}\newline
Catégorie d’ange ayant un rang et une dignité plus élevés que les autres. Voir 1 Th. 4 : 16 et Jud. 1 : 9.

\DicoEntry{ARCHE }\textit{de Noé}\newline
Embarcation construite par Noé selon les instructions de Dieu. Elle fut créée pour préserver les animaux ainsi que Noé et sa famille du jugement de Dieu qui allait s’abattre sur toute la terre au travers du déluge. Voir Ge. 6 : 8-16 ; Mat. 24 : 37-39 et Lu. 17 : 26-27.

\DicoEntry{ARCHE }\textit{du Témoignage ou de l'Alliance}\newline
Coffre rectangulaire de bois d’acacia recouvert d’or pur, elle contenait les tables de l'alliance, la verge d'Aaron et une urne contenant un échantillon de manne. Construite selon le modèle que Moïse avait reçu au mont Sinaï, elle avait sur son couvercle deux chérubins entre lesquels Dieu se tenait pour rencontrer son serviteur. Ayant accompagné Israël dans sa marche dans le désert, l’arche fut ensuite placée dans le lieu très saint du tabernacle, puis du temple. Elle ne fut plus mentionnée dans les écritures après la destruction du temple. Voir Ex. 25 : 10-22 ; Hé. 9 : 4 ; 1 R. 8 : 6 et 2 R. 25 : 8-9.

\DicoEntry{ASAPH}\textit{, de l'hébreu « 'Acaph » : « celui qui rassemble, collecteur »}\newline
Lévite et chef des chantres sous David, il participa au transfert de l’arche à Jérusalem. On lui attribue les psaumes 50 et de 73 à 83. Voir 1 Ch. 15 : 16-19 et 1 Ch. 16 : 4-7.

\DicoEntry{ASER}\textit{, de l'hébreu « asher » : « heureux »}\newline
Fils de Jacob et de Zilpa, servante de Léa, il est le père de la tribu d’Aser. Voir Ge. 30 : 13.

\DicoEntry{ASHERA }\textit{ou ASTARTE, de l’hébreu « asherah » : « poteau, arbre »}\newline
Pieu sacré qui symbolise la déesse païenne de la fécondité dans la tradition cananéenne. Associée à Baal et identifiée plus tard à Astarté, cette divinité fut l’objet de l’idolâtrie des babyloniens et parfois même du peuple d’Israël. Voir 1 R. 14 : 23 ; 1 R. 18 : 19 et 1 S. 7 : 4.

\DicoEntry{AUTEL}\textit{}\newline
Table généralement façonnée avec des monticules de pierre ou en terre et élevée spécialement pour offrir des holocaustes et des sacrifices en l'honneur de Dieu. Dans le lieu saint se trouvait un autel pour bruler quotidiennement des parfums. Voir Ge. 12 : 7 ; Ge. 35 : 7 ; Ex. 20 : 24-26 et Ex. 30 : 1-8.

\DicoEntry{BAAL}\textit{, de l'hébreu « ba al » : « maître, possesseur, seigneur »}\newline
Dieu primaire des phéniciens et des cananéens auquel les israélites s’attachèrent à plusieurs reprises pour l’adorer. Voir No. 25 : 3 ; Jg. 2 : 11 et 1 R. 18 : 21.

\DicoEntry{BABEL }\textit{ou BABYLONE, de l'hébreu « babel » : « confusion »}\newline
Ville de Mésopotamie, les hommes y entreprirent sous le règne de Nimrod un projet s’opposant à la volonté de Dieu. Cependant, Yahweh confondit leur language et les dispersa sur toute la terre (Ge 10 : 8-10 et Ge 11 : 1-9).

\DicoEntry{BALAAM}\textit{, de l'hébreu « bil am » : « sans peuple », « dévorant »}\newline
Prophète de Yahweh ayant vécu pendant la marche d’Israël dans le désert, il fut séduit par Balak, roi de Moab, qui lui proposa de maudire Israël contre de généreux présents. Son témoignage a été utilisé plusieurs fois pour avertir les enfants de Dieu des scandales dont ils pourraient être la cause en suivant la capudité de Balaam qui fut tué. Voir No. 22-24 ; No 31 : 8 ; Jud. 1 : 11 et Ap. 2 : 14.

\DicoEntry{BALAK}\textit{, de l'hébreu de « balaq » : « gaspilleur, dévastateur »}\newline
Roi de Moab, il essaya de convaincre Balaam de maudire Israël qu’il redoutait. Voir No. 22-24.

\DicoEntry{BANNIERE}\textit{}\newline
Drapeau, étandard élevé en signe d'appartenance à ce qu'il représente. Moise bâtit un autel du nom de Yahweh-Nissi : « Yahweh ma bannière ». Voir Ex. 17 : 15.

\DicoEntry{BAPTEME}\textit{, du grec « baptizo » : « plonger, immerger »}\newline
Suivant la conversion, acte par lequel une personne est immergée dans l'eau - symbolisant la mort et la résurrection en Jésus-Christ. Il s’agit selon Pierre de l’« engagement d'une bonne conscience envers Dieu ». Voir Ac. 2 : 38 ; Ac. 16 : 30-33 ; Col. 2 : 12-13 et 1 Pi. 3 : 21.

\DicoEntry{BARAK}\textit{, de l'hébreu « baraq » : « éclair, foudre »}\newline
Fils d'Abinoam et issu de la tribu de Nephtali, il vécut en Israël dans le temps des juges. Encouragé et accompagné par Débora, il mit battit l'armée de Jabin, roi de Canaan. Voir Jg. 4.

\DicoEntry{BARTIMEE}\textit{, du grec « bartimaios » : « Fils de Timée »}\newline
Fils de Timée, mendiant aveugle, Jésus le guérit suite à ses cris de supplication sur la route Jéricho. Voir Mc 10 : 46-52.

\DicoEntry{BEELZEBUL }\textit{de l'hébreu « ba al zebuwb » : « seigneur des mouches »}\newline
Esprit mauvais considéré comme le prince des démons et par lequel Jésus chassait les démons, selon les dires des pharisiens. Voir Mt. 12 : 24-27 et Mc 3 : 22-26.

\DicoEntry{BELIAL }\textit{de l'hébreu « beliya al » : « indigne, sans valeur, méchant ruine, destruction »}\newline
Symbolisant l'infidélité envers Dieu, il s’agit d’un autre nom de Satan. Voir 2 Co 6 : 15.

\DicoEntry{BETE}\textit{}\newline
Dans les récits à caractère apocalyptique, les bêtes sont des animaux symbolisant les puissances politiques. Voir Dn. 7 ; Ap. 13 et 17.

\DicoEntry{BETHLEHEM}\textit{, de l’hébreu « bethleem » : « maison du pain »}\newline
Ville de Juda très présente dans les écritures, elle fut témoin de la naissance de Benjamin et du décès de Rachel, femme de Jacob. Lieu où David fut oint par Samuel, ce fut aussi la ville de naissance de Jésus-Christ où Hérode ordonna le massacre des enfants de moins de deux ans. (Ge 35 : 16-20 ; 1 S. 16 ; Mat 2 : 16 et Lu 2 : 4-7).

\DicoEntry{BIBLE}\textit{, du grec « biblia » : « bibliothèque »}\newline
Aussi appelée "Parole de Dieu", recueil de livres inspirés de Dieu et utiles pour enseigner, convaincre, corriger et instruire dans la justice. Voir 2 Ti. 3 : 16.

\DicoEntry{BLASPHEME}\textit{, du grec « blasphemia » : « parole qui blesse »}\newline
Parole ou acte outrageant ou insultant envers Dieu. Voir 2 S. 12 : 14 et Ap. 16 : 9.
Le blasphème contre le Saint Esprit, péché impardonnable, désigne le fait d'attribuer l'œuvre de Dieu au diable (Mat 12 : 22-32).

\DicoEntry{BREBIS}\textit{}\newline
Femelle du bélier, c’est l'animal pour qui le berger donne sa vie. Elle est le symbole du véritable disciple qui n’obéit qu’à la voix de son Maître et qui se laisse conduire et choyer par Jésus le bon berger. Voir Jn. 10 : 1-16.

\DicoEntry{CAIN}\textit{, de l'hébreu « Qayin » : « possession », « artisan, forgeron »}\newline
Fils d'Adam et d'Eve, il fut le premier auteur d’un homicide en tuant son frère Abel. Il engendra Lémec, premier polygame de l'histoire. Voir Ge. 4 : 1-8 et 16-19.

\DicoEntry{HEBRAIQUE}\textit{}\newline
Nisan (ou Abib) = Mars ; Iyyar (ou Ziv) = Avril ; Sivan = Mai ; Thammuz = Juin ; Ab = Juillet ; Elul = Août ; Tisri (ou Ethanim) = Septembre ; Marchesvan (ou Bul) = Octobre ; Chislev (ou Kisleu) = Novembre ; Tébeth = Décembre ; Schebat = Janvier ; Adar = Février

\DicoEntry{CAMP}\textit{}\newline
Lieu de stationnement temporaire d’un groupement civil ou militaire. Voir Ex 14 : 2.

\DicoEntry{CANAAN}\textit{, de l'hébreu « kena an » : « terre basse », « marchand »}\newline
Fils de Cham. Ses descendants occupèrent la région éponyme qui correspond plus ou moins aujourd'hui aux territoires réunissant la Palestine, l'Etat d'Israël, l'Ouest de la Jordanie, le Sud du Liban et l'Ouest de la Syrie. Ce territoire correspond également à la terre promise par Dieu aux Israélites dont ils prirent possession sous la conduite de Josué. Voir Ge. 9 : 18 ; Jos. 6 à 21 et Ac. 13 : 19.

\DicoEntry{CESAR }\textit{Jules, (100 à 44 av. J.-C.)}\newline
Général romain. Son nom devint par la suite celui de certains empereurs romains. Dans les écritures, César symbolise également les autorités séculières. Voir Mt. 22 : 21.

\DicoEntry{CHAIR}\textit{}\newline
Selon le contexte, désigne le corps humain, l'être humain ou la nature humaine conduite par le péché. Voir Lu. 24 : 39 ; Lu. 3 : 6 ; Jn. 17 : 2 ; Ro. 8 : 5-9; Ga. 5 : 16-21 et Ep. 2 : 3.

\DicoEntry{CHALDEE}\textit{}\newline
Région située au sud de la Mésopotamie dont Abraham est originaire. Voir Ge. 11 : 28.

\DicoEntry{CHAM}\textit{, de l’hébreu « cham » : « chaud, bouillant »}\newline
Fils cadet de Noé. Son fils, Canaan, fut maudit par Noé. Voir Ge. 9 : 18-27.

\DicoEntry{CHARAN}\textit{, de l’hébreu « charan » : « montagnard », « route,caravane »}\newline
Région proche d'Ur en Chaldée où Abraham séjourna jusqu'à la mort de son père Térach. Voir Ge. 11 : 31 et Ge. 12 : 4.

\DicoEntry{CHEMIN }\textit{de Shabbat}\newline
Selon la loi de Moïse, distance maximum de laquelle les juifs peuvent s'éloigner de leur demeure le jour du sabbat (cf tableau mesures/distances). Voir Ac. 1 : 12.

\DicoEntry{CHERUBINS}\textit{, de l'hébreu « keruwb » : « être angélique, chérubin »	}\newline
Catégorie d’anges portant et/ou gardant la gloire de Dieu. Yahweh en avait placé à l’entrée du jardin d'Eden pour empêcher l'homme d'y accéder. Deux chérubins étaient représentés sur le propitiatoire répliquant ainsi le temple céleste où Dieu est assis sur des chérubins. Avant sa chute, Satan était lui-même un chérubin protecteur. Voir Ge 3 : 24 ; Ex 25 : 17-20 ; Es. 37 : 16 et Ez. 28 : 14.

\DicoEntry{CHRETIEN}\textit{, du grec « christanos » : « de Christ », « petit christ », « comme Christ »}\newline
Comme son étymologie le suggère, le chrétien appartient à Christ, a sa nature en lui et lui ressemble. Il est donc un disciple de Jésus-Christ qui suit son enseignement et le met en pratique. Ce terme fut employé pour la première fois à Antioche. Voir Ac. 11 : 26.

\DicoEntry{CHRIST}\textit{, du grec « christos » et de l'hébreu « mashiyach » : « oint »}\newline
Souvent accolé au nom de Jésus, ce terme suggère que ce dernier est l'oint de Dieu tant attendu. Pierre reçut la révélation de Jésus comme étant le Christ. Jésus annonça l’émergence de faux christs (= faux ouvriers de Christ) à la fin des temps. Voir Ro. 1 : 1 ; Mat 16 : 15-16 ; Hé 1 : 9 ; Mat 24 : 24 et Mc 13 : 22-23.

\DicoEntry{CIRCONCISION}\textit{}\newline
Section et ablation du prépuce. En signe d'alliance, Dieu ordonna à Abraham de circoncire tous les mâles de sa maison ; les enfants d’Israël ont perpétré cette pratique. Sous la Nouvelle Alliance, elle n’est plus requise ; ce qui importe c’est la circoncision du cœur, l’observation des commandements de Dieu. Voir Ge 17 : 9-14 ; Lu. 1 : 59 ; Ro. 2 : 25-29 et 1 Co. 7 : 19.

\DicoEntry{COEUR}\textit{}\newline
Organe permettant la circulation du sang, les écritures définissent le cœur comme un grand abîme. Siège des émotions et des pensées intimes, il peut être une bonne ou une mauvaise source. Voir Ge. 8 : 21 ; Ps. 64 : 7 ; Pr. 12 : 25 ; Jg. 19 : 6 ; 2 S. 24 : 10 et Mc. 7 : 21 .

\DicoEntry{COMMUNION}\textit{}\newline
Association de deux personnes ou plus. Le disciple de Christ est appelé à vivre deux types de communion. Il doit tout d’abord être en communion intime avec Dieu puis avec d’autres membres du corps de Christ pour vivre la communion fraternelle. Voir 1 Jn. 1 : 3 ; 2 Co. 13 : 11-13 ; Ps. 133 et Ac. 2 : 42.

\DicoEntry{CONCILE}\textit{, du latin « concilium » : « assemblée »}\newline
Assemblée d'évêques de l'église catholique (également connus sous l’appellation « pères de l’église catholique) réunis dans le but de définir les règles de la foi chrétienne. Cette pratique va à l’encontre du message de Christ puisqu’il a strictement condamné la modification du message qu’il a lui-même prêché et confié aux apôtres. Voir Mt. 5 : 18 et Ga. 1 : 8-9.

\DicoEntry{CONFESSION}\textit{}\newline
Exposition orale et sincère de la vérité, la confession peut concerner le péché dans le cadre de la repentance ou le nom du Seigneur dans le cadre de la conversion ou de la prise de position pour Christ. Voir 1 Jn. 1 : 9 ; Pr. 28 : 13 et Ro. 10 : 9

\DicoEntry{CONVERSION}\textit{, du grec « epistrepho » : « action de se retourner, de se tourner vers »}\newline
Fruit d’une sincère repentance, la conversion est la transformation opérée par la rencontre avec Christ. Il s’agit aussi du changement de nature produit par la naissance d’en haut. Voir Ac. 3 : 19 ; Mt. 18 : 3 et 2 Co. 5 : 17.

\DicoEntry{CONVOITISE}\textit{}\newline
Précédant l’acte du péché, désir amorcé par les sens humains et lié à la soif de posséder ce qui est défendu et ce que le monde offre. Voir Ja. 1: 14-15 et 1 Jn. 2 : 15-17.

\DicoEntry{CORINTHE}\textit{}\newline
Ville de la Grèce située à l'ouest d'Athènes et capitale de l'Achaïe. Après y avoir implanté une assemblée lors de son premier voyage missionnaire, Paul écrivit aux chrétiens de Corinthe deux lettres qui figurent parmi le canon biblique.

\DicoEntry{CROIX}\textit{}\newline
Châtiment romain consistant à clouer les mains et les pieds des condamnés sur des poteaux en bois, en forme de croix. Punition horriblement douloureuse réservée aux plus grands malfaiteurs, Jésus-Christ accepta de vivre cette mort pour payer le prix nécessaire au rachat du monde. Symbole du sacrifice de Jésus pour le pardon des péchés, la croix est aussi l'image de la vie de souffrance et de consécration totale à laquelle est appelée tout disciple du Seigneur. Voir Es. 53 ; Lc. 9 : 23 et Mt. 16 : 24.

\DicoEntry{CUPIDITE}\textit{}\newline
Forme d’idolâtrie, péché consistant à désirer de manière excessive les biens de ce monde (argent, richesses etc.) et menant à la perdition. Voir Pr. 1 : 19 et Col. 3 : 5.

\DicoEntry{DAGON}\textit{}\newline
Divinité païenne adoré par les philistins, il était considéré comme le protecteur du blé et des semences. Voir 1 S. 5 : 1-5.

\DicoEntry{DAN}\textit{, de l'hébreu « dan » : « un juge »}\newline
Fils de Jacob et de Bilha, servante de Rachel, il est le père de la tribu des Danites. Voir Ge. 30 : 1-6 et Ge. 49 : 16-18.

\DicoEntry{DANIEL}\textit{, de l'hébreu « daniye'l » : « Dieu est mon juge »}\newline
Issu d’une famille princière de Juda, il fut déporté de Jérusalem à Babylone pendant sa jeunesse sous le règne de Nebucadnestar. Son histoire est racontée dans le livre éponyme.

\DicoEntry{DATHAN}\textit{, de l'hébreu « dathan » : « appartenant à une fontaine »}\newline
Issu de la tribu de Ruben, fils d’Eliab et frère d’Abiram, il participa avec Koré à la révolte contre Moïse et Aaron. Voir No 16 : 1-35.

\DicoEntry{DAVID}\textit{, de l'hébreu « dawid » : « bien aimé »}\newline
Issu de la tribu de Juda et dernier fils d'Isaï, il entra dès son plus jeune âge au service du roi Saül à qui il succéda sur le trône royal. Il connut de grands succès sur les champs de bataille dont la célèbre victoire contre Goliath et fut l’auteur de nombreux psaumes. Il régna quarante-quatre ans sur Israël. Voir 1 S. 13 : 14 ; 1 S. 16 : 14-23 ; 1 S. 17 et 1 R. 2 : 10-11.

\DicoEntry{DEBORAH}\textit{, de l'hébreu « debowrah » : « abeille »}\newline
Femme de Lapiddoth, elle exerça les fonctions de prophétesse et juge en Israël. Elle fut utilisée par Dieu pour prophétiser la victoire d’Israël sur Canaan par Barak qu’elle accompagna sur le champ de bataille. Voir Jg. 4 et 5.

\DicoEntry{DELUGE}\textit{}\newline
Pluie torrentielle s’étant abattue sur la terre pendant quarante jours et quarante nuits au temps de Noé. Le déluge symbolisait le jugement de Dieu sur une génération dont la méchanceté avait atteint un niveau sans précédent. Tous les habitants et les animaux de la terre furent emportés par les eaux du déluge hormis Noé, sa famille et les animaux avec eux dans l’arche. Voir Ge. 6 à 8.

\DicoEntry{DEMONS}\textit{}\newline
Egalement appelés « esprits impurs », anges déchus ayant pris part à la révolte et à la chute de Satan. Ils peuvent posséder le corps d'une personne mais sont soumis à la puissance de Jésus au nom duquel les chrétiens peuvent les chasser. Leur fin sera le tourment éternel dans l’étang de feu avec Satan. Voir Ez. 28 : 15-19 ; Ap. 12 : 4 ; Mt. 12 : 43-45 ; Lc. 10 : 17-20 ; Mc. 16 : 17 ; Jud. 6 et Ap. 20 : 10-15.

\DicoEntry{DIABLE }\textit{(voir SATAN)}\newline

\DicoEntry{DIACRE}\textit{}\newline
Homme de bon témoignage, rempli de l'Esprit-Saint et de sagesse, le diacre devait se charger des besoins matériels des saints (Ac 6 : 1-6). Etienne, premier martyr chrétien, était diacre (Ac 6 : 5-8). Une femme pouvait aussi diaconesse comme Phoebe, de l'église de Chenchrées (Ro 16 : 1-2). Le caractère du diacre (1 Ti 3 : 8-13).

\DicoEntry{DIANE}\textit{, du grec « artemis » : « de la lumière »}\newline
Aussi appelée « Artemis d’Ephèse », divinité révérée dans toute l’Asie. Il existait un temple en son honneur à Ephèse. Voir Ac 19 : 24-37.

\DicoEntry{DINA}\textit{, de l'hébreu « diynah » : « jugement, justice »}\newline
Fille de Jacob et Léa. Elle fut enlevée et déshonorée par Sichem, fils de Hamor, prince du pays, qui aima la jeune fille. Ses frères, Siméon et Lévi vengèrent son déshonneur en tuant Hamor, Sichem et tous les mâles de leur ville. Voir Ge. 34.

\DicoEntry{DIEU }\textit{x}\newline
\DicoEntry{L'E}\textit{tre suprême, éternel, parfait, connu par ses œuvres dans sa puissance éternelle et sa divinité (Romains 1:20 ; Psaume 19:1). Il est le Seigneur Tout-puissant (Apocalypse 4:8). Créateur du ciel et de la terre (Ge 1 : 1). Il s'est révélé dans la Personne du Seigneur Jésus-Christ, Dieu manifesté en chair (1 Timothée 3:16 ; Jean 1:14). « Dieu est amour » (1 Jean 4:8 ; Jean 3:16). « Dieu est lumière » (1 Jean 1:5). Il est notre Père (Jean 20:17). Les caractéristiques et les attributs principaux de Dieu mentionnés dans l'Ecriture sont : Unique, un et indivisible (Deutéronome 6:4) ; Eternité (Esaïe 57:15 ; 1 Timothée 1:17) ; Immortalité (1 Timothée 6:16 ; Psaume 90:2) ; Omnipotence (Job 11:7 ; Romains 1:20) et Souveraineté (1 Timothée 6:15) ; Invisibilité (1 Timothée 1:17 ; 6:16) ; Omniprésence (Psaume 139:7-10 ; Jérémie 23:23, 24) ; Omniscience (1 Chroniques 28:9 ; Jérémie 1:5 ; Romains 8:29, 30 ; Hébreux 4:13) ; Incorruptibilité (Romains 1:23 ; Jacques 1:13) ; Immutabilité (Mal. 3:6 ; Jacques 1:17) ; Sagesse (Psaume 104:24 ; Romains 11:33-36) ; Sainteté (Amos 4:2 ; Luc 1:49) ; Justice (Romains 2:5-7 ; 2 Timothée 4:8) ; Grâce et miséricorde (Luc 1:50 ; Romains 3:24 ; Ephésiens 2:4) ; Patience (Romains 2:4 ; 15:5) ; fidélité (Psaume 92:1, 2 ; 1 Corinthiens 1:9).}\newline

\DicoEntry{DIME}\textit{}\newline
Dixième d’un ensemble. Abraham donna à Melchisédek la dîme du butin d'une bataille remportée (Ge. 14 : 17-20 et Hé. 7 : 1-2). Yahweh instaura, au travers de Moïse, la dîme comme une loi à respecter par les enfants d’Israël. Il en existait quatre sortes :
- la dîme que les Lévites prélevaient sur le peuple (No. 18 : 21-24)
- la dîme de la dîme, que les sacrificateurs prélevaient sur les Lévites (No. 18 : 25-31 ; Né. 10:38)
- la dîme consommée par les juifs eux-même lors des fêtes de Yahweh (De. 14 : 22-26)
- la dîme pour l’étranger, la veuve, l’orphelin et le Lévites, donnée tous les trois ans (De. 14 : 28-29).
Cette loi concernait exclusivement Israël et non l’Eglise – Jésus Christ ayant accompli la loi. Sous la grâce, les chrétiens sont exhortés à donner libéralement et sans contrainte. Voir Mt. 5 : 17 et 2 Co. 9 : 6.

\DicoEntry{DISCIPLE}\textit{}\newline
Personne qui écoute les enseignements de son maître et les met en pratique en vue de devenir comme lui. Jésus en choisit douze qu’il forma pendant son ministère. Le disciple de Christ doit manifester le caractère de son maître, lui être pleinement consacré et être prêt à souffrir en son nom. Voir Lc. 6 : 12-16 ; Lc. 14 : 26-33 et Mt. 10.

\DicoEntry{DIVORCE}\textit{}\newline
Brisement des liens du mariage. Il fut autorisé sous la loi de Moïse à cause de la dureté des cœurs mais Christ rappela l'indissolubilité du mariage au commencement. Voir Mt. 19 : 3-8.

\DicoEntry{DOCTEUR}\textit{, du grec « didaskalos » : « professeur »}\newline
Sous la loi de Moïse, les docteurs de la loi était chargés d'expliquer la Tohra. Certains d’entre eux s'opposèrent à Jésus. Sous la nouvelle alliance, le docteur est un des cinq ministères de la Parole. Il enseigne la Parole de Dieu avec l'éclairage du Saint Esprit. Voir Lc. 2 : 46 ; Lc. 5 : 17 ; 1 Co. 12 : 28 et Ep. 4 : 11.

\DicoEntry{SPIRITUELS}\textit{}\newline
Capacités distribuées par le Saint Esprit aux chrétiens en vue de la formation et de l'édification des saints. Ce sont des grâces accordées par Dieu pour l’utilité commune. Voir 1 Co. 12 : 1-11 ; 1 Co. 14 : 12 ; Ro. 12 : 6 et 1 Pi. 4 : 10 .

\DicoEntry{EGLISE}\textit{, du grec « ekklesia » : « appel hors de »}\newline
Peuple mis à part dont Christ est le chef. L’église est la sainte habitation de Dieu en esprit, le corps de Christ, l’épouse de l’Agneau. On distingue l’église universelle - qui regroupe tous les saints du monde entier - de l’église locale - qui est composée de tous les chrétiens d'une ville. Voir 1 Co. 3 : 16 ; Ep. 2:20-22 ; 1 Co. 12 : 27 ; Ep. 5 : 22-32 ; Ac. 2 : 47 ; 1 Co. 1:2 ; 1 Th. 1:1 ; Ph. 1:1.

\DicoEntry{ELECTION}\textit{}\newline
Choix parfait et omniscient de Dieu. Israël est le peuple élu de Dieu (Es 45 : 4), les chrétiens sont de même élus (Ep. 1: 4-6 ; 2 Pi. 1 : 10), ainsi que les anges (1 Ti 5 : 21).

\DicoEntry{ELIE}\textit{, de l’hébreu « Eliyah » : « Yahweh est mon Dieu »}\newline
Prophète d’origine tschibite que Dieu suscita en Israël au temps du roi Achab. Son histoire, ses combats et ses exploits sont racontés dans les livres des Rois.

\DicoEntry{ENLEVEMENT}\textit{, du grec « metatithemi » : « transfert d’un lien à un autre »}\newline
Ravissement d'hommes au ciel sans que ces derniers ne connaissent la mort physique. Dans le tanakh, se trouvent deux cas d'enlèvement : Hénoch (Ge. 5 : 24 ; Hé. 11 : 5) et Elie (2 R. 2 : 11). L'Eglise sera de même enlevée par le Seigneur au son de la dernière trompette. Voir 1 Co. 15 : 51-57 et 1 Th. 4 : 17.

\DicoEntry{EPHESE}\textit{, du grec « Epheseos » : « permis »}\newline
Figurant parmi les principales villes de l’Empire romain sous le règne de l’empereur Claude 1er (10 av. J.-C. – 54 apr. J.-C.) Ephèse possédait le plus grand port de l’Asie Mineure, ce qui lui attribuait le contrôle du trafic commercial. Richissime et prospère, elle était renommée pour son faste, sa liberté de parole et constituait donc un endroit privilégié pour les philosophes.
L’église d’Ephèse naquit du ministère de Paul, qui y enseigna pendant au moins deux ans lors de son troisième voyage missionnaire. Cette église - figurant parmi les sept du livre d’Apocalypse - fit preuve de discernement et pratiquait de bonnes œuvres mais le Seigneur avait néanmoins un reproche à lui adresser. Voir Epître aux éphésiens et Ap. 2 : 1-7.

\DicoEntry{EPHRAIM}\textit{, de l'hébreu « ephrayim »: « double fertilité »}\newline
Second fils de Joseph né en Egypte, il fut adopté par Jacob avant sa mort et devint ainsi l’ancêtre d’une des douze tribus d’Israël : la tribu d'Ephraïm. Voir Ge. 41 : 52 ; Ge. 48 : 5 et Jos. 14 : 4.

\DicoEntry{ESAIE}\textit{, de l'hébreu « yesha yah » : « le secours de Yahweh»}\newline
Premier prophète majeur en Israël et contemporain des rois, Ozias, Jotham, Achaz et Ezéchias, il annonça la venue du Messie. L’ensemble de ses prophéties est contenu dans le livre portant son nom.

\DicoEntry{ESA}\textit{Ü, de l’hébreu « esav » : « velu »}\newline
Fils d’Isaac et Rebecca et frère jumeau de Jacob, on le connait aussi sous le nom d’Edom. Il vendit son droit d'aînesse pour un plat de lentilles et devint l'ancêtre des Edomites. Voir Ge. 25 : 25-34 et Ge. 36.

\DicoEntry{ESPRIT}\textit{}\newline
Partie invisible et immatérielle de la personne, qui quitte le corps à la mort (Lu 8 : 55). Essence dépourvu de toute matière, possédant un pouvoir de connaissance, de désir, de décision, d'action.

\DicoEntry{IMPUR }\textit{(voir DEMONS)}\newline

\DicoEntry{FEU}\textit{}\newline
Lieu de douleur et de damnation éternelle où seront jetés la bête et le faux prophète, le diable, la mort et le séjour des morts ainsi que tous ceux dont le nom ne trouvera pas dans le livre de vie. Voir Ap. 19 : 20 et Ap. 20 : 7-15.

\DicoEntry{ETIENNE }\textit{du grec « Stephanos » : « couronne »}\newline
Diacre de l'église de Jérusalem rempli de sagesse et d'Esprit Saint. Premier martyr chrétien, sa mort marqua le début d'une grande persécution contre l’église. Voir Ac. 6 : 1-6 ; Ac. 7 et Ac. 8 : 1-3.

\DicoEntry{EUNUQUE}\textit{}\newline
Homme dans l'incapacité de procréer ou émasculé. Dans l'antiquité, les rois se choisissaient des eunuques pour les servir. En les castrant, ils s’assuraient de la fidélité et l’intégrité de ces derniers. Voir 2 R. 20 : 18 ; 1 Ch. 28 : 1 et Da. 1 : 7. En outre, Jésus distinga 3 types d'eunuques (Mt. 19 : 12).

\DicoEntry{EVANGELISTE}\textit{, du grec « euaggelistes » : « messager de bonne nouvelle »}\newline
Un des cinq ministères de la Parole dont la mission est principalement d’annoncer et publier la bonne nouvelle de l’évangile. Philippe exerça ce ministère, Timothée fut de même encouragé à faire l'œuvre d'un évangéliste. Voir Ep. 4 : 11 ; Ac. 21:8 et 2 Ti. 4 : 5.

\DicoEntry{EVANGILE}\textit{, du grec « evangelion » : « bonne nouvelle »}\newline
Message que Jésus est venu prêcher, à savoir la repentance, le royaume des cieux et le salut accordé par la foi en Jésus-Christ. Les apôtres propagèrent l'évangile ; de même, tous les chrétiens sont appelés à le faire. Voir 1 Co. 15 : 1-4 ; Mt. 28 : 19-20 ; Ac. 16 : 31.

\DicoEntry{EVE}\textit{, de l'hébreu « chavvah » : « vie »}\newline
Femme d'Adam et mère de tous les hommes, elle fut formée à partir de la côte de son mari dans le but d’être l’aide de ce dernier. Séduite par Satan déguisé en serpent, elle mangea le fruit de la connaissance du bien et du mal et fut avec Adam chassée du jardin. Elle donna naissance à Caïn, Abel et Seth. Voir Ge. 2 : 18-24 ; Ge. 3 : 1-13 et Ge. 4 : 1-2 + 25.

\DicoEntry{EVEQUE }\textit{(voir ANCIEN)}\newline

\DicoEntry{SACRIFICE }\textit{D’EXPIATION}\newline
Action de couvrir les fautes et les souillures de l'homme afin qu'il soit réconcilié avec Dieu (Ex 29 : 36-37). Christ a expié les péchés de l'homme en les prenant sur lui à la croix ; Il est l'Agneau qui ôte les péchés du monde (Jn 1 : 29). Par cet acte, l'homme est gratuitement justifié par la grâce de Dieu (Ro 3 : 23-24 ; Ep 2 : 8-9)

\DicoEntry{EZECHIAS}\textit{, de l'hébreu « yechizqiyah » : « Yahweh est ma force »}\newline
Fils d'Osée, il fut le treizième roi de Juda. Figurant parmi les rois les plus intègres, son règne fut caractérisé par la droiture et la fidélité à Yahweh. Il régna 29 ans en Juda. Son histoire est racontée dans les chapitres 18 et 19 du deuxième livre des Rois.

\DicoEntry{EZECHIEL}\textit{, de l'hébreu « Yechezqe'l » : « Dieu fortifie »}\newline
Fils de Buri, le sacrificateur, Ezechiel était un prophète d’origine Lévite ayant été déporté à Babylone. Il reçut de nombreuses visions - sur son temps et les temps de la fin - racontées dans le livre qui porte son nom.

\DicoEntry{YAHWEH}\textit{}\newline
Selon la loi juive, sept fêtes étaient célébrées en l’honneur de Yahweh : la Pâque de Yahweh, la fête des pains sans levain ; la fête des prémices ; la Pentecôte ; la fête des trompettes ; le jour des expiations ; la fête des tabernacles. Voir Lé. 23 : 6-43.

\DicoEntry{FIGUIER}\textit{}\newline
Arbre fruitier sous lequel il était coutume d’étudier la Torah en Israël. Ses fruits bons et doux servent en médecine. Le figuier est utilisé dans de nombreuses histoires et paraboles. Voir Jg. 9 : 11 ; 2 R. 20 : 1-7 et Jn. 1 : 43-51 et Lc. 13 : 6-9 .

\DicoEntry{FOI}\textit{, du grec « pistis » : « conviction de la vérité »}\newline
Confiance en la véracité de Dieu, ses paroles et l’accomplissement de ses promesses. Bien qu’il n’existe qu’une seule foi, elle est présentée sous trois formes principales sous la nouvelle alliance :
- en tant que fruit de l’esprit, c’est la foi qui sauve (Ga. 5 : 22 et Ro. 10 : 9)
- en tant que don de l’esprit, c’est la foi accordée pour accomplir une tâche particulière (1 Co. 12 : 9)
- en tant que parole, c’est la foi liée à la saine doctrine, la vérité (Ro. 10 : 17 et 2 Ti. 4 : 7)
Condition sine qua none pour être agréable à Dieu et élément déclenchant les miracles, la foi est éprouvée tout au long de la vie du croyant. Voir Hé. 11 ; Lc. 7 : 50 et 1 Pi. 1 : 7.

\DicoEntry{FORNICATION }\textit{ou IMPUDICITE, du grec « poerneia » : « prostitution du corps »}\newline
Relations sexuelles en dehors du mariage et tous rapports sexuels illicites. La Parole condamne fermement ces actes. Voir Lé. 18 ; 1 Co. 6 : 13 + 16-18 et 1 Co. 7 : 2.

\DicoEntry{FRUIT }\textit{de l'Esprit}\newline
Résultat de l'action de l'Esprit Saint dans l’homme intérieur dans le but de communiquer le caractère de Dieu au chrétien né d’en haut. Voir Ga. 5 : 22.

\DicoEntry{GABRIEL}\textit{, de l'hébreu « Gabriy'el » : « héros de Dieu » ou « homme de Dieu »}\newline
Archange que Dieu envoya pour délivrer des messages importants à Daniel, Zacharie et Marie. Voir Da. 9 : 21-27 ; Lc. 1 : 11-20 et Lc. 1 : 26-38.

\DicoEntry{GAD}\textit{, de l’hébre « gawd » : « bonheur », « heureux », « troupe »}\newline
Fils de Jacob et Zilpa, servante de Léa, il est l'ancêtre de la tribu de Gad. Voir Ge. 30 : 11 et Ge. 49 : 16.

\DicoEntry{GALATIE}\textit{}\newline
Région située dans l'actuelle Turquie, autour d'Ankara. Lors de ses voyages missionnaires, Paul traversa la Galatie, dans laquelle il y avait plusieurs assemblées (Ga. 1 : 1-2 ; 1 Co. 16 : 1).

\DicoEntry{GALILEE}\textit{, de l’hébreu « galilaia » : « Circuit »}\newline
Région située au nord de la Palestine dans laquelle se trouve la localité de Nazareth où Jésus grandit. Il y débuta son ministère, c’est aussi là qu’il se montra vivant à ses disciples après la résurrection. Les disciples de Jésus étaient originaires de Galilée. Voir Mt. 2 : 19-23 ; Jn. 2 ; Mc. 16 : 7 ; Ac. 1 : 11 et Ac. 2 : 7.

\DicoEntry{GARIZIM}\textit{, de l'hébreu « Geriziym » : « lieux arides »}\newline
Montagne située au sud de Sichem, en face du mont Ebal, de laquelle les enfants d’Israël devaient prononcer la bénédiction une fois entré en Canaan. Voir Jg. 9 : 7 ; De. 11 : 29 et Jos. 8 : 33.

\DicoEntry{GEDEON}\textit{, de l'hébreu « gid own » : « coupant, abattant »}\newline
Issu de la tribu de Manassé et fils de Joas. L'ange de Yahweh lui apparut et le mandata pour délivrer Israël de la main des Madianites. Gédéon fut juge en Israël pendant quarante ans, période durant laquelle la paix régna dans le pays. Voir Jg. 6 à 8.

\DicoEntry{GEHENNE}\textit{, du grec « geena » : « vallée de Hinnom »}\newline
Initialement, vallée située au sud de Jérusalem où des enfants étaient jetés dans le feu en sacrifice à Moloc (2 R. 23 : 10). Le terme « géhenne » représente la destruction future des méchants et se rapporte à l'étang de feu (voir ETANG DE FEU). Voir Mt. 10 : 28.

\DicoEntry{GOG}\textit{}\newline
Très certainement, le chef du pays de Magog. Voir Ez. 38 : 2 + 18.

\DicoEntry{GOLGOTHA}\textit{, de l’hébreu « golgotha » : « crâne »}\newline
Lieu de la crucifixion de Jésus-Christ, situé non loin de Jérusalem. Voir Jn. 19 : 17-20.

\DicoEntry{GRACE}\textit{, du grec « charis » : « bonne volonté », « bonté », « faveur »}\newline
Don immérité de Dieu, elle est la source du salut de tous les hommes et invite à la crainte de Dieu. La grâce est venue par Jésus Christ et fut révélée au travers de l’œuvre parfaite de la croix. Voir Jn. 1 : 17 ; Ti. 2 : 11-12 et Ro. 3 : 23-24.

\DicoEntry{TRIBULATION}\textit{}\newline
Voir commentaire

\DicoEntry{HARMAGUEDON}\textit{, de l’hébreu « Armageddon » : « montagne de Méguiddo »}\newline
Lieu situé au nord d'Israël dans la tribu de Zabulon où mourut le roi Josias. Pendant le millenium, les rois et puissants de la terre s’y rassembleront pour combattre Yahweh et son armée. Voir 2 R. 23 : 29 et Ap. 16 : 13-16.

\DicoEntry{HELLENISTE}\textit{}\newline
Israélites nés hors de la terre promise ayant adoptés le mode de vie grec et parlant la langue grecque. Voir Ac. 6 : 1.

\DicoEntry{HENOCH}\textit{, de l’hébreu « chanowk » : « consacré, dédié »}\newline
Fils de Jéred et père de Metuschéla. Homme pieux ayant vécu 365 ans avant d’être enlevé au ciel sans connaître la mort. Voir Ge. 5 : 21-24 et Hé. 11 : 5.

\DicoEntry{GRAND}\textit{}\newline
Roi de Judée, il fut l'instigateur du massacre des enfants de la région de Bethléhem au moment de la naissance de Jésus. Avertie par un ange du Seigneur, la famille de ce dernier fuit en Egypte à cause de lui et revint en Israël seulement à sa mort. Voir Mt. 2.

\DicoEntry{ANTIPAS }\textit{(ou le Tétrarque)}\newline
Fils d'Hérode le Grand, il exerça la fonction de tétrarque de Galilée et fut contemporain à Jésus-Christ homme pendant presque toute la vie de ce dernier. Hérode épousa sa belle-sœur Hérodias et fit décapiter Jean-Baptiste. Il fut qualifié de « renard » par Jésus et s’accorda avec son ennemi Pilate lors de la crucifixion du Seigneur. Voir Lc. 3 : 1 ; Mc. 6 : 14-28 ; Lc. 13 : 31-32 et Lc. 23 : 8-12.

\DicoEntry{AGRIPPA }\textit{Ier}\newline
Roi et tétrarque de Judée et petit fils du roi Hérode le Grand, il accéda au pouvoir à la genèse de l’église primitive. Pour plaire aux juifs, il fit mourir Jacques et emprisonna Pierre. Il mourut brusquement après avoir reçu du peuple la gloire qui devait revenir à Dieu. Voir Ac. 12.

\DicoEntry{II}\textit{}\newline
Fils d’Agrippa Ier, il est appelé « roi Agrippa » dans les écritures. Il fut inspecteur du temple de Jérusalem et avait le pouvoir de choisir les grands sacrificateurs. Il rencontra Paul à Césarée lors d’une visite au gouverneur Festus. Voir Ac. 25 et 26.

\DicoEntry{HOLOCAUSTE}\textit{, de l’hébreu « ola » : « offrande entièrement consumée »}\newline
Prescrit par la loi de Moïse, sacrifice consumé par le feu d'une agréable odeur à Yahweh. Il préfigurait le sacrifice à la croix de Jésus-Christ, l’agneau de Dieu. Voir Lé. 1 : 1-17 ; Hé. 9 : 11-22 et Hé. 10 : 1-19.

\DicoEntry{HOMOSEXUALITE}\textit{}\newline
Pratique abominable et fermement réprouvée par Dieu consistant en l’union de deux personnes du même sexe. Voir Lé. 18 ; Ro. 1: 24-32 et 1 Co. 6 : 9-10.

\DicoEntry{IDOLE}\textit{, IDOLATRIE, du grec « eidolon » : « image »}\newline
Une idole peut être l’image d’un faux dieu, l’image faussée de Yawheh ou encore une personne, un objet, une activité à qui l’on donne le rang de Dieu. L’idolâtrie est le culte rendu à ces idoles.
Les commandements de Yahweh tout comme sa nature (il est le Dieu jaloux) condamnent fermement l’idolâtrie. Ce péché pouvant mené à la mort ne doit pas être trouvé chez les enfants de Dieu. Voir 1 R. 15 : 11-13 ; Ex. 20 : 3-5 ; Ex. 32 ; 1 Co. 6 : 9 ; Ep. 5 : 5 et Col 3 : 5.

\DicoEntry{MAINS}\textit{}\newline
Acte de poser les mains sur une personne. Avant leur mort, les patriarches imposaient les mains à leurs enfants pour les bénir (Ge. 48 : 14). Moïse imposa également les mains à Josué qui devait lui succéder (De. 34 : 9). Sous la nouvelle alliance, on peut imposer les mains à quelqu’un en vue de lui transmettre la guérison divine, l’autorité liée à une fonction particulière, les dons spirituels et même le Saint Esprit dans certains cas. Ce geste ne doit cependant pas être fait dans la précipitation. Voir Lc. 4 : 40 ; Mc. 16 : 18 ; 1 Ti. 4 : 14 ; Ac. 6 : 6 ; Ac. 8 : 17 et psg précipitation.

\DicoEntry{INCORRUPTIBILITE}\textit{}\newline
Terme désignant ce qui ne peut ni se corrompre, ni se flétrir, ni se détruire. La parole et l’amour de Dieu sont incorruptibles. A l'enlèvement de l'Eglise, les morts en Christ ressusciteront incorruptibles et les chrétiens revêtiront de même des corps incorruptibles. Voir Mt. 24 : 35 ; 1 Co. 13 et 1 Co. 15 : 40-57.

\DicoEntry{INCREDULITE}\textit{, du grec « apistos » : « sans foi »}\newline
Rejet, doute par rapport à la véracité de Dieu et de sa parole. L'apôtre Thomas fit preuve d’incrédulité quant à la résurrection de Christ avant de le voir vivant. Les incrédules ne peuvent hériter le royaume de Dieu. Voir Jn. 1 : 1-14 ; Jn. 14 : 6 ; Jn. 20 : 24-29 et Ap. 21 : 8.

\DicoEntry{INIQUITE}\textit{, du grec « adikia » : « injustice »}\newline
Ce qui va à l'encontre de la volonté et de la justice de Dieu. Voir Ro. 1 : 18 ; 2 Th. 2 : 7 et 1 Jn. 5 : 17).

\DicoEntry{INTERCESSION }\textit{« enteuxis »}\newline
Prière adressée à Dieu en faveur d'autres personnes que soi-même. Les chrétiens intercèdent pour les hommes (Ac. 12 : 5 ; 1 Ti. 2 : 1-2) et Christ pour les saints (Jn. 17 : 6-10 ; Ro. 8 : 34). Le Saint Esprit en nous, intercède pour nous par des « soupirs inexprimables » (Ro 8 : 26).

\DicoEntry{ISAAC}\textit{, de l'hébreu « Yitschaq » : « rire, il rit »}\newline
Fils d'Abraham et de Sara, il fut le deuxième patriarche d'Israël.
Pour éprouver la foi d'Abraham, Dieu lui demanda Isaac en sacrifice, mais Yahweh l'épargna (Ge. 22 : 1-13). Il eut pour femme Rebecca, et Dieu lui confirma l'alliance avec Abraham (Ge. 26 : 2-5). Il accumula une grande richesse à Guérar (Ge. 26 : 12-14) et fut le père de Jacob et Esaü (Ge. 25 : 19-26).

\DicoEntry{ISSACAR}\textit{, de l'hébreu « Yissaskar » : « il donnera un salaire »}\newline
Fils de Jacob (Ge. 30 : 18), il est l'ancêtre de la tribu d'Isaacar.

\DicoEntry{ISRAEL}\textit{, de l'hébreu : « Yisra'el » : « Dieu prévaut, lutteur avec Dieu »}\newline
Ce fut le nom que Dieu donna à Jacob après voir lutté avec lui (Ge. 32 : 28). Le nom « Israël » regroupa tous les hébreux ainsi que leur territoire, puis le royaume unifié dont la capitale fut Jérusalem, et les dix tribus du nord après le schisme. Jésus a été acclamé comme le Roi d'Israël (Jn. 1 : 49 ; 12 : 13).

\DicoEntry{IVRAIE }\textit{(zizanion ; a donné : zizanie)}\newline
Comme le blé, l'ivraie est une plante de la famille des graminées.
Cette plante veneneuse ne peut se différencie du blé quand elle est en herbe, tellement que son aspect est identique. L'ivraie est l'image des fils du diable qui sont destinés à la damnation éternelle (Mt. 13 : 36-42).

\DicoEntry{JACOB}\textit{, de l'hébreu « Ya`aqob » : « celui qui prend par le talon » ou « qui supplante »}\newline
Fils d'Isaac et de Rébecca, frère d'Esaü, il fut le troisième patriarche d'Israël. Il obtint le droit d’aînesse (Ge. 25 : 27-34) et recût la bénédiction d'Isaac (Ge. 27 : 27-29). L'alliance avec Abraham lui fut également confirmée (Ge. 28 : 13-15). Il bâtit un autel à Yahweh à Béthel. Menacé par son frère Esaü, il fuit à Charan chez son oncle Laban où il prit Léa et Rachel pour femmes (Ge. 29 : 1-30). Sa richesse augmenta et il retourna habiter en Canaan. Ses douze fils donnèrent naissances aux douze tribus d'Israël (Ge. 49 : 1-28).

\DicoEntry{JACQUES}\textit{, du grec : « Iakobos » : « qui supplante »}\newline
- Fils de Zébédée et frère de Jean : Jésus l'appela à le suivre et il devint un des douze apôtre (Mt. 4 : 21-22). Jacques était avec Jésus sur la montagne de la transfiguration (Mc. 9 : 2) et à Gethsémané (Mc. 14 : 33). Le roi Hérode Agrippa Ier le fit mourir par l'épée (Ac. 12 : 1-2).
- Fils d'Alphée et frère de Jude : Il fut l'un des douze apôtres et est appelé le mineur (petit) (Mt. 10 : 3 ; Mc 15 : 40)
- Fils de Joseph et frère de Jésus (Ga. 1 : 19). Il fut l'un des douze apôtre et a probablement écrit l'épître du même nom (Jacques 1 : 1).

\DicoEntry{JAPHET}\textit{, de l'hébreu « Yepheth » : « ouvert », « qui s'étend »}\newline
Dernier des trois fils de Noé (Ge. 10 : 1).

\DicoEntry{JEAN}\textit{, de l'hébreu « Yowchanan » : « l'Eternel a gracié »}\newline
- Fils de Zébédée, frère de Jacques et disciple du Seigneur. Il était le disciple que Jésus aimait (Jn. 13 : 23). Sur la croix, Christ lui confia Marie (Jn 19 : 27). Jean est l'auteur de trois épîtres qui portent son nom, et de l'Apocalypse qu'il a écrite en exil sur l'île de Patmos.
- Fils de Zacharie et de Elisabeth, il était cousin de Jésus (Lu. 1 : 34-38). Son apparition fut prophétisée par Malachie ( Mal 3 : 1-6). Jean fut envoyé de Dieu comme précurseur du Christ, afin de lui rendre témoignage et annoncer sa venue (Jn 1 : 6-9). Appelé Jean le baptiseur, Il prêchait la repentance et le royaume des cieux, tout en baptisant d'eau dans le Jourdain (Mt. 3 : 1-6). Jésus le définit comme le plus grand des prophètes (Lu. 7 : 28). Hérode Antipas le fit décapiter (Mt. 14 : 1-12).

\DicoEntry{JEREMIE}\textit{, de l'hébreu « Yirmeyah » : « celui que Yahweh a désigné »}\newline
Prophète de Yahweh et fils du sacrificateur Hilkijia. Il fut contemporain du roi Josias (Jé. 1 : 2), et vit la déportation babylonienne à partir du règne de Jojakim jusqu'à Sédécias (Jé. 1 : 3). Il proclama le retour dans un royaume de Judas gangrené par l'apostasie (Jé. 2 ; 3 ; 10 ; 11) et dénonça les faux prophètes qui sévissaient dans le pays (Jé. 23). Il prophétisa la chute de Jérusalem (Jé. 6 ; 22 : 1-9) et pleura sur la ville après sa destruction (La. 1). Il fut eunuque durant toute sa vie et n'eut point d'enfants (Jé. 16 : 1-2).

\DicoEntry{JERICHO}\textit{, de l'hébreu « Yeriychow » : « ville de la lune ou ville des palmiers »}\newline
Ville située à l'est de la tribu de Benjamin, près des rives du Jourdain.
Rahab la prostituée, y cacha les espions hébreux (Jos. 2). La ville fut par la suite prise et maudite par Josué (Jos. 6 ; Hé. 11 : 30). Jésus-Christ y guérit l'aveugle Bartimée (Mc. 10 : 46-53) et il fut reçut par Zachée dont la maison se trouvait dans la ville (Lu. 19 : 1-10).

\DicoEntry{JEROBOAM}\textit{, de l'hébreu « Yarob`am » : « le peuple devient nombreux »}\newline
Serviteur de Salomon, il devint plus tard son ennemi. Après le schisme, il fut le premier roi du royaume d'Israël composé des dix tribus du nord. Il fut un roi apostat et établit des cultes païens (1 R. 12 : 25-33 ; 13 : 33-34). Il régna vingt-deux ans sur le royaume (1 R. 12 : 20).

\DicoEntry{JERUSALEM}\textit{, de l'hébreu « Yeruwshalaim » : « fondement de la paix »}\newline
Ville située en Palestine, au nord de la Judée. Lors de la conquête de Canaan, la ville fut sous le contrôle des jébusiens. Aux environs du Xe siècle av. J.-C., David reprit la ville alors devenue forteresse Jébisienne (2 S. 5 : 6-8). Il en fit la capitale politique (2 S. 5 : 9) et religieuse du royaume en y faisant établir l'arche de l'alliance (2 S. 6). Salomon y construisit le temple sur le mont Morija (2 Ch. 3 : 1). En 586 av. J.-C., bien après le schisme (voir SCHISME), les Babyloniens la détruisirent. Elle fut rebâtie par Néhémie après le retour de la captivité babylonienne. Jésus-Christ se lamenta sur la ville à cause de son incrédulité et y annonça sa futur destruction (Lu. 19 : 41-44). Jérusalem fut en effet détruite par le général romain Titus en 70 ap. J.-C puis de nouveau rebâtie. Lors de son retour glorieux, le Seigneur Jésus posera ses pieds sur le Mont des Oliviers qui est situé à Jérusalem (Za. 14 : 1-4). Apocalypse nous annonce après la fin du monde l'apparition de la nouvelle Jérusalem (Ap. 21).

\DicoEntry{JESUS}\textit{, de l'hébreu « Yehowshuwa` » : « Yahweh est salut »}\newline
Nom du Seigneur en tant qu'homme (Mt. 1 : 21), Jésus a été conçu dans le ventre de Marie par la puissance du Saint Esprit alors que cette dernière n'avait point connu d'homme. Il est le fils de Joseph le charpentier (Mt. 13 : 55) et le cousin de Jean le baptiseur. Il a vécu en Galilée, dans la ville de Nazareth. Vers l'âge de 30 ans, Il se fit baptiser par Jean dans le Jourdain et commença par la suite son ministère public. Jésus est venu sauver les hommes en se presentant comme l'Agneau de Dieu (Jean 1 : 29). Par conséquent, Il a fait de tous ceux qui l’ont reçu, le droit d’être enfants de Dieu (Jn. 1:12). Ceux qui sont sanctifiés en lui sont affranchis du péché ; ils sont justifiés et se trouvent dans une nouvelle position en Christ, par le Saint-Esprit. Il fut trahit par Judas Iscariot, arrêté puis crucifié mais le troisième jour, il ressuscita d’entre les morts.

\DicoEntry{JETHRO}\textit{, de l'hébreu « Yether » : « son abondance, excellence »}\newline
Nom du beau-père de Moïse (Ex. 3 : 1).
Après sa fuite d'Egypte, Moïse se réfugia chez les Madianites, où Jéthro est le sacrificateur. Moise épousa Séphora, la fille de Jethro (Ex. 2 : 21).

\DicoEntry{JE}\textit{ÛNE (nèsteia ; « état de celui qui ne mange pas »}\newline
Abstinence plus ou moins complète de nourriture (2 S. 12 : 16-23 ; Da. 10 : 3 ; Lu. 2 : 37). Le jeune est une attitude d'affliction et de rabaissement de la chair. C'est aussi une humiliation du cœur pour être entendue du Seigneur (Da 10 : 3).

\DicoEntry{JEZABEL}\textit{, de l'hébreu « Iyzebel » : « Baal est l'époux ou impudique »}\newline
Femme du roi Achab (1 R. 16 : 31). Cruelle et apostate, elle extermina les prophètes de Yahweh (18 : 4), et fit mourir Naboth afin qu'Achab prenne possession de sa vigne. Symboliquement elle représente ceux qui, dans la chrétienté, allient des pratiques idoles.

\DicoEntry{JOB}\textit{, de l'hébreu « Iyowb » : « haï, ennemi ou Je m'exclamerai »}\newline
« Homme intègre et droit », il fut originaire du pays d'Uts (Job 1 : 1). Dieu permit à Satan de lui ravir ses biens et de lui altérer la santé afin d'éprouver son intégrité (Job 1 : 2). Cette épreuve fut aussi le moyen que Dieu utilisa pour se révéler plus profondément à lui (Job 42 : 5). L'apôtre Jacques témoigne de sa patience dans l'épreuve (Ja. 5 : 11).

\DicoEntry{JOEL}\textit{, de l'hébreu « Yow'el » : « Yahweh est Dieu »}\newline
Prophète du Tanakh, il annonça la venue du Saint-Esprit « sur toute chair » (Joë. 2 : 28 ; Ac. 2 : 1-21).

\DicoEntry{JONAS}\textit{, de l'hébreu « Yonah » : « colombe »}\newline
Prophète de Dieu qui fut envoyé à Ninive pour y prêcher la repentance (Jon. 1: 1). Cependant, son aversion pour les ninivites (Jon. 4 : 1-11) le conduisit à désobéir à l'ordre de Dieu et se retrouver dans le ventre d'un grand poisson (Jon. 1 : 2-11). Finalement il prêcha aux ninivites et ses derniers se repentirent (Jon. 3 : 1-10).

\DicoEntry{JOSEPH}\textit{, de l'hébreu « Yowceph » : « que Yahweh ajoute ou il enlève »}\newline
- Fils de Jacob (Ge. 30 : 24). Jaloux de l'amour de leur père pour Joseph, ses frères le jetèrent dans une citerne et des marchands Madianites le vendirent comme esclave en Egypte (Ge. 37). Après un passage en prison dû à la fausse accusation de la femme de son maître, il sorti et devint gouverneur en Egypte (Ge. 39 à 40). Il sauva sa famille de la famine qui sévit en Canaan en la faisant descendre en Egypte (Ge. 45 : 16-28 ; 46). Ses deux fils, Ephraïm et Manasée constituèrent deux tribus (Jos. 14 : 4).
- Epoux de Marie, la mère nourricière de Jésus. Il fut divinement averti de la naissance de Jésus (Mt. 1 : 18-25). Lors du massacre des innocents, il fuya en Egypte (Mt. 2 : 13-18) puis revint vivre en Galilée après la mot d'Hérode et habita à Nazareth (Mt. 1 : 19-23).

\DicoEntry{JOSIAS}\textit{, de l'hébreu « Yo'shiyah » : « Yahweh guérit »}\newline
Fils d'Amon, il fut le seizième roi de Juda et y régna durant 31 ans.
Grand réformateur, Josias fut à l'origine d'un immense réveil spirituel en Juda. Il répara le temple (2 R. 22 : 3-10), lut le livre de la loi devant le peuple et fit alliance avec Yahweh (2 R. 23 : 1-3). Il retira toutes les idoles des dieux païens (2 R. 23 : 4-6) ainsi que les prostituées (2 R. 23 : 7) et rétablit la Pâque à Israël (2 R. 23 : 21-25).

\DicoEntry{JOSUE}\textit{, de l'hébreu « Yehowshuwa`» : « Yahweh est salut »}\newline
Successeur de Moïse (Jos. 1 : 1-9), il guida les enfant d'Israël à la conquête de Canaan (Jos. 6 à 22).

\DicoEntry{JOURDAIN}\textit{, de l'hébreu « Yarden » : « celui qui descend »}\newline
Très certainement le fleuve le plus connut de la Bible. Il est situé aux limites est de l'actuel territoire d'Israël. Josué et le peuple d'Israël passèrent le fleuve à sec (Jos. 3).
Par la puissance de Dieu, Elie (2 R. 2 : 8), puis Elisée (2 R. 2 : 12-14) partagèrent les eaux du fleuve en deux. Jésus s'y fit baptiser (Mt 3 : 13-17).

\DicoEntry{SEIGNEUR}\textit{}\newline
Moment où Dieu jugera et frappera les nations à cause des péchés commis par l'humanité (Es. 13 : 6-16 ; So. 1).

\DicoEntry{JUDA}\textit{, de l'hébreu « Yehuwdah » : « Yahweh sera loué »}\newline
Fils de Jacob (Ge. 29 : 35), il fut le père de la tribu des judéens. Sa descendance reçu la prédominance et la royauté (Ge. 49 : 8-12). David et Jésus-Christ furent ses descendants directs (Mt 1 : 1-16). Les judéens s'installèrent au sud de Canaan (Jos. 15 : 1-12). Après le schisme, Juda désigna aussi le nom du royaume du sud composé de la tribu de Juda et celle de Benjamin (1 R. 12 : 16-24). Ses habitants furent déportés à Babylone après la destruction du temple et de Jérusalem sous Nebucadnetsar (2 R. 25).

\DicoEntry{ISCAARIOT}\textit{, de l'hébreu « Yehuwdah » : « Yahweh sera loué »}\newline
Il fut l'un des douze disciples de Jésus-Christ (Lu. 6 : 16). Il était le gérant de la trésorerie (Jn. 12 : 4-6) et trahit le Seigneur (Mt. 26 : 14-16). Il regrettera amèrement son acte et se suicida (Mt. 27 : 3-5).

\DicoEntry{JUDE}\textit{, de l'hébreu « Yehuwdah » : « Yahweh sera loué »}\newline
Frère de Jésus-Christ et un des 12 apôtres (Lu. 6 : 16 ; Ac. 1 : 13). Jude est l'auteur d'une épître qui porte son nom (Jud. 1).

\DicoEntry{JUDEE}\textit{, du grec « Ioudaia » : « il sera loué ou louange »}\newline
Région située au sud de la Palestine où se trouvent Jérusalem (Mt. 3 : 5) et Bethléem, lieu de naissance de Jésus-Christ (Mt 2 : 1). Elle correspond approximativement au territoire de l'ancien royaume de Juda. Plusieurs assemblées se trouvèrent en Judée (Ac. 9 : 31 ; Ga. 1 : 22).

\DicoEntry{JUPITER}\textit{}\newline
dieu romain, il est assimilé à Zeus chez les grecs. Lors d'une guérison miraculeuse à Lystres, la foule pensa voir en Paul la réincarnation de Mercure et en Barnabas celle de Jupiter (Ac.14 : 5-13). Ces derniers refusèrent avec véhémence cette adoration.

\DicoEntry{JUSTIFICATION }\textit{« dikaiôsis »}\newline
Acte de disculper un individu des charges qui pèsent sur lui. Jésus-Christ, dans sa mort, accompli notre justification (Ro. 3 : 23-28)

\DicoEntry{KORE }\textit{ou Coré, de l'hébreu « Qorach » : « chauve »}\newline
Homme de la tribu de Lévi qui se révolta contre Dieu et Moïse. Le jugement du Seigneur tomba sur lui (No. 16 : 1-35).

\DicoEntry{LANGUES }\textit{(glôssa ; a donné : glose, glossaire)}\newline
La langue est l'image de la parole, utilisée par les hommes pour communiquer entre eux. Jacques enseigne que la langue peut souiller l'homme, et il exhorte le chrétien à la contrôler (Ja. 3 : 2-18). Des langues de feu descendirent sur les cent vingt (Ac. 1 : 12-15) et l’Esprit les ayant saisis, ils parlèrent des langues étrangères (Ac. 2 : 1-12). Il existe un don des langues humaines (1 Co. 12 : 10 ; 28). et un don des langues célestes (1 Co. 13 : 1). Les langues des hommes disparaîtront (1 Co. 13 : 8).

\DicoEntry{LAODICEE}\textit{}\newline
Ville à l'ouest de l'Asie Mineure où se trouvaient des chrétiens (Col. 2 : 1). L'assemblée de Laodicée se réunissait dans la maison de Nymphas (4:15-16). Elle est la dernière des sept assemblées à qui est adressée une lettre dans l'Apocalypse (Ap. 3 : 14-22.).

\DicoEntry{LEMEC}\textit{, de l'hébreu « Lemek » : « puissant »}\newline
Fils de Caïn. Rempli d'orgueil, il devint en outre le premier polygame de l'histoire (Ge. 4 : 16-24).

\DicoEntry{LEPRE}\textit{}\newline
Maladie cutanée contagieuse, dont le virus peut généralement se développer dans tout le corps ; commune en Egypte et en Orient.
- Dans le peuple: maladie de peau maligne (Lé. 13-14)
- Dans les vêtements: une rouille ou une moisissure (Lé. 13 : 47-52)
- Dans les maisons: un champignon ou une moisissure (Lé. 14 : 34-53)

\DicoEntry{LEVI}\textit{, de l'hébreu « Leviy » : « attachement »}\newline
Fils de Jacob (Ge. 29 : 34), il participa avec son grand frère Siméon au meurtre de Sichem, prince de Canaan (Ge. 34). Ses descendants, les Lévites, n'eurent point d'héritage en Canaan, mais habitèrent des villes en Israël et furent consacré au service de Yahweh (Jos. 13 : 14).

\DicoEntry{LEVITES}\textit{, de l'hébreu « Leviyiy » : « joint à}\newline
Descendants de Lévi, ils n'eurent point d'héritage en Canaan et furent consacrés au service de Dieu (Jos. 13 : 14 ; ). Il habitèrent dans des villes situées au sein des autres tribus (Jos. 21 : 1-42).

\DicoEntry{LOI}\textit{}\newline
Préceptes et ordonnances relatifs à la première alliance. Elle fut donné par Dieu aux hébreux. Les ordonnances de la loi ont été accomplies par Jésus-Christ (Mt. 5 : 17), la rendant obsolète. Terme souvent utilisé dans la nouvelle alliance dans le sens de principe des œuvres (avec allusion à la loi mosaïque), en contraste avec la grâce (Ro. 3 : 19-31).

\DicoEntry{LOI }\textit{(du péché).}\newline
Loi spirituelle inscrite dans la chair, elle pousse l'homme charnel à se révolter contre Dieu en commettant le péché (Ro. 7 : 15-25).

\DicoEntry{LOT}\textit{, de l'hébreu : « Lowt » : « voile, couverture »}\newline
Neveu d'Abraham, que Dieu sauva de la destruction de Sodome (Ge. 19 : 12-23) ; par contre, sa femme fut transformée en une statue de sel (Ge. 19 : 26). Lot est appelé « juste » (2 Pi. 2 : 6-8).

\DicoEntry{LUC}\textit{, du grec « Loukas » : « qui donne la lumière »}\newline
Médecin (Col. 4 : 14), auteur de l'évangile qui porte son nom et des Actes des Apôtres. Il fut en outre l'un des compagnon d'œuvre de Paul (Phm. 24).

\DicoEntry{MACEDOINE}\textit{}\newline
Province romaine située au nord de la Grèce dans laquelle Pau vint et évangélisa (Ac. 16 : 10 ; Ac. 20 : 1-3). Les chrétiens de Macédoine pourvurent aux besoins de Paul (2 Co. 11 : 9).

\DicoEntry{MADIAN}\textit{, de l'hébreu « Midyan » : « lutte, dispute »}\newline
Fils d'Abraham et de Ketura (Ge. 25 : 1-2).
Ancêtre des madianites, peuple qui habita à l'est de Canaan et au nord du désert d'Arabie. Ces derniers furent ennemis du peuple hébreu (No. 31 : 1-12).

\DicoEntry{MAGOG}\textit{, de l'hébreu « Magowg » : « territoire de montagne, qui domine »}\newline
Fils de Japheth (Ge. 10 : 2). Associée à Gog, elle correspond aussi à une région qui doit envahir Juda (Ez. 38)

\DicoEntry{MAIN}\textit{}\newline
Organe qui se situe à l'extrémité du bras, et qui est composé de 5 doigts. Elle nous permet de toucher, saisir, posséder. La main représente aussi une action, une oeuvre ou la protection (Es. 40 : 2 ; Ac. 11 : 21).

\DicoEntry{MALACHIE}\textit{, de l'hébreu « Mal`akiy » : « Mon messager »}\newline
Prophète de Yahweh, il condamna les péchés et l'hypocrisie des enfants d'Israël (Mal. 1 : 6-14). Il annonça aussi la venue de Jean-Baptise (Mal. 3 : 1-6).

\DicoEntry{MALFAITEUR }\textit{(repentant)}\newline
Homme coupable qui eût la grâce d'être crucifié en même temps et à côté de Jésus-Christ. A cause de son humilité, de sa sincérité, et de sa repentance, Jésus lui offrit le salut (Lu 23 : 16-43).

\DicoEntry{MAMON}\textit{, du grec « mammonas » : « richesses » : « oublieux »}\newline
Jésus utilise ce terme pour personifier la richesse et la cupidité et les met en opposition avec Dieu (Mat. 6 : 24 ).

\DicoEntry{MANASSE}\textit{, de l'hébreu « Menashsheh »}\newline
- Fils de Joseph (Ge. 41 : 51), il est l'ancêtre de la tribu de Manassé (Jos. 14 : 4).
- Fils d'Ezéchias, il fut le quatorzième roi de Juda. Il fut l'un des pires rois du royaume de Juda. Malgré le réveil impulsé par son père (Ezéchias), il se détourna entièrement de Yahweh et servit les dieux étrangers, entraînant le peuple dans sa déchéance (2 R. 21 : 1-9). Ses crimes furent si abominables que Yahweh prononça contre lui et Jérusalem une sévère sentence (2 R. 21 : 10-18), annonçant en filigrane la future destruction de la ville (I R. 23 : 26-27). Il régna 55 ans en Juda (2 R. 21 : 1).

\DicoEntry{MANNE}\textit{, de l'hébreu « man » : « qu'est ce que cela ? »}\newline
Nourriture que Dieu donna aux Israélites durant leur marche dans le désert (Ex. 16)

\DicoEntry{MARANATHA}\textit{, de l'araméen « maran atha » : « Le Seigneur vient » ou « Seigneur, viens »}\newline
Expression prononcée par Paul quand il s'adressa aux corinthiens (1 Co 16 : 22).

\DicoEntry{MARC}\textit{, du grec « Markos » : « une défense ou un grand marteau »}\newline
Auteur de l'évangile du même nom Son vrai nom est Jean et est le cousin de Barnabas (Ac. 12 : 12 ; Col. 4 : 10). Il se dirigea avec ce dernier vers Chypre (Ac. 15 : 37-39). Paul le mentionne comme un compagnon d'œuvre (Phm. 24).

\DicoEntry{MARIAGE}\textit{}\newline
Principe instauré par Dieu depuis le commencement (Ge. 2 : 22-24). C'est une alliance entre un homme et une femme dans le but d'accomplir le plan de Dieu.

\DicoEntry{MARIE}\textit{, de l'hébreu « Miryam » : « rébellion, obstination »}\newline
- mère de Jésus : Elle conçu, par la vertu d Saint-Esprit, Jésus homme (Es. 7 : 14, Lu. 1 : 26-38 ; Mt. 1 : 18-25). Elle fut averti de son ministère, et Jésus, sur la croix, la confia à son disciple Jean (Jn. 19 : 26-27). Marie était du nombre de ceux qui persévéraient dans la prière dans la chambre haute (Ac. 1 : 13-14).
- de Magdala : Femme qui fut délivrée de sept démons (Lu. 8 : 2). Elle avait suivi Jésus pendant son ministère (Mt. 27 : 55-56). Au sépulcre de Christ, un ange lui apprit sa résurrection, et Jésus lui-même lui apparut (Mc. 16 : 1-10 ; Lu. 24 : 1-10).
- de Béthanie : Sœur de Lazare et de Marthe. Assise aux pieds de Jésus elle « choisit la bonne part » en écoutant sa parole (Lu. 10 : 38-42). Elle assista à la résurrection de Lazare (Jn 11 : 1-44).

\DicoEntry{MARTHE}\textit{, du grec « Martha » : « maîtresse, dame »}\newline
Sœur de Lazare et de Marie de Béthanie. Elle reçu Christ dans sa maison mais ce dernier lui reprocha son activisme au détriment de l'écoute de Sa parole (Lu. 10 : 38-42), Elle fut, comme sa soeur, témoin de la résurrection de Lazare (Jn. 11 : 1-44).

\DicoEntry{MATTHIAS}\textit{, de l'hébreu « Mattithyah » : « don de Dieu »}\newline
Il devint disciple du Seigneur en remplacement de Judas Iscariote, et il rejoignit les onze apôtres déjà présent (Ac. 1 : 15-26).

\DicoEntry{MEDIATEUR }\textit{mesitès ; litt. : « celui qui va au milieu », « intermédiaire »}\newline
Personne chargée de régler un conflit entre deux parties et les mettre en accord.
Christ, garant et médiateur d'une nouvelle alliance (Hé. 8 : 6) est l'unique intermédiaire et médiateur entre Dieu et les hommes (1 Ti. 2 : 5).

\DicoEntry{MELCHISEDEK}\textit{, de l'hébreu : « Malkiy-Tsedeq » : « roi de justice »}\newline
Roi de Salem et sacrificateur de Dieu qui bénit Abraham (Ge. 14 : 13-20). Il préfigure Jésus Christ qui est souverain sacrificateur a perpétuité, selon l'ordre de Melchisédek (Hé. 5 : 5-10 ; 6 : 20).

\DicoEntry{MENSONGE}\textit{}\newline
Le mensonge est un péché qui consiste à déformer la vérité ou à la dissimuler (Lé. 19 : 11). En cela, Satan est le père du mensonge (Jn. 8 : 44) et les menteurs auront droit à la même sentence que lui (Ap. 21 : 8). Il est une forme de manipulation et consiste, entre autre, à déguiser sa pensée dans l'intention de tromper.

\DicoEntry{MESOPOTAMIE}\textit{, de l'hébreux « Aram Naharayim » : « pays entre deux fleuves »}\newline
Région correspondant à l'actuel Irak. Avant son appel, Abraham vivait à Ur en Chaldée qui se trouvait au sud de la Mésopotamie.

\DicoEntry{MESSIE}\textit{, de l'hébreu « mashiyach » : « oint, celui qui est l'oint »}\newline
Voir Christ.

\DicoEntry{MICHEE}\textit{, de l'hébreu « Miykayah » : « qui est semblable à Dieu ? »}\newline
Prophète de Juda (Mi. 1 : 1), il annonça l'invasion et la destruction de la Samarie, capitale d'Israël (dix tribus du nord) par les assyriens (Mi. 1 : 6), et la destruction de Jérusalem et du temple par les babyloniens (II R. 25 : 1-21). Il prophétisa en outre la naissance de Jésus-Christ à Beethléem (Mi. 5 : 1) ainsi que le rétablissement d'Israël lors du millénium (Mi. 4 : 1-8)

\DicoEntry{MICHEL }\textit{ou MICAEL, de l'hébreu « Miyka'el » : « qui est semblable à Dieu ? »}\newline
Archange de Dieu, il est appellé « grand chef » et « défenseur des enfants » d'Israël (Da. 10 : 13-21). Il triompha lors de son combat dans le ciel, face à Satan (Ap. 12 : 3-12) . Micaël contesta avec Satan le corps de Moïse (Jud. 9).

\DicoEntry{MILLENIUM}\textit{}\newline
Terme faisant référence au règne de mille ans de règne qu'exercera le Seigneur sur la terre après la bataille d'Harmaguédon (Ap. 20 : 2-7). Cette période est décrite comme un règne de paix et de gloire (Es. 11 et 12).

\DicoEntry{MISERICORDE}\textit{, eleos ; oiktirmos : « compassion »}\newline
Elan de compassion pour un coupable. Dieu est le Père des miséricordes (2 Co. 1 : 3) et riche en miséricorde (Ep. 2 : 4). Notre salut est dû à son unique miséricorde (Tit. 3 : 5) ; et ce caractère de Dieu doit se manifester dans le coeur du chrétien (Ro. 12 : 8).

\DicoEntry{MOAB}\textit{, de l'hébreu « Mow'ab » : « issu d'un père »}\newline
Fils de Lot, né de sa relation incestueuse avec sa fille, il donna naissance au peuple des moabites (Ge. 19 : 37). Ils s'établirent au sud-est de la mer morte et s'opposèrent plusieurs fois aux israélites (Jg. 3 : 12 ; 2 S. 8 : 2 ; Ez. 25 : 8-11).

\DicoEntry{MODALISME}\textit{}\newline
Doctrine qui conteste le dogme de la Trinité en ce qu'elle fait du Fils et du Saint-Esprit des « modes » de révélation du Père. Doctrine enseigné à Rome au début du 3ème siècle par Sabellius qui fut condamné par le pape Callixte.

\DicoEntry{MOISE}\textit{, de l'hébreu « Mosheh » : « tiré de »}\newline
Il est l'auteur des cinq premiers livres du Tanakh.
Par sa main, Dieu délivra les hébreux d'une servitude égyptienne de plus de 400 ans (Ex. 12 : 40-41) et leur fit traverser la Mer Rouge (Ex. 14 : 21-22). Moïse reçut les tables de la loi de Dieu (Ex. 24 : 12). Il marcha avec le peuple durant 40 ans dans le désert (Dt. 8 : 2 ; Ac. 7:20-43 ; Hé. 11:23-28). Il mourut dans le pays de Moab avant de rentrer en Canaan, la terre promise (Dt. 34 : 5-7)

\DicoEntry{MOISSON }\textit{(therismos ; racine ther : être chaud ; a donné : thermique)}\newline
Désigne la récolte des blés et céréales. Sous la loi, la fête des prémices avait lieu lors de la moisson (Lé. 23 : 10-11). Jésus utilise ce terme pour désigner le peuple à qui s'adresse l'évangile (Mt. 9 : 37-38). La moisson désigne le jugement de Dieu sur « les fils du diable » à la fin du monde (Mt. 13 : 33-43).

\DicoEntry{MOLOC}\textit{, MILCOM ou MALCOM, de l'hébreu Molek : « roi, conseiller »}\newline
Dieu vénéré des Ammonites à qui l'on sacrifiait des enfants brûlés vifs (1 R. 11 : 5-7 ; 2 R. 23 : 10).

\DicoEntry{MORT }\textit{du gec « thanatos »}\newline
La mort est la fin de la vie terrestre pour tout être vivant. Elle est le résultat du péché et par conséquent la séparation de l'homme d'avec Dieu (Ro. 5 : 12 ; 6 : 23). Après le jugement, elle sera jetée dans l'étang de feu avec le séjour des morts (Ap. 20 : 11-15).

\DicoEntry{MYRIAM}\textit{, de l'hébreu « Miryam » : « rébellion, obstination »}\newline
Sœur de Moïse et d'Aaron. Elle se rebella contre Moïse et fut frappée de lèpre. Elle fut par la suite guérit (No. 12).

\DicoEntry{NAHUM }\textit{de l'hébreu « Nachuwm » :« consolation, qui a compassion »}\newline
Prophète de Yahweh, il annonça la destruction de Ninive (Na. 3 : 7).

\DicoEntry{HAUT}\textit{}\newline
Elle correspond à la naissance d'en haut produite par le Père (Ja. 1 : 18; 1 Jn. 3 : 9). Une naissance d'eau, qui symbolise la Parole de Dieu pour nous purifier et façonner notre homme intérieur (Ez. 36 : 25; Jn 15 : 3; Ep. 5 : 26) et d'Esprit, nous rendant participant au Corps de Christ (1 Co. 12 : 13) et aux oeuvres de Dieu (Ep. 2 : 10).

\DicoEntry{NAZAREEN}\textit{, de l'hébreu « naziyr » : « celui qui est consacré ou voué »}\newline
Désigne un habitant de la ville de Nazareth mais aussi une personne consacrée à Yahweh ayant fait voeu de naziréat (No.6)

\DicoEntry{NAZARETH}\textit{, du grec « Nazareth » : « verdoyant, germe, rejeton »}\newline
Ville située dans la région de Galilée (Mt. 2 : 22-23). Jésus y passa son enfance.

\DicoEntry{NEPHILIM}\textit{, de l'hébreu « nephiyl » : « géant »}\newline
Etres de grandes tailles issus de l'union des fils de Dieu et des filles des hommes (Ge. 6 : 4). Une race de néphilim, enfant d'Anak, peuplaient Canaan (No. 13 : 32-33).

\DicoEntry{NEPHTHALI}\textit{, de l'hébreu « Naphtaliy » : « lutte, mon combat »}\newline
Fils de Jacob (Ge. 30 : 8), il est l'ancêtre de la tribu de Nephthali (Ge. 49 : 21).

\DicoEntry{NICODEME}\textit{, du grec « Nikodemos » : « victorieux du peuple »}\newline
Docteur de la loi qui questionna Jésus, de nuit, sur la naissance d'en haut (Jn. 3 : 1-21), prit position pour lui, l'embauma et le mit dans le sépulcre (Jn 7 : 43-53 ; 19 : 38-42).

\DicoEntry{NICOLAITE}\textit{, du grec « Nikolaites » : « destruction du peuple »}\newline
Adepte de la doctrine de Nicolas. Leur doctrine est identifiée avec celle de Balaam (Ap. 2 : 14-15) leurs oeuvres étaient haï par les saints de l'assemblée (Ap. 2 : 6).

\DicoEntry{NIL}\textit{, de l'hébreu « Shiychowr » : « sombre, noir, boueux »}\newline
Principal fleuve d’Egypte situé à l'est du pays (Jé. 2 : 18 ; Es. 23 : 3)

\DicoEntry{NINIVE}\textit{, de l'hébreu « Niyneveh » : « habitation de Ninus »}\newline
Immense ville dont les habitants se repentirent suite à la prédication de Jonas (Jon. 3 : 3 ; 4) mais qui retomba dans l'apostasie et fut détruite sous le jugement de Dieu (livre de Nahum).

\DicoEntry{NOCES }\textit{du grec « gamos »}\newline
Rite de réjouissance qui accompagne le transport de l'épouse à la maison de l'époux.
Fête du mariage,cérémonie d’épousailles

\DicoEntry{NOE}\textit{, de l'hébreu « Noach » : « repos, tranquillité »}\newline
« Homme juste et intègre » marchant avec Yahweh (Ge. 6 : 8-9). Il fut choisit par Dieu pour construire l'arche abritant les animaux de la terre avant le déluge (Ge. 6 : 13-22). Il engendra Sem, Cham et Japhet (Ge. 10 : 1). Noé est appelé prédicateur de justice (2 Pi. 2 : 5). Son nom se trouve dans la généalogie de Jésus-Christ (Lu. 3 : 36).

\DicoEntry{NAISSANCE}\textit{}\newline
voir « NAITRE D'EN HAUT »

\DicoEntry{OFFRANDE}\textit{}\newline
\DicoEntry{L'}\textit{offrande est un présent fais dans l'objectif d'honorer celui à qui il est présenté. Les Hébreux avaient plusieurs sortes d'offrandes qu'ils présentaient au temple : certaines étaient obligatoires et d'autres étaient libres. Les offrandes sont appelées aussi corban (Mc. 7 : 11).}\newline

\DicoEntry{OLIVIER}\textit{}\newline
Arbre fruitier donnant des olives. Son fruit donne une huile (Jg. 9 : 8-9). Le fruit comme l'huile, sont tous deux utilisés dans plusieurs domaines comme la médecine ou la gastronomie...

\DicoEntry{OMEGA}\textit{}\newline
Dernière lettre de l'alphabet grec désignant aussi la fin d'un chose (voir ALPHA).

\DicoEntry{ONCTION}\textit{, du grec « chrisma »}\newline
Action de oindre une personne, à l'image des sacrificateurs et des rois, ou encore un objet, pour un service donné (Ex. 30 : 22-31). Elle se rapporte aussi au Saint Esprit (Ac. 1 : 8 ; 1 Jn. 2 : 20-27; Jn. 14 : 26).

\DicoEntry{ORDINATION}\textit{}\newline
Acte qui confère, par l'imposition des mains, un des sacréments mis en place par l'église catholique.

\DicoEntry{OTHNIEL }\textit{de l'hébreu « `Othniy'el » : « Dieu est puissant »}\newline
Fils de Kenaz, et frère cadet de Caleb, il fut le premier des treize juges d'Israël.
Il tua le roi de Mésopotamie, Cuschan-Rischeathaïm, délivrant ainsi les enfants d'Israël de son joug (Jg 3 : 8-11).

\DicoEntry{OSEE}\textit{, de l'hébreu « Howshea` » : « salut, sauve »}\newline
Prophète qui, sous les ordres de Yahweh, épousa une prostituée pour mettre en évidence la situation apostate des enfants d'Israël (Os. 1 : 2). Les noms de ses trois enfants symbolisèrent l'attitude, et le ressenti de Dieu envers son peuple devenu infidèle (Os. 1 : 3-9). A travers le nom de son premier fils, Jizreel (II R. 10 : 1-11 ; Os. 1 : 4-5) il prophétisa la dispersion du royaume d'Israël (dix tribus du nord) après l'invasion assyrienne (2 R. 17).

\DicoEntry{PAIEN}\textit{}\newline
 Du grec « ethnos », désigné aussi par « Gentil », selon les traductions, il représente l'ensemble des nations, en dehors du peuple d'Israël (les non Juifs), considéré comme impur à cause de leurs pratiques et croyances (Mt. 18 : 17).

\DicoEntry{PAIX}\textit{}\newline
De l'hébreu « shalowm », la paix est un état de tranquilité, de bien-être, de repos (Lé. 26 : 6 ; No. 6 : 26). C'est aussi un fruit de l'Esprit (Ga. 5 : 22) accordé par Dieu à ceux qu'Il s'est acquis. Elle est différente de la paix qu'il y a dans le monde (Jn. 14 : 27 ; 16 : 33), elle nous donne de tenir ferme malgré les tribulations et de ne pas nous décourager. Elle est notre assurance quant aux promesses du Seigneur.

\DicoEntry{PALMIER}\textit{}\newline
Parmi les nombreux arbres bibliques, le palmier est un arbre à tronc peu ou pas ramifié. On le retrouve essentiellement dans le désert (Né. 8 : 15 ; Joë 1 : 12)

\DicoEntry{PAQUE}\textit{, de l'hébreu « pecach » : « passer outre, épargner »}\newline
Le terme désigne le salut des israélites après qu'ils aient aspergés les linteaux de leur portes avec le sang d'un agneau. Grâce au sacrifice de l'agneau, la mort ne toucha point les premiers-nés des hébreux (Ex. 12 : 1-30). Cet évènement fut par la suite célébré parmi le peuple le 14ème jour du mois (Lé. 23 : 4-5), et symbolisait le salut des âmes par le sacrifice de Jésus-Christ, « notre Pâque » (1 Co. 5 : 7-8), l'agneau parfait (Jn. 1 : 29).

\DicoEntry{PARADIS}\textit{}\newline
Du grec « paradeisos », terme qui désigne dans la nouvelle alliance un lieu de félicité et de bonheur céleste. Grâce à Christ, le brigand repentant entra dans le paradis (Lu. 23 : 43). Paul fut ravi dans le paradis où il entendit des paroles ineffables et merveilleuses (2 Co. 12 : 2-4).

\DicoEntry{PASTEUR}\textit{, de l'hébreu « ra`ah » : « berger »}\newline
Le pasteur prend soin des âmes qui représentent les brebis de Dieu.
Le Seigneur Jésus est le berger par excellence (Jn. 10 : 11-16) et par conséquent le gardien des âmes (1 Pi. 2 : 25)
\DicoEntry{C'}\textit{est aussi l'un des cinq ministères cités dans la Parole (Ep. 4 : 11).}\newline

\DicoEntry{PATMOS}\textit{}\newline
Ile grecque de la mer Egée sur laquelle l'apôtre Jean fut exilé. Il y reçut la révélation de l'Apocalypse (Ap. 1 : 9).

\DicoEntry{PAUL}\textit{, du grec « Paulos » : « petit »}\newline
Homme juif, né dans la ville de Tarse. Son zèle le poussa à persécuter violemment les saints au service de Dieu (Ac. 22 : 3-4). Sa rencontre avec Christ sur la route de Damas, changea sa destinée et passa ainsi de bourreau à prédicateur de l'Evangile (Ac. 22 : 6-17).
Devenu l'apôtre des nations, il parcourut l'Asie et l'Europe où il fut transporter comme prisonnier (Ac. 28 : 16).

\DicoEntry{PARDON}\textit{}\newline

\DicoEntry{PEAGER }\textit{(ou PUBLICAIN)}\newline
Fonctionnaire qui avait la charge de collecter les taxes et les droits de douane. Le péager, d'origine juive, était méprisé par les juifs, car il était un collaborateur de l'occupant romain et certains profitaient de s'enrichir lors des collectes (Lu. 3 : 13).

\DicoEntry{PECHE}\textit{, du grec « hamartano » : « erreur, faux état d'esprit »}\newline
Désigne tout ce qui va à l'encontre de la volonté de Dieu, de ses principes et de ses lois. Certains anges péchèrent avant la création de l'homme (voir DEMON). Le péché entra dans le monde par la transgression d'Adam et Eve (Ge. 3 ; Ro. 5 : 12). La pratique du péché conduit immanquablement à la mort (Ro. 6 : 23). La chair est conduite par la loi du péché (Ro. 7 : 15-25).
Jésus Christ s'est sacrifié afin que nous soyons libérés du péché et de la mort (1 Co. 15 : 3 ; Ro. 8 : 1-4), bien qu'il n'a jamais commis de péché (1 Pi. 2 : 21-24). Il est celui qui « ôte le péché du monde » (Jn. 1 : 29).

\DicoEntry{PENTECOTE}\textit{, du grec « pentekoste » : « le cinquantième jour »}\newline
Fête annuelle juive célébrée après la fête des Prémices (Lé. 23 : 15-22 ; Dt. 16 : 9-12). La venue du Saint-Esprit annoncée par Jésus (Jn. 7 : 37-39 ; 16 : 7-11) survint lors de la Pentecôte (Ac. 2 : 1-21).

\DicoEntry{PHARAON}\textit{, de l'hébreu « Par`oh » : « grand palais »}\newline
Titre donné aux rois égyptiens durant l'antiquité (Ge. 37 : 36 ; 41).

\DicoEntry{PHARISIEN}\textit{, du grec « Pharisaios » : « séparé »}\newline
Parti religieux et politique d'une grande influence, très attaché aux coutumes et traditions juives (Mc. 7 : 1-13). Les pharisiens croyaient en la resurrection des morts et aux anges (Ac. 23 : 6-9). Ils s'attachaient à une forme de piété mais cachaient leurs vices et leur orgueil par leur dévouement hypocrite envers le Seigneur (Mt. 23 : 23 ; Lu. 18 : 9-4 ; Mc. 6 : 1-6).

\DicoEntry{PHILADELPHIE }\textit{(amour fraternel)}\newline
Ville de Lydie en Asie Mineure. Elle fut construite par le roi de Pergame, et plusieurs fois plus ou moins détruite par des tremblements de terre. Une des sept lettres aux assemblées d'Asie lui est adressée (Ap. 1 : 11 ; 3 : 7-13) ; les chrétiens de Philadelphie étaient d'une fidélité remarquable ; ils avaient gardé la Parole de Dieu (v. 8, 10).

\DicoEntry{PHILEMON}\textit{, du grec « Philemon » : « attentionné, qui embrasse »}\newline
Chrétien qui reçut, dans sa maison, l'assemblée de Colosse (Phm. 1-2) et qui recueilli l'esclave Onésime (v. 17).

\DicoEntry{PHILIPPE}\textit{, du grec « Philippos » : « aimant les chevaux »}\newline
- Homme de Bethsaïda, il fut l'un des douze apôtres de Jésus (Mt. 10:3 ; Mc. 3 : 18 ; Lu 6 : 14). A son tour, il parla de Jésus à Nathanaël (Jn 1 : 44-49).
- Evangéliste, il prêcha Christ dans une ville de Samarie (Ac. 8 : 4-8) et à l'eunuque éthiopien qu'il bâptisa (v. 26-40).

\DicoEntry{PHILIPPES}\textit{, du grec « Philippoi » : « appartenant à Philippe »}\newline
Ville de Macédoine qui fut rebâtit par le roi Philippes II. Paul y alla durant un de ses voyages (Ac. 16 : 9-12)

\DicoEntry{PHILISTINS}\textit{, de l'hébreu : « Pelesheth » : « terre de ceux qui séjournent »}\newline
Peuple qui habita à l'extrême ouest de Canaan, le long de la mer méditerranée. Ils furent plusieurs fois en conflit avec les israélites (Jg. 13 à 16 ; 1 S. 17). Goliath était philistin (1 S. 17 : 4).

\DicoEntry{PHILOSOPHIE}\textit{, du grec « philosophia » litt. : « amour de la sagesse »}\newline
Doctrine poussant ses adeptes à rechercher intellectuellement la sagesse. Paul met les chrétiens en garde contre cette doctrine (Col. 2 : 8). D'ailleurs quelques philosophes épicuriens et stoïciens s'en prirent à Lui (Ac. 17 : 16-20).

\DicoEntry{PIERRE}\textit{, du grec « Petros » : « un roc ou une pierre »}\newline
Simon est son nom primitif. Il est le fils de Jonas et le frère d'André (Jn. 1 : 42-44). Pierre était un pêcheur originaire de la ville de Bethsaïda (Jn. 1 : 44). Il fut choisit comme apôtre par Jésus (Mt. 10 : 2) pour les circoncis (Ga. 2 : 7-8) et les païens (Ac. 10 et 11). On le retrouve présent, souvent accompagné de Jacques et de Jean, dans des évènements marquants de la vie du Seigneur (Mt. 17 : 1-6 ; 26 : 36-37 ; Mc. 5 : 36-38). Il écrivit deux épitres qui portent son nom. Selon la tradition, il auraité été crucifié à Rome.

\DicoEntry{PILATE}\textit{, de grec « Pilatos » : « armé d'une lance »}\newline
Il est le sixième gouverneur romain de la Judée (Lu. 3 : 1). Ponce Pilate est connu pour s'être lavé les mains de la crucifixion de Jésus Christ (Mt. 27 : 24) .

\DicoEntry{PREDESTINER }\textit{proorizô : « déterminer à l'avance »}\newline
Vouer quelqu’un d'avance à faire quelque chose. Les croyants ont été prédestiné à l'adoption, en Jésus-Christ (Ep. 1 : 5), à lui ressembler (Ro. 8 : 29). Ils ont été ainsi mis à part pour accomplir ce pourquoi Dieu les a appelé.

\DicoEntry{PROPHETE}\textit{, de l'hébreu « nabiy' » : « l'homme qui parle, un prophète »}\newline
Le prophète est le porte-parole de Dieu. Revêtu du Saint-Esprit, il communique la pensée de ce dernier aux hommes.
Son ministère consiste donc à parler au nom du Seigneur afin d'édifier, d'encourager et de consoler (1 Co. 14 : 3-4).
La prophétie est aussi un don de Dieu attribuée par le Saint-Esprit (1 Co. 12 : 4-10)

\DicoEntry{PROPITIATION}\textit{}\newline
Acte servant à apaiser, à rendre favorable pour soi.
Dieu s'est rendu propice au travers du sacrifice expiatoire de Christ.
Jésus-Christ est donc la propitiation pour nos péchés permettant aux hommes qui croient en lui, de trouver grâce devant Yahweh (1 Jn 2 : 2 ; 4 : 10 ; Hé. 2 : 17).

\DicoEntry{PROPITIATOIRE}\textit{}\newline
De l'hébreu « kapporeth », le propitiatoire est un couvercle de l'arche surmonté, à chaque extrémités, par un chérubin et Yahweh se manifestait entre les deux (Ex. 25 : 17-22 ; Lé 16 : 2 ; Hé. 9 : 5). Une fois par an, le souverain sacrificateur entrait dans le lieu très saint et aspergeait le propiatoire de sang expiatoire pour la purification des péchés d'Israël (Lé. 16 : 14-17).

\DicoEntry{PYTHON}\textit{}\newline
Dans la Mythologie grecque, le python était un serpent ou un dragon.
Dans le testament de Jésus, il représente un esprit de divination ou démon (Ac. 16 : 16-19).

\DicoEntry{RABBI}\textit{, du grec « rhabbi » : « maître »}\newline
Chez les juifs, on appelait rabbi les chefs spirituels (docteurs de la loi) qui enseignaient la Parole. Les disciples appelaient Jésus ainsi (Mc 11 21 ; Jn 9 : 2). Lui-même a exhorté la foule et les docteurs de la loi à ne point se donner cette marque de disctinction car seul Yawheh est notre maître et nous sommes tous au même niveau (Mt. 23 : 8).

\DicoEntry{RACHEL}\textit{, de l'hébreu « Rachel » : « agnelle ou brebis »}\newline
Fille de Laban et cousine de Jacob dont elle devint l'épouse (Ge. 29 : 10 ; Ge. 29 : 28). Rachel était stérile (Ge. 29 : 31) mais Yawheh lui fit grâce de deux garçons Joseph et Benjamin (Ge. 35 : 24). Elle mourut à l'accouchement du cadet (Ge. 35 : 16-19). Plus tard Jérémie prophétisera sur la descendance de Rachel (Jé. 31 : 15) qui s'accomplira à la naissance de Jésus, au temps d'Hérode (Mt. 2 : 16-18).

\DicoEntry{RAHAB}\textit{, de l'hébreu « Rachab » : « large, spacieux, tumultueux »}\newline
Prostituée habitant Jéricho, elle cacha les deux espions juifs chez elle (Jos. 2 : 1). Josué lui laissa la vie sauve lorsqu'il assiégea la ville avec le peuple (Jos. 6 : 17). En faisant preuve de bienveillance envers les espions juifs, elle fut sauvée par sa foi et justifiée par ses oeuvres (Hé. 11 : 31 ; Ja. 2 : 25). Elle habita ensuite au milieu d'Israël (Jos. 6 : 25) et engendra Boaz ; elle fit parti de la généalogie de Jésus-Christ (Mt 1 : 5-16).

\DicoEntry{REBECCA}\textit{, de l'hébreu « Ribqah » : « ensorcelante, qui prend au piège »}\newline
Petite nièce d'Abraham, elle devint l'épouse d'Isaac ( Ge. 22 : 23 ; Ge. 24 : 67). Stérile (Ge. 25 : 21), Isaac implora Yahweh de donner des enfants à sa femme, ce qu'Il fit en lui accordant Esaü et Jacob (Ge. 25 : 21-26), symboles de deux nations.

\DicoEntry{RECONCILIATION}\textit{}\newline
Restaurer une relation avec quelqu'un, s'accorder avec son adversaire (Mt. 5 : 25). Jésus est mort à la croix pour réconcilier l'homme avec Dieu (Ro. 5 : 11), laissant à l'église la mission de réconcilier l'homme pêcheur avec Dieu en lui annonçant l'oeuvre rédemptrice de Jésus à la croix (2 Co.5 : 18).

\DicoEntry{REDEMPTION}\textit{, de l'hébreu : « peduwth » et du grec: « apolutrosis » : « rachat »}\newline
Racheter au prix d'une rançon. Jésus-Christ a racheté tout homme de la mort éternelle par le sacrifice de son sang à la croix (Ro. 3 : 24). Oeuvre rédemptrice ultime pour la rémission des péchés permettant de réconcilier l'homme avec Dieu (Col. 1 : 14 ; Ep. 1 : 7 ; Hé. 9 : 12) et d'accéder à la vie éternelle.

\DicoEntry{REFORME}\textit{, de l'hébreu « yatab » : « agir bien » et du grec « katorthoma » :« une action droite », « diorthosis » « remettre droit, restaurer »}\newline
Changer sa façon d'être et/ou de faire dans un but de s'améliorer et de bien agir. Revenir et pratiquer les voies justes et droites du Seigneur (Jé. 7 : 5 ; Jé. 26 : 13). Jésus-Christ est la figure du temps de réformation, de restauration (Hé. 9 : 10).

\DicoEntry{REPENTANCE}\textit{, du grec « metanoia » : « changement de pensée, douleur d'avoir offensé Dieu »}\newline
Profonde tristesse et douleur d'avoir offensé Dieu par ses péchés (2 Co. 7:9), elle est provoquée par la bonté du Seigneur lui-même (Ro. 2:4) et entraîne un changement de pensée et d'attitude pour faire ce qui est bon et juste à ses yeux (Ac. 26:20). Jean-Baptiste baptisait du baptême de repentance (Mc. 1:4 ; Ac. 13:24).

\DicoEntry{RESURRECTION}\textit{, du grec « anastasis » : « ressuciter de la mort, se lever ».}\newline
Action de revenir à la vie après avoir été mort. Christ est réssucité des morts (Mt. 28:6 ; 1 Co. 15:12 ; 1 Pi. 1:3) et il est le premier à l'avoir été (Ap. 1:5). Les chrétiens qui sont morts ressuciteront lors de l'enlèvement de l'église (1 Co. 15:52 ; 1 Th. 4:16).

\DicoEntry{REVEIL}\textit{}\newline
Prise de conscience personnelle sur sa condition de péché et du besoin de la grâce de Dieu dans sa vie (Ep. 5:14). Cette prise de conscience est également collective sur le temps de Dieu (Jonas 3). Action de sortir de son sommeil pour accomplir la volonté du Seigneur.

\DicoEntry{ROBOAM}\textit{, de l'hébreu « Rechab`am » : « un peuple est agrandi »}\newline
Fils et successeur de Salomon. Son intransigeance à ne pas vouloir baisser les impôts conduisirent les dix tribus du nord à se séparer du reste du royaume (I R. 12:1-24). Il fut le premier roi du royaume de Juda rassemblant la tribu de Juda et de Benjamin (1 R. 12:21) pendant dix-sept ans à Jérusalem (I R. 14:21). Durant tout son règne il fut en guerre avec Jéroboam, roi des dix tribus du nord (1 R. 15 : 6) et fit ce qui est mal aux yeux de Dieu (I R. 14:21-24).

\DicoEntry{ROMAIN}\textit{, du grec « rhome » : « force »}\newline
Ils dominaient sur les juifs à l'époque de Jésus-Christ. Les juifs en avaient peur pensant que s'ils croyaient en Jésus-Christ, les romains détruiraient leur ville et leur nation (Jn. 11 : 48). Les deux nationalités juives et romaines de Paul lui permirent de recouvrer la liberté à Philippes (Ac. 16:35-39) ou encore d'échapper au fouet (Ac. 22:25-29 ; 23:27 ; 25:16).


\DicoEntry{ROME}\textit{, du grec « rhome » : « force »}\newline
Point de mire de l'empire romain située en Italie dont elle est la capitale, Rome jouit d'une grande notoriété à cette époque. Paul eut le désir d'y témoigner l'évangile (Ac. 19:21) ce que le Seigneur lui confirma dans un songe (Ac. 23:11). En route il rencontra un couple Aquilas et Priscille qui durent quitter Rome sous les ordres de l'empereur Claude (Ac. 18:1-2) et avec qui il se lia d'amitié. Après quelques temps il parti pour Rome en tant que prisionnier (Ac. 24:23) et y annonça l'évangile.


\DicoEntry{DIEU}\textit{, du grec « basileia » : « pouvoir royal, royaume des cieux »}\newline
Etat céleste gouverné par Dieu où règne son amour incommensurable et inaltérable (Ro. 14:17) Se repentir sincèrement de ses péchés en croyant au fils unique de Dieu, Jésus-Christ et naître d'en haut (Jn. 3:3-5 : voir NAITRE D'EN HAUT) sont les conditions sine qua non pour devenir citoyen des cieux. Jésus représentait le royaume de Dieu lors de son séjour sur terre (Lu. 11:17-20 ; Lu. 17:20-21), partout où il allait il prêchait le royaume de Dieu avec ses disciples (Lu. 9:1-2). Actuellement nous sommes sous le temps de la grâce depuis la mort de Jésus-Christ sur la croix, et en ces temps de la fin, c'est à l'église, corps de Christ, qu'il a été donné la mission de prêcher le royaume de Dieu en chassant les démons, guérissant les malades...

\DicoEntry{RUBEN}\textit{, de l’hébre « … » : « voici un fils »}\newline
Premier fils de Jacob (Ge. 29:32), ce dernier n'eut pas la prééminence car il alla avec Bilha, l'une des femmes de son père (Ge. 49:3-4). Il sauva Joseph de la main de ses frères qui voulaient le tuer (Ge. 37:20-22). Il est le père de la tribu des rubénites qui s'installa à l'est de la terre promise (Jos. 13:8-12).

\DicoEntry{RUTH}\textit{, de l'hébreu « Ruwth » : « amitié, une amie »}\newline
Femme moabite ayant épousé un juif, à la mort de celui-ci, Ruth suivit sa belle-mère Naomi à Bethléhem (Ru. 1:1-19). Elle devint la femme de Boaz (Ru. 4:13) et intégra la généalogie de Jésus-Christ (Mt. 1:5-16).

\DicoEntry{SABBAT}\textit{, de l'hébreu « shabbath » et du grec « sabbaton » : « repos, cessation d'activité »}\newline
Jour de repos prescrit dans la loi de Moïse par Yahweh pour son peuple où aucune activité ne devait être pratiquée (Ex. 20:8-11 ; De. 5:12-15). C'est le septième et dernier jour de la semaine (Ex. 20:11). Il symbolise une alliance perpétuelle entre Yahweh et Israël (Ex. 31:16-17) s'étendant à toutes les personnes présentent chez un juif (De. 5:14) et rappelle au verset 15 la délivrance de la servitude en Egypte du peuple Juif par Yahweh. Le sabbat s'oppose aux traditions de l'époque où il était normal de travailler sans repos. Aujourd'hui, le sabbat est l'image du repos accordé à celui qui croit en Jésus-Christ (Hé. 4:8-11 ; 1 Pi. 5:6-7) qui en est le maître (Mc. 2:23-28)

\DicoEntry{SACRIFICATURE}\textit{}\newline
Service effectué au tabernacle puis au temple, sous la responsabilité des Lévites au temps de la loi de Moïse, tribu consacrée à cet effet (Ex. 38:21). Elle consistait notamment à offrir des sacrifices à Yahweh pour l'expiation des péchés (Lév. 1). Depuis le sacrifice de Jésus à la croix, elle concerne chaque personne née d'en haut qui s'offre au Seigneur (Ro. 12:1) et qui pratique de bonnes oeuvres par la foi (TIt. 3:8 ; Ep 2:10 ; Ja 2:14-18).

\DicoEntry{SADDUCEEN}\textit{, du grec « saddoukaios » : « les justes »}\newline
Parti religieux juif en désaccord avec les pharisiens. Ils étaient incrédules face à la résurrection des morts, ainsi qu'à l'existence des anges et des esprits (Mt. 22:23 ; Ac 23:1-10). Ils furent aussi en opposition avec Jésus-Christ qui réfuta leur thèses (Mt 16:6-12).

\DicoEntry{SAINT}\textit{, de l'hébreu « qodesh » et du grec « hagios » : « consacré, mis à part »}\newline
Celui qui se met à part et qui se consacre au Seigneur après avoir reconnu l'oeuvre rédemptrice de Jésus-Christ à la croix. Il est justifié et possède la vie éternelle (1 Co. 6:11 ; Jn. 3:16). Le saint est par définition sanctifié, il refuse le péché et se laisse purifier par le Saint-Esprit (1 Th 4:1-8 ; Hé 12:14).

\DicoEntry{ESPRIT}\textit{voir étymologie du mot saint. Esprit de l'hébreu « ruwach » : « vent, souffle, esprit » et du grec « pneuma » : « vérité, inspiration, souffle, vents, âme »}\newline
Appelé aussi consolateur (Jn. 14:16-17), il convainc du justice, de jugement et de péché (Jn. 16:7-11). Il se trouve au milieu du peuple de Dieu (Es. 63:11) et en tout homme ayant accepté Jésus comme Seigneur et Sauveur dans sa vie (Ro. 8:9 ; 1 Co. 3:16, 12:13 ; Ep. 1:13). Il enseigne le chrétien et le guide dans sa marche quotidienne (Jn. 14:26 ; Lu. 12:12 ; Ga. 5:16). Le Saint-Esprit régénère et sanctifie (1 Co. 6:11). C'est lui qui accorde des dons en vue d'un ministère et de l'édification des saints (1 Co. 12:4-11 ; 1 Pi. 4:10).

\DicoEntry{SALOMON}\textit{, de l'hébreu « Shelomoh » : « paix, pacifique »}\newline
Fils de David, il succéda à son père après la mort de ce dernier. Son règne dura quarante-ans en Israël (1 R. 11 : 42). Il construisit le premier temple de Yahweh sur le mont Morija à Jérusalem (2 Ch. 3 : 1 ; 1 R. 5 à 7) et fut un grand bâtisseur (1 R. 9 : 15-28). Réputé pour sa grande sagesse (1 R. 4 : 29-34), il fut l'auteur des livres cantique des cantiques et l'ecclésiaste. Il rédigea également certains psaumes (Ps. 72 ; 127) et proverbes (Pr. 25 à 29), . Malheureusement, son goût prononcé pour le luxe et les femmes le détournèrent de Yahweh et il servit d'autres dieux (1 R. 11 : 1-13).

\DicoEntry{SALUT}\textit{, de l'hébreu «yeshuw`ah » et du grec « soteria » : « délivrance »}\newline
Il consiste à délivrer l'homme de la condamnation de ses péchés par la mort de Jésus-Christ à la croix (Ro. 8 : 1 ; 1 Th. 5 : 9). Il s'obtient par la repentance et la foi en Jésus-Christ, qui est un don de Dieu, et non par les oeuvres de l'homme (Ro. 5 : 1 ; Tit. 3 : 4-6 ; Ep. 2 : 5 ; 8). C'est une grâce imméritée du Seigneur qui nous aime (Jn. 3 : 16).

\DicoEntry{SAMARIE}\textit{, de l'hébreu « Shomerown » : « montagne de guet »}\newline
Actuellement située en Cisjordanie, cette ville fut fondée par Omri roi d'Israël (1 R. 16 : 23-24) . Elle était l'ancienne capitale du royaume du nord (2 R. 3 : 1), Osée le prophète dénonça l'idolâtrie de ces habitants sous le règne de Jéroboam II (Os. 7). La ville fut assiégée par le roi d'Assyrie, Salmanasar (2 R. 18 : 9). Plus tard, les chrétiens s'y réfugièrent après avoir été victimes de persécutions à Jérusalem (Ac. 8 : 1) et l'évangile s'y propagea.

\DicoEntry{SAMARITAINS}\textit{}\newline
Après l'assujettissement de la Samarie par Salmanasar, roi d'Assyrie (2 R. 17 : 3 ; 18 : 9), des peuples étrangers s'y établirent et servirent leur dieux au lieu de Yahweh (2 R. 17 : 24-29). Au IVe siècle av J.-C., les samaritains construisirent un temple sur le mont Garizim, qui devint le centre religieux de Samarie, entrainant une séparation avec le reste des juifs qui adoraient à Jérusalem (Jn. 4 : 20). Le temple fut détruit par le roi hasmonéen Jean Hycran I au IIe siècle ap J.C. Les samaritains étaient considérés par certains comme des étrangers et non comme de véritables juifs du fait de la mixité de leur religion. Jésus-Christ déstigmatisa leur mauvaise réputation dans une de ses paraboles (Lc. 10 : 30-37).

\DicoEntry{SAMSON}\textit{, de l'hébreu « Shimshown » : « petit soleil »}\newline
Fils de Manoach, de la tribu de Dan, il fut juge en Israël pendant vingt-ans. L'ange de Yahweh apparut à sa mère et lui annonça la naissance de l'enfant (Jg. 13 : 1-3). Ce dernier fut consacré dès le sein maternel (Jg. 13 : 4-5). Doté d'une force surhumaine (Jg. 14 : 6 , 18-19 ; 15), il fut destiné à libérer Israël de la domination des Philistins (Jg. 13 : 5) cependant son amour pour les femmes, notamment Delila, le conduisit à sa perte (Jg. 16).

\DicoEntry{SAMUEL}\textit{, de l'hébreu « Shemuw'el » : « entendu , ou exaucé de Dieu »}\newline
Prophète du Tanak, juge et sacrificateur, Samuel était un homme de foi (Hé. 11 : 32) qui servit Yahweh dès son plus jeune âge conformément au voeu de sa mère. Il fut juge toute sa vie et vécu le début de la royauté car c'est lui qui oignit Saül puis David comme roi (1 S. 10 : 1 ; 1 S. 15 : 1 ; 1 S : 16 : 13 ; Ac. 13 : 20). De par sa notoriété, l'apôtre Pierre y fait référence dans l'un de ses discours après la pentecôte (Ac. 3 : 24).

\DicoEntry{SANCTIFICATION}\textit{, du grec « hagiasmos » : « consécration, sainteté, purification »}\newline
Action de se mettre à part, de purifier un objet ou une personne pour un service bien précis devant Dieu (Ex. 40 : 9-13 , 28:41 ; 1 Th. 4 : 7). Devenir esclave de la justice est l'effet de la consécration (Ro. 6 : 19), elle s'opère par l'action de la parole et de l'esprit de Dieu (Jn. 17 : 17 ; 2 Th. 2 : 13). Voir saint

\DicoEntry{SANCTUAIRE}\textit{, du grec « hagion » : « digne de vénération, mis à part pour Dieu, services et offrandes, pur, sans péché, droit, saint »}\newline
Lieu sacré et consacré à Yahweh (Ex. 20 : 24-26). Le sanctuaire terrestre, dont Moïse avait reçu les ordonnances (Ex. 25 : 8-9 ; Hé. 9 : 1-10), est un symbole de celui qui se trouve au ciel et où Jésus y a présenté son sang (Hé. 9 : 11-24).

\DicoEntry{SANHEDRIN}\textit{, du grec « sunedrion » : « conseil, tribunal »}\newline
Situé à Jérusalem à l'époque greco-romaine, le sanhédrin désigne à la fois les petits tribunaux communaux et le grand conseil de la nation juive. Il était composé de 70 membres depuis Moïse (No. 11 : 16-17) sélectionnés parmi l'élite hiératique de la société juive de l'époque et également composé de scribes et d'anciens. Le souverain sacrificateur (Mt. 5 : 22 ; 26 : 59 ; Jn. 11 : 47 ; Ac. 5 : 21-41 ; 6 : 12-15) en assurait la direction sans pour autant prendre la décision finale d'une peine de mort qui était soumise au gouverneur romain (Jn. 18:31).

\DicoEntry{SARAI }\textit{ou SARA, de l'hébreu « Saray » ou « Sarah » : « princesse, femme noble »}\newline
Femme d'Abraham (Ge. 12 : 5 ; 17 : 15-16). Elle enfanta Isaac à l'âge de 90 ans (Ge. 18 : 9-15) selon la promesse de Yahweh et donna ainsi une postérité à Abraham (Hé. 11 : 11).

\DicoEntry{SARDES}\textit{, du grec « sardeis » : « rouge »}\newline
Capitale antique de la Lydie, Sardes se situait sur la rivière Pactole ; à environ 50 km au sud de Thyatire et 75 km à l’est de Smyrne. Réputée riche et puissante en raison de ses ressources en or, ces épithètes étaient sournois car sa forteresse reposait sur un sol boueux. En effet, au VIè siècle avant Jésus-Christ, Cyrus Le Grand –vainqueur de Crésus alors roi de Lydie - s’empara de Sardes par une attaque nocturne. Par la suite la ville subit plusieurs invasions puis un tremblement de terre en 17 ap. J.-C.
L’église de Sardes fut probablement fondée par Paul au cours d’un voyage à Ephèse. Au moment où ils reçurent le message de l’ange, il semblerait que certains chrétiens de Sardes étaient retournés au culte licencieux de Cybèle « déesse mère et gardienne des savoirs ». Ceux qui s’étaient gardé purs devaient ainsi revivifier les autres membres. Cette église symbolise l’église morte.

\DicoEntry{SATAN}\textit{, de l'hébreu « Satan » : « adversaire, ennemi »}\newline
Autrefois « chérubin protecteur » (Ez 28 : 14), il se laissa séduire par le péché et se révolta contre Dieu avec un tier des anges (Ap 12 : 4). Ils furent chassés du ciel et précipités sur la terre (Ez 28 : 15-19 ; Es 14 : 12-17), Satan hait les hommes, et en particulier ceux qui s'attachent aux lois de Yahweh (Job 1 à 2). Il est l'adversaire des enfants de Dieu, qui, par Jésus-Christ ont le pouvoir de le chasser (Lu 10 : 18-19 ; Ja 4 : 7). Il sera enchaîné pendant le millénium et libéré par la suite (Ap 20 : 1-9). Son jugement est déjà établi (Jn 16 : 11) et il sera jeté dans l'étang de feu à la fin du monde pour l'éternité (Ap 20 : 10-15).

\DicoEntry{SA}\textit{ÜL, de l'hébreu « Sha'uwl » : « désiré, demandé (à Dieu) »}\newline
Israélite de la tribu de Benjamin, il fut choisi par Dieu pour être le premier roi d'Israël (1 S. 10). Néanmoins il se détourna de Yahweh en lui désobéissant à plusieurs reprises (1 S. 13 ; 15) ; de ce fait le Seigneur lui retira la royauté (1 S. 13 : 14 ; 1 S. 15 : 10-11). Durant ses quarante-ans de règne (Ac. 13:21) Saül tenta plusieurs fois d'assassiner David du fait de sa renommée (1 S. 18 : 8-16 ; 19 : 8-17 ; 23 : 14), sans succès puisque le Seigneur avait résolu d'oindre David comme roi à la place de Saül (1 S : 16 : 13).

\DicoEntry{SCEAU}\textit{, du grec « sphragizo » : « cachet, bague pour sceau, sceller »}\newline
Marque d'authentification, d'appartenance et de protection (Ap. 9 : 4 ; Ap. 7 : 3), souvent représenté par des signes apposés sur un document dont ils prouvent la validité. Au sens spirituel, le chrétien nouvellement converti est scellé du Saint-Esprit témoignant de son apparteance à Dieu (Ep. 1 : 13). Par ailleurs, les chrétiens de Corinthe étaient le sceau de l'apostolat de Paul (1 Co. 9 : 2).

\DicoEntry{D'ISRAEL}\textit{}\newline
Le schisme fut la séparation d'Israël en deux royaumes (1 R. 12 : 16-24). En 931 av. J.-C. Roboam fils et successeur du roi Salomon ne voulut pas alléger les impôts, cela entraîna la séparation du royaume en deux ; le royaume d'Israël dirigé par Jéroboam (appelé aussi royaume du Nord ou royaume de Samarie), composé des dix tribus du nord (Ruben, Gad, Aser, Nephtali, Siméon, Issacar, Zabulon, Dan, Ephraïm, Manassée) et le royaume de Juda gouverné par le roi Roboam et composé des deux tribus du sud (Benjamin et Juda).

\DicoEntry{SCRIBE}\textit{, de l'hébreu « caphar » : « compter, dénombrer, relater » et du grec « grammateus » : « homme instruit dans la loi mosaïque, dans les écritures sacrées, enseignant »}\newline
Ils avaient la connaissance des textes sacrés et jouaient un rôle important auprès du peuple à qui ils enseignaient la loi (Esd. 7 : 6, 11). Ils occupaient également une fonction non négligeable au sein de la justice juive à savoir le sanhédrin (Mt. 16 : 21).

\DicoEntry{SECTE}\textit{, du grec « hairesis » : « action de prendre, capturer »}\newline
Groupement de personnes adhérant à une doctrine particulière et vivant à part, comme les sadducéens (Ac. 5 : 17), les pharisiens (Ac. 26 : 5), les nazaréens (Ac. 24 : 5). Mais également tous les faux docteurs ayant une opinion totalement contraire à l'évangile de paix essayant de s'infiltrer au milieu des chrétiens pour ravir leur foi et les entraîner dans la dissolution (2 Pi. 2 : 1). Tite recommande aux chrétiens de s'éloigner de ces hommes provoquant des divisions (Tit. 3 : 10).

\DicoEntry{SEDECIAS}\textit{, de l'hébreu « Tsidqiyah » : « Yahweh est justice »}\newline
Oncle de Jojakin, et frère de Jojakim, il fut le vingtième et le dernier roi de Juda. Son nom initial, Matthania (« don de Dieu ») fut changé en Sédécias par Nebucadnetsar, roi de Babylone (2 R. 24 : 17). Il ne servit pas Yahweh (2 R. 24 :19) et connut un destin tragique : ses fils furent égorgés devant lui, Nebucadnestar lui creva ensuite les yeux (Jé. 39 : 6-7), Jérusalem et le temple furent détruits et enfin, il fut emmené captif avec le peuple à Babylone (2 R. 25 : 1-21). Sédécias fut contemporain du prophète Jérémie qui lui parla plusieurs fois de la part de Yahweh (Jé. 21 ; 22 : 1-9 ; 37 à 38).

\DicoEntry{SEIN }\textit{d'Abraham}\newline
Image désignant le séjour des âmes attendant la résurrection des morts (Lc. 16 : 22).

\DicoEntry{MORTS }\textit{de l'hébreu « she'owl » : « Scheol, monde souterrain, tombe, enfer, fosse »}\newline
Endroit céleste comprenant d'une part un lieu de tourment et de souffrance extrême de l'âme d'un mort. Il est destiné à ceux qui ont vécu dans le péché durant leur vie terrestre et qui n'y ont pas renoncé. D'autre part, on y trouve le « sein d'Abraham » (Lc. 16 : 19-31) où séjournaient les âmes justes et qui ont rejoint le Seigneur depuis la résurrection de Jésus-Christ (Ep. 4 : 8).

\DicoEntry{HADES}\textit{, du grec « hades » : « dieu des profondeurs de la terre »}\newline
Egalement appelé séjour des morts, lieu de tourment temporaire où vont les âmes des méchants à leur mort. Le séjour des morts est à mettre en opposition avec le sein d'Abraham où se trouvent les rachetés dans l’attente de la résurrection. La mort et le séjour des morts seront jetés dans l'étang de feu après le jugement. Voir Mt. 11 : 23 ; Lc. 16 : 19-31 ; Ac. 2 : 27 et Ap. 20 : 14.

\DicoEntry{SEM}\textit{, de l'hébreu « Shem » : « nom, renommée »}\newline
Fils aîné de Noé (Ge. 10 : 1) et ancêtre d'Abraham (Ge. 11 : 10-27).

\DicoEntry{SENEVE}\textit{}\newline
Plante des régions orientales donnant un condiment, la moutarde. C'est la plus petite de toutes les semences (Mt. 13 : 32-33) pouvant grandir jusqu' à trois mètres de haut. Spirituellement, cela sous-tend une foi minuscule comme ce grain de moutarde mais pouvant avoir de grands effets dans le Royaume de Dieu (Mt. 17 : 20).

\DicoEntry{SEPHORA}\textit{, de l'hébreu « Tsipporah » : « petit oiseau, moineau »}\newline
Fille de Jéthro appelé aussi Réuel, femme de Moïse et mère d'Eliézer et de Guerschom. Lorsque Moïse se sauva d'Egypte (Ex. 2 :15-19), il se retira au pays de Madian. Il défenda les filles de Jéthro, sacrificateur, contre des bergers qui voulaient les empêcher d'abreuver leurs troupeaux. Jéthro le reçut chez lui, et lui donna en mariage sa fille Séphora (Ex. 2 : 21), dont il eut Eliézer et Guerschom.

\DicoEntry{SERAPHINS }\textit{du verbe hébreux « saraph » : « brûler »}\newline
Créatures célestes possédant 3 paires d'ailes, et proclamant la sainteté de Dieu (Es. 6 : 1-7).

\DicoEntry{SERPENT}\textit{}\newline
Reptile sans pattes, au corps allongé, se déplaçant dans la poussière. Le serpent est rusé (Ge. 3 : 1 ; 2 Co. 11 : 3) et désigne le diable , le serpent ancien (Ap. 12 : 9, 14, 15) qui fut maudit d'entre tous les animaux pour avoir séduit Eve dans le jardin d'Eden (Ge. 3 : 1-6, 14). Il s'oppose au serpent d'airain représentant Jésus (No. 21 : 4-9) qui a donné à L'église le pouvoir de marcher dessus c'est-à-dire de prendre autorité sur les oeuvres du diable afin qu'elles ne lui nuisent point (Lc. 10 : 19).

\DicoEntry{SERVITEUR}\textit{, du grec « diakonos » : « domestique, ministre, messager » ou encore du grec « douloo » : « esclave »}\newline
Celui qui est au service de quelqu'un. Christ a pris la forme d'un serviteur en devenant semblable aux hommes (Ph. 2 : 5-8). De même les chrétiens sont des serviteurs de Dieu, esclaves de la justice (2 Co. 6 : 4 ; Ph. 1 : 1 ; Ro. 6 : 18-19).

\DicoEntry{SETH}\textit{, de l'hébreu « Sheth » : « compensation, mis à la place »}\newline
Il compensa la perte de son frère Abel que Caïn avait tué (Ge. 4 : 25). Ce fut le troisième enfants d'Adam et Eve, ancêtre de Noé (Ge. 5 : 6-29) et de Jésus-Christ (Lc. 3 : 38).

\DicoEntry{SCHEOL}\textit{}\newline
Voir SEJOUR DES MORTS.

\DicoEntry{SILO}\textit{, de l'hébreu « Shiyloh » : « lieu de repos »}\newline
Ville située au nord est de la tribu d'Ephraïm et dans l'ancien royaume d'Israël après le schisme. C'est à Silo que les enfants d'Israël se répartirent les territoires à conquérir (Jos. 18 : 10 ; 19 : 51). Avant d'être placé à Jérusalem, l'arche de l'alliance se trouvait à Silo (1 S. 4 : 3) et Samuel y officiait en tant que prophète (1 S. 3 : 19-21).

\DicoEntry{SIMEON}\textit{, de l'hébreu « Shim`own » : « qui écoute, qui à été entendu »}\newline
1. Fils de Jacob (Ge. 29 : 33). Avec Lévi, sont petit frère, il vengea le déshonneur de sa sœur Dina, en tuant Sichem, prince de Canaan, son père Hamor, et tous leur hommes (Ge. 34). Il fut gardé en otage lorsqu'en Egypte, Joseph voulut éprouver la sincérité de ses frères (Ge. 42 : 21-38). Il est le père de la tribu des siméonites qui s'installèrent au sud de Canaan (Jos. 19 : 1-9).

2. Homme de foi à qui le Saint-Esprit avait promis qu'il ne mourrait pas sans avoir vu le Messie. Il vit l'enfant Jésus avec ses parents à Jérusalem et prophétisa (Lc. 2 : 25-35).

\DicoEntry{SIMON}\textit{, de l'hébreu « Simon » : « qui écoute, qui entend »}\newline
1. Simon le Zélote, faisait parti du groupuscule des zélotes avant d'être avec les apôtres (Lc. 6 : 13-16).

2. Nom originel de l'apôtre Pierre (Jn. 1 : 40-42).

3. Simon de Cyrène fut obligé d'aider Jésus à porter la croix (Mt. 27 : 32).

4. Magicien de la ville de Samarie qui fut baptisé et cru pouvoir s'acheter à prix d'argent la puissance du Saint-Esprit (Ac. 8 : 9-24).

\DicoEntry{SION}\textit{, de l'hébreu « Siy'on » : « un lieu élevé »}\newline
Sion est une montagne appelé aussi mont Hermon (De. 4 : 48). Situé à Jérusalem, il désigne la ville en elle-même (1 R. 8 : 1) ainsi que la montagne de l'Eternel (Es. 2 : 3 ; Joë. 3 : 17). Les Jébusiens y avaient bâti une citadelle dont David s'empara pour en faire la cité éponyme (2 S. 5 : 6-7). Dans la nouvelle alliance, la montagne de Sion est l'image de la Jérusalem céleste (Hé. 12 : 22).

\DicoEntry{SISERA}\textit{, de l'hébreu « Ciycera' » : « déploiement, champs de bataille »}\newline
Chef de l'armée du roi cananéen Jabin (Jg. 4 : 2), son armée fut vaincu par Barak (Jg. 4 : 15) et Sisera fut tué par Jaël, femme de Héber le Kénien (Jg. 4 : 17-22).

\DicoEntry{SMYRNE}\textit{, du grec « Smurna » : « myrrhe » voir défintion Guylaine}\newline
Cité de la côte occidentale de l’Asie Mineure, Smyrne (aujourd’hui Izmir) était située au nord d’Ephèse et réputée pour sa splendeur et ses richesses. Ses forteresses et ses tours de l’acropole évoquaient une couronne. Très unie à Rome, des cultes en l’honneur du dieu Zeus, de la déesse Cybèle, ou encore de l’empereur Tibère et sa mère Julie y étaient célébrés.
Proche d’Ephèse, l’église de Smyrne fut probablement le fruit du travail apostolique de Paul. En proie à ces doctrines impies, l’église de Smyrne était fortement persécutée aussi bien par les romains que par « les faux juifs » membres « d’une synagogue de Satan ». Sa persévérance face aux afflictions lui permit de recevoir un bon témoignage par l’ange de l’apocalyse. Elle incarne l’église persécutée.

\DicoEntry{SODOME}\textit{, de l'hébreu « Cedom » : « qui brûle »}\newline
Ville située dans la plaine du Jourdain (Ge. 13 : 10) où Lot s'installa après s'être séparé d'Abraham (13:12). Ses habitants étaient de grands pêcheurs devant Yahweh (Ge. 13: 13) à un tel point qu'Il détuisit la ville en faisant tombé du ciel du feu et du soufre. Il épargna Lot, sa femme qui devint une statue de sel et ses deux filles à cause d'Abraham son serviteur (Ge. 19 : 1-29).

\DicoEntry{SOPHONIE}\textit{, de l'hébreu « Tsephanyah » : « Yahweh a caché, protégé »}\newline
Prophète de Yahweh au temps du roi Josias (So. 1 : 1), il annonça le jugement de plusieurs nations païennes (So. 2 : 4-15 ; 3 : 8) et le rétablissement d'Israël (So. 3 : 14-20).

\DicoEntry{SACRIFICATEUR}\textit{, de l'hébreu « kohen » et du grec « hiereus » :« prêtre, intendant principal, ministre d'état »}\newline
Le sacrificateur, ou prêtre était un ministre attitré du culte. Il servait le Seigneur dans le sanctuaire et enseignait la loi au peuple. Il consultait Yahweh pour la communauté au moyen de l'urim et du thummim (Ex. 28 : 30 ; Esd. 2 : 63) tel un médiateur entre le peuple et Dieu. Exerçant sa fonction jusqu'à sa mort (No. 35 : 25), il devait une fois par an, le jour de l'expiation, entrer dans le Saint des Saints, et offrir des sacrifices d'animaux pour ses propres péchés et pour ceux du peuple (Hé. 9 : 6, 7, 25). Par la suite, Jésus-Christ est devenu souverain sacrificateur devant Dieu en s'offrant comme victime expiatoire et en présentant son sang dans le lieu très saint céleste. (Hé. 9 : 12 ; Hé. 7 : 25-28 ; Hé. 4 : 14-16).

\DicoEntry{SYNAGOGUE}\textit{}\newline
Lieu de culte juif (Mt. 4 : 23), où se rendait le peuple le jour du sabbat ( Mc. 6 : 2). Jésus y prêcha de nombreuses fois ainsi que ses disciples (Mt. 9 : 35 ; Ac. 14 : 1).

\DicoEntry{TABERNACLE}\textit{, de l'hébreux « mishkan » : « sanctuaire, demeure, habitation »}\newline
Maison de Yahweh du temps de l'Exode qu'on appelait aussi tente d'assignation (Ex. 39 : 32). Elle fut construite selon le modèle que Dieu donna à Moïse (Ex. 25 : 9). Elle devait être mobile afin que le peuple puisse la transporter avec eux dans le désert pendant les jours de marche, (No. 1 : 51) qui avaient lieu lorsque la nuée (Yahweh) s'élevait du dessus du tabernacle (Ex. 40 : 36, Ex. 40 : 38). Les Lévites en assuraient le service avec tous les ustensiles qui lui étaient dédié (No. 1 : 50 ; 1 Ch. 6 : 48). Une fête porte ce nom, elle est célébrée en l'honneur de Yahweh et durait sept jours (Lé. 23 : 34).

\DicoEntry{TANAKH }\textit{: (voir intro)}\newline

\DicoEntry{TEMPLE}\textit{, du grec « hieron » et au sens restreint « naos » désignant «le sanctuaire » (voir illustration)}\newline
Lieu érigé en l'honneur de l'Eternel ( 1 S. 3 : 3) ou d'une divinité païenne (1 Ch. 10 : 10). Pour remplacer le tabernacle selon le désir du roi David et suivant l'ordre de Dieu, Salomon, son fils, constuisit le temple dédié à Yahweh (2 S. 7 ; 1 R. 6). Il fut détruit une première fois par les babyloniens au 6ème siècle av. J.-C. (2 R. 25 : 9). Reconstruit lors du retour d'exil des juifs (Esd. 6 : 15), il fut complètement restauré sous le règne d'Hérode le Grand (Jn 2:20), puis de nouveau détruit en 70 par les romains. Le temple revêt dans la nouvelle alliance un tout autre édifice, en effet, nous y apprenons que le corps du croyant est le temple du Saint-Esprit (1 Co. 6 : 19) et que de ce fait, l'ensemble des chrétiens c'est-à-dire l'église corps de Christ forme un temple saint dans le Seigneur (Ep. 2:21).

\DicoEntry{TENEBRES}\textit{, du grec « skotia, skotos »}\newline
Représentant l'obscurité (1 Th. 5 : 5), les ténèbres symbolisent le péché (Ge. 1 : 2 ; Ro. 13 : 12) ainsi que l'ignorance concernant la Parole de Dieu (Os. 4 : 6) qui nous amènent en enfer. Dans la genèse, Dieu sépare la lumière des ténèbres (Ge. 1 : 4) or la lumière n'est autre que l'Eternel lui-même (2 S. 22 : 29) et les ténèbres représentent le séjour des morts (Job. 17 : 13), les abîmes (2 Pi. 2 : 4). Par extansion, on constate que les ténèbres sont semblables à une prison, elles retiennent les gens captifs (Ps. 107 : 10) les rendant aveugles (1 Jn. 2 : 11). Les ténèbres sont réservées pour les anges déchus, satan et tous ceux qui se repaissent du mal (Jud. 1 : 6-13).

\DicoEntry{TEREBINTHE}\textit{, de l'hébreu « 'elah » : « térébinthe ou chêne »}\newline
Arbre robuste assimilé au chêne, son ombrage est agréable de par sa grandeur (Os. 4 : 13 ; 2 S. 18 : 9). Il servait de lieu de culte (Ge. 35 : 4 ; Jg. 6 11 , 19 ; 1 Ch. 10 : 12 ; Os. 4 : 13 ; Es. 57 : 5). Une vallée porte ce nom, c'est à cet endroit que David tua Goliath (1 S. 17 : 2, 21 : 9).

\DicoEntry{THESSALONIQUE}\textit{, de l'hébreu « Thessalonike » : « victoire de ce qui est faux »}\newline
Autrefois appelée Therme ou Therma,qui signifie « source chaude », Thessalonique reçut son nouveau nom par Cassandre l’un des successeurs d’Alexandre le Grand (356 av. J.-C. - 323 av. J.-C.) en l’honneur de sa femme Thessalonike, sœur de ce dernier. Cette ville est située au Nord de la Grèce actuelle, sur la côte de la mer Egée. Elle jouissait d’une importante fréquentation puisqu’elle figurait parmi les trois ports principaux de la Méditerranée et se situait sur l’une des plus grandes routes commerciales de l’époque : la Voie Egnatienne reliant Rome à Byzance. A compléter avec intro NT

\DicoEntry{THOMAS}\textit{, de l'hébreu « Thomas » : « jumeau »}\newline
Surnommé Didyme, il était l'un des douzes disciples (Jn. 11 : 16, Mc. 3 : 18). Incrédule quant à la résurrection de Jésus (Jn. 20 : 27), il confessa la Seigneurie de ce dernier lorsqu'il le vit (Jn. 20 : 28). Il persévéra dans la prière (Ac. 1 : 13) jusqu'au baptême du Saint-Esprit.

\DicoEntry{TIMOTHEE}\textit{, du grec « Timotheos » : « qui adore, ou honore Dieu »}\newline
Fils d'une femme juive croyante et d'un père grec (Ac. 16 : 1), Timothée fut converti par Paul qui le circoncis à cause des juifs (Ac. 16 : 1-3). Il devint l'un de ses plus fidèles collaborateur, son enfant dans la foi (1 Ti. 1 : 2). Il l'accompagna à de nombreuses reprises dans ses différents voyages missionaires (Ac.18 : 5 ; 1 Co. 16 : 10). Timothée dirigea l'église d' Ephèse (1 Ti. 1 : 3) et Paul lui envoya deux lettres afin de lui donner des instructions précises pour être un bon serviteur de l'évangile. Nous apprenons dans l'épître aux hébreux que Timothée fut relâché (Hé. 13 : 23) un évènement sur lequel nous n'avons pas plus de précisions.

\DicoEntry{TITE}\textit{, du grec « Titos » : « nourrice, honorable »}\newline
\DicoEntry{D'}\textit{origine grecque, Tite fut un fidèle compagnon d'oeuvre de l'apôtre Paul (2 Co. 8 : 23). Il l'accompagna à Jérusalem (Ga. 2 : 1) et oeuvra entre autre à Corinthe (2 Co. 8 : 6), en Crète (Tit. 1 : 5) et en Dalmatie (2 Ti. 4 : 10). Il s'occupa plus particulièrement de l'église de Crète (Tit. 1 : 5).}\newline

\DicoEntry{TRIBULATION}\textit{, du grec « thlipsis » : « une pression, une oppression »}\newline
Persécution, tourment provoqué par l'annonce de l'évangile (Mc. 4 : 17), Elles sont indubitables pour entrer dans le royaume de Dieu (Ac. 14 : :22 ; 1 Th. 3 : 3) et ont pour but de façonner le chrétien à être patient (2 Co. 6 : 4), joyeux (2 Co. 8 : 2), perséverant (2 Th. 1 : 4). Ces souffrances sont une grâce du Seigneur (Ph. 1 : 29).

\DicoEntry{TRIBUNAL}\textit{, de l'hébreu « qahal » : « assembler, convoquer » et du grec « bema » : « tribune »}\newline
Lieu où l'on est jugé pour des fautes commises afin de recevoir une sentence proportionnelle à la faute perpétrée. Chaque être humain comparaîtra devant le tribunal de Christ (2 Co. 5 : 10) afin de rendre compte pour lui-même (Ro. 14 : 10-12, 2 Co. 5 : 10).

\DicoEntry{TROMPETTE}\textit{}\newline
Instrument ayant pour objectif de donner un signal (Ex. 19 : 13) publier une sainte convocation (Lé. 23 : 24), fêter des moments de joie (No. 10 : :10), signifier une victoire (No. 10 : 9), chanter des cantiques en l'honneur de Yahweh (1 Ch. 16 : 42), avertir et rassembler le peuple (Ez. 33 : 3 ; Mt. 24 : 31), signaler l'enlèvement de l'église (1 Co. 15 : 52), ressemble à une voix dans les cieux (Ap. 1 : 10).

\DicoEntry{UR}\textit{, de l'hébreu « 'Uwr » : « flamme, éclat, feu »}\newline
Ville de Chaldée située au sud de Babylone d'où fut originaire Abraham (Ge. 11 : 27-31)

\DicoEntry{ETERNELLE}\textit{}\newline
Il s'agit de la vie sans fin auprès de Yahweh. Pour avoir la vie éternelle il faut croire en Jésus-Christ le fils unique de Dieu (Jn. 3 : 16 ; Jn. 3 : 36) et persévérer à bien faire selon les voies du Seigneur (Ro. 2 : 7). C'est un don gratuit de Dieu (Ro. 6 : 23), un héritage (Tit. 3 : 7), une promesse (1 Jn. 2 : 25). La vie éternelle est l'image de Christ (1 Jn. 1 : 2 ; 1 Jn. 5 : 20).

\DicoEntry{VIGNE}\textit{}\newline
Plante cultivée pour son fruit le raisin que l'on consomme tel quel ou que l'on transforme sous forme de vinaigre ou de vin. Au sens figuré, la vigne représente Israël. (Es. 6 : 7) mais encore le royaume de Dieu, le cep quant à lui est une figure de Jésus-Christ (Jn. 15 : 1) et les sarments sont les enfants de Dieu (Jn. 15 : 2). Tout sarment qui ne porte pas de fruit est jeté au feu c'est-à-dire l'enfer (Jn. 15 : 6).

\DicoEntry{VOILE}\textit{, de l'hébreu « parokhet »}\newline
Etoffe de fin retors, bleue, pourpre et cramoisi qui servait de séparation entre le lieu saint et le lieu très saint (Ex. 26 : 31-33). Seul le souverain sacrificateur pouvait aller au-delà du voile une fois par an pour l'expiation de tous les péchés du peuple (Lé. 16 : 11-19 ; Hé. 9 : 7-8). Lorsque Jésus-Christ fut crucifié, ce voile se déchira en deux, de haut en bas (Mt. 27 : 50-51) ouvrant l'accès au lieu très saint. Jésus permis ainsi à tout être humain d'avoir un libre accès au Père (Hé. 10 : 19-20).
Le voile servait également à se couvrir la tête (Ge. 24 : 65), plus tard Paul expliqua que les cheveux étaient une gloire pour la femme car ils lui servaient de voile (1 Co. 11 : 15). Au sens figuré, le voile empêchait de comprendre la loi ce qui n'est plus le cas lorsqu'un individu se converti au Seigneur (2 Co. 3 : 14-16).

\DicoEntry{ZABULON}\textit{, de l'hébreu « Zebuwluwn » : « Habitation »}\newline
Fils de Jacob (Ge. 30 : 19-20) et l'une des douze tribu d'Israël (No. 2 : 7 ; Ap. 7 : 8).

\DicoEntry{ZACHARIE}\textit{, de l'hébreu « Zekaryah » : « Yahweh s'est souvenu »}\newline
1. Prophète et sacrificateur, fils de Bérékia et petit fils d'Iddo (Za..1 : 1 ; Esd. 6 : 14). Avec le prophète Aggée, ils assistaient Zorobabel, gouverneur de Juda et Josué, souverain sacrificateur, dans la restauratoin du temple de Yahweh au retour de la captivité des juifs (Esd. 5 : 1-2). Il annonça la venue du Messie (Za. 9 : 9-10, 12 : 10), le rétablissement d'Israël (Za. 8 : 1-8 , 20-23) et des événements de la fin des temps (Za. 9 à 14).
2. Sacrificateur de la classe d'Abia (Lc. 1 : 5) et père de Jean-Baptiste.

\DicoEntry{ZELOTE}\textit{, du grec « Zelotes » : « celui qui est zélé »}\newline
Patriotes juifs fervent défenseurs de la loi et des traditions ayant pour objectif de résister à l'invasion romaine. L'un des douze disciples, Simon en faisait partie (Lc. 6 : 15 ; Ac. 1 : 13).

\DicoEntry{YHWH}\textit{}\newline
Nom avec lequel Dieu s'est présenté à Moïse (voir commentaire)

\DicoEntry{ZOROBABEL}\textit{, de l'hébreu « Zerubbabel » : « Rejeton ou semence de Babylone »}\newline
Il était gouverneur de Juda (Ag. 1 : 14) et participa à la restauration du temple de Yahweh après le retour de la captivité du peuple juif (Esd. 3 : 2 ; Esd. 5 : 2). C'est un des ancêtres de Jésus (Mt. 1 : 13 ; Lc. 3 : 27).

\end{multicols}
